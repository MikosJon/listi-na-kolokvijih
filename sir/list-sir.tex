\documentclass[a4paper,oneside,10pt]{article}

\usepackage[slovene]{babel}    % slovenian language and hyphenation
\usepackage[utf8]{inputenc}    % make čšž work on input
\usepackage[T1]{fontenc}       % make čšž work on output
\usepackage[reqno]{amsmath}    % basic ams math environments and symbols
\usepackage{amssymb,amsthm}    % ams symbols and theorems
\usepackage{mathtools}         % extends ams with arrows and stuff
\usepackage{url}               % \url and \href for links
\usepackage{icomma}            % make comma a thousands separator with correct spacing
\usepackage{units}             % \unit[1]{m} and unitfrac
\usepackage{enumerate}         % enumerate style
\usepackage{array}             % mutirow
\usepackage[usenames]{color}   % colors with names
\usepackage{graphicx}          % images
\usepackage[all]{xy}

\usepackage[bookmarks, colorlinks=true, linkcolor=black, anchorcolor=black,
  citecolor=black, filecolor=black, menucolor=black, runcolor=black,
  urlcolor=black, pdfencoding=unicode]{hyperref}  % clickable references, pdf toc
\usepackage[
  paper=a4paper,
  top=2cm,
  bottom=2cm,
  textwidth=15cm,
% textheight=24cm,
]{geometry}  % page geomerty

\newtheorem{izrek}{Izrek}
\newtheorem{posledica}{Posledica}

\theoremstyle{definition}
\newtheorem{definicija}{Definicija}
\newtheorem{opomba}{Opomba}
\newtheorem{zgled}{Zgled}

% basic sets
\newcommand{\R}{\ensuremath{\mathbb{R}}}
\newcommand{\N}{\ensuremath{\mathbb{N}}}
\newcommand{\Z}{\ensuremath{\mathbb{Z}}}
\renewcommand{\C}{\ensuremath{\mathbb{C}}}
\newcommand{\Q}{\ensuremath{\mathbb{Q}}}
\newcommand{\F}{\ensuremath{\mathcal{F}}}
\newcommand{\PP}{\ensuremath{\mathcal{P}}}
\newcommand{\TT}{\ensuremath{\mathbb{T}}}
\newcommand{\V}{\ensuremath{\mathbb{V}}}
\newcommand{\Mon}{\ensuremath{\textrm{Mon}}}

% linearna orgrinjača
\newcommand{\LL}{\ensuremath{\mathcal{L}}}

% vectors
\newcommand{\vv}{\vec{v}}
\newcommand{\vu}{\vec{u}}
\newcommand{\vr}{\vec{r}}
\newcommand{\vn}{\vec{n}}
\newcommand{\va}{\vec{a}}
%\newcommand{\vb}{\vec{b}}
% \newcommand{\vc}{\vec{c}}
\newcommand{\vt}{\vec{t}}
\newcommand{\vf}{\vec{f}}
\newcommand{\vF}{\vec{F}}
\newcommand{\vE}{\vec{E}}
\newcommand{\ve}{\vec{e}}
\newcommand{\vx}{\vec{x}}
\newcommand{\vi}{\vec{\imath}}
\newcommand{\vj}{\vec{\jmath}}
\newcommand{\vS}{\vec{S}}
\newcommand{\vw}{\vec{w}}
\newcommand{\vom}{\vec{\omega}}
\newcommand{\vzeta}{\vec{\zeta}}
\newcommand{\er}{\vec{e}_r}
\newcommand{\ef}{\vec{e}_{\varphi}}
\newcommand{\et}{\vec{e}_{\vartheta}}

% greek letters
\let\oldphi\phi
\let\oldtheta\theta
\newcommand{\eps}{\varepsilon}
\renewcommand{\phi}{\varphi}
\renewcommand{\theta}{\vartheta}

% vector operators
\newcommand{\grad}{\operatorname{grad}}
\newcommand{\rot}{\operatorname{rot}}
\renewcommand{\div}{\operatorname{div}}
\newcommand{\lap}{\Delta}

% transpose
\newcommand{\T}{\ensuremath{\mathsf{T}}}
\renewcommand{\sl}{\ensuremath{\operatorname{sl}}}

% partial derivatives
\newcommand{\dpar}[2]{\ensuremath{\frac{\partial #1}{\partial #2}}}
\newcommand{\dpr}[1]{\dpar{#1}{r}}
\newcommand{\dpt}[1]{\dpar{#1}{t}}
\newcommand{\dpx}[1]{\dpar{#1}{x}}
\newcommand{\dpy}[1]{\dpar{#1}{y}}
\newcommand{\dpz}[1]{\dpar{#1}{z}}
\newcommand{\dpth}[1]{\dpar{#1}{\theta}}
\newcommand{\dpfi}[1]{\dpar{#1}{\varphi}}

% total derivatives
\newcommand{\dd}[2]{\ensuremath{\frac{d #1}{d #2}}}
\newcommand{\ddr}[1]{\dd{#1}{r}}
\newcommand{\ddt}[1]{\dd{#1}{t}}
\newcommand{\ddx}[1]{\dd{#1}{x}}
\newcommand{\ddy}[1]{\dd{#1}{y}}
\newcommand{\ddz}[1]{\dd{#1}{z}}
\newcommand{\ddth}[1]{\dd{#1}{\theta}}

% material derivatives
\newcommand{\D}[1]{\ensuremath{\frac{D#1}{Dt}}}
\newcommand{\Dn}[2]{\ensuremath{\frac{D^{#1}#2}{Dt^{#1}}}}
\newcommand{\cor}[1]{\ensuremath{#1^\circ}}
\newcommand{\con}[1]{\ensuremath{#1^\diamond}}

% tensors
\renewcommand{\t}[1]{\ensuremath{\underline{\underline{#1}}}}
\newcommand{\ta}{\t{a}}
\newcommand{\tl}{\t{\ell}}
\newcommand{\tF}{\t{F}}
\newcommand{\td}{\t{d}}
\newcommand{\tw}{\t{w}}
\newcommand{\tI}{\t{I}}
\renewcommand{\tt}{\t{t}}

% lists with less vertical space
\newenvironment{itemize*}{\vspace{-6pt}\begin{itemize}\setlength{\itemsep}{0pt}\setlength{\parskip}{2pt}}{\end{itemize}}
\newenvironment{enumerate*}{\vspace{-6pt}\begin{enumerate}\setlength{\itemsep}{0pt}\setlength{\parskip}{2pt}}{\end{enumerate}}
\newenvironment{description*}{\vspace{-6pt}\begin{description}\setlength{\itemsep}{0pt}\setlength{\parskip}{2pt}}{\end{description}}

\newcommand{\Title}{List SIR}
\newcommand{\Author}{Jure Slak}
\title{\Title}
\author{\Author}
\date{\today}
\hypersetup{pdftitle={\Title}, pdfauthor={\Author}, pdfcreator={\Author}, pdfproducer={\Author}, pdfsubject={}, pdfkeywords={}}  % setup pdf metadata

% \pagestyle{empty}              % vse strani prazne
\setlength{\parindent}{0pt}    % zamik vsakega odstavka
\setlength{\parskip}{10pt}     % prazen prostor po odstavku
% \setlength{\overfullrule}{30pt}  % oznaci predlogo vrstico z veliko črnine

\usepackage{titlesec}
\titlespacing*{\section}{0px}{0px}{-2px}
\titleformat*{\section}{\Large\bf}
\titlespacing*{\subsection}{0px}{0px}{-2px}
\titleformat*{\subsection}{\large\bf}

\newcommand{\lstar}{\overset{*}{\gets}}
\newcommand{\rstar}{\overset{*}{\to}}
\newcommand{\istar}{\overset{*}{\leftrightarrow}}
\newcommand{\red}{\downarrow}
\newcommand{\st}{\operatorname{st}}
\newcommand{\rez}{\operatorname{Rez}}
\newcommand{\vk}{\operatorname{vk}}
\newcommand{\mst}{\operatorname{mst}}
\newcommand{\vc}{\operatorname{\text{vč}}}
\newcommand{\vm}{\operatorname{vm}}

\usepackage{algpseudocode}
\usepackage{algorithm}

\floatname{algorithm}{Algoritem}
\algnewcommand\algorithmicto{\textbf{to}}
\algrenewtext{For}[3]{$\algorithmicfor\ #1 \gets #2\ \algorithmicto\ #3\ \algorithmicdo$}


\let\oldtextbf\textbf
\renewcommand{\textbf}[1]{\oldtextbf{\boldmath #1}}

\begin{document}

\section*{Prepisovalni sistemi} % slog nastaviš zgoraj
Imejmo operacije $f_i$ mestnosti $d_i$ nad množico objektov $A$. \textbf{Konkretna predstavitev} te množice uporablja neko končno abecedo $\Sigma$ in odločljivo množico $T \subseteq \Sigma^*$ ter izračunljive operacije $g_i$ enakih mestnosti $d_i$. Poleg tega imamo vrednostno funkcijo $v\colon T \to A$, tako da je $v$ surjektivna in
da $v(g_i(t_1, \dots, t_{d_r})) = f_i(v(t_1), \dots , v(t_r))$, oz.\ da komutira diagram
\[
\xymatrix{
  T^{d_i} \ar[r]^{g_i} \ar[d]_{v^{d_i}} & T \ar[d]_{v} \\
  A^{d_i} \ar[r]^{f_i} & \ A\ .
}\]

\textbf{Ekvivalenca termov:} Za dva terma $t_1, t_2 \in T$ definiramo
$t_1\sim t_2 \iff v(t_1) = v(t_2)$.
Struktura $(A, f_i)$ je izračunljiva, če obstaja poleg pogojev zgoraj tudi tak izračunljiv epimorfizem $v$.

\textbf{Kanonska funkcija:} izbere po enega predstavnika iz vsakega ekvivalenčnega razreda. Funkcija $f$ je kanonska za $(T, \sim)$, če
$f(t)  \sim t$, $t_1 \sim t_2 \implies f(t_1) = f(t_2)$, $f$ izračunljiva.

\textbf{Redukcijska relacija: } Relacija $R \subseteq T\times T$. \\
\begin{tabular}{lll}
     oznaka & pomen & razlaga \\ \hline
     $\to$ & $R$ & redukcijska relacija \\
     $\overset{n}{\to}$ & $R^n$ & v $n$ korakih \\
     $\gets$ & $R^{-1}$ & inverz redukcijske relacije \\
     $\leftrightarrow$ & $R \cup R^{-1}$ & simetrična ovojnica \\
     $\overset{+}{\to}$ & $R^+$ & tranzitivna ovojnica \\
     $\rstar$ & $R^*$ & relf. in tranzitivna ovojnica \\
     $\istar$ & $(R \cup R^{-1})^*$ & ekvivalenčna ovojnica \\
\end{tabular} \\
Za dva elementa definiramo \textbf{imata skupnega naslednika}: $a \red b \iff a \rstar c \lstar b$. \\
Za element definiramo, da je \textbf{reduciran}: $a \red \iff \neg \exists b \in T\colon a \to b$. \\
\textbf{Prepisovalni sistem:} je par $(T, \to)$, v katerem za $\sim$ vzamemo $\istar$.

Relacija $\to$ je \textbf{Noetherska}, če ne vsebuje neskončnih redukcijskih verig. \\
\textbf{Trd:} $\to$ Noetherska $\iff \exists f \colon T \to \N \ \forall t_1, t_2 \in T\colon t_1 \to t_2 \implies f(t_1) > f(t_2)$\\
\textbf{Trd:} $\to$ Noetherska $\iff \forall t \in T\ \exists r \in T\colon t \rstar r\red$.

\textbf{Def:} Relacija $\to$ ima \textbf{enolične reducirane oblike} (ERO), če
$\forall a, b, c \in T\colon (\red b \lstar a \rstar c \red) \implies b=c$. \\
Relacija $\to$ je \textbf{konfluentna} (KON), če $\forall a, b, c \in T\colon
(b \lstar a \rstar c) \implies b \red c$. \\
Relacija $\to$ je \textbf{lokalno konfluentna} (LKON), če $\forall a, b, c \in T\colon (b \gets a \to c) \implies b \red c$. \\
Relacija $\to$ ima \textbf{Church-Rosserjevo} lastnost (CR), če $\forall a, b \in T\colon
(a \istar b) \implies a \red b$.

Velja: CR $\implies$ KON $\implies$ LKON in KON $\implies$ ERO. \\
\textbf{Newmanova lema:} Če je $\to$ Noetherska, so KON, LKON, ERO in CR ekvivalentne.

\textbf{Def: } $\to$ je \textbf{polna}, če je Noetherska in konfluentna.
Problem napolnitve relacije: radi bi razširili $\to$ tako, da bo polna, relacija $\istar$ pa se bo ohranila. \\
\textbf{Izrek: } Za vsako Noethersko $\to$ v $T$ obstaja napolnitev.
Poiščemo jo tako, da najdemo izraz, ki ima dve ereducirani obliki
in dodamo v relacijo pravilo, ki eno prevede na drugo.

To naprej (kritični pari) je nagravžno in upamo da ne bo na izpitu.

\textbf{Hint:} Kadar dokazujemo, da je zaporedje Noethersko, si lahko pogosto pomagamo s kakšno funkcijo (glej trditev pod defincijo Noetherskosti). Ker pogosto delamo na nizih števk, lahko pogosto uporabimo kar samo vrednost števila, ki nam ga zapis predstavlja (recimo za nize 0 in 1 uporabimo funkcijo, ki vrednoti dvojiški zapis). Pogosto moramo na začetek niza postaviti še 1 (0 in 000 oba kodirata 0, z dodajanjem pa dobimo 1 0 in 1 000, ki kodirata različni števili).\\
Ko testiramo enolične reducirane oblike si lahko pogosto pomagamo z ugotovitvami narave `vsa pravila ohranjajo sodost oz. lihost števila ničel'. V nekaterih primerih, moramo gledati po kakšnem modulu.\\
Če imajo pravila med sabo `prazen presek' (torej delujejo na disjunktnih podintervalih, recimo 00 in 11), iz tega sledi LKON. Če primerjamo za $P_1$ in $P_2$, nam redukcija $P_1$ in nato $P_2$ reducira v isti element kot $P_2$ in nato $P_1$, torej imata redukciji z $P_1$ oz. $P_2$ skupnega naslednika.

\section*{Polinomska aritmetika}
Vedno naj bo $K$ komutativen kolobar z $1 \neq 0$.

\textbf{Rezultanta: } Naj bosta $p(x) = \sum_{i=0}^n a_ix^i$ in
$q(x) = \sum_{i=0}^m b_i x^i$ polinoma iz $K[x]$, $a_n \neq 0$, $b_m \neq 0$, $\st p + \st q > 0$ (to velja skos). Potem je rezultanta determinanta te $(n+m) \times (n+m)$ matrike
\[
\rez(p, q) = \det
\begin{bmatrix}
      a_n & a_{n-1} & a_{n-2} & \cdots & 0 & 0 & 0 \\
      0 & a_n & a_{n-1} & \cdots & 0 & 0 & 0 \\
      \vdots & \vdots & \vdots & & \vdots & \vdots & \vdots \\
      0 & 0 & 0 & \cdots & a_1 & a_0 & 0 \\
      0 & 0 & 0 & \cdots & a_2 & a_1 & a_0  \\
      b_m & b_{m-1} & b_{m-2} & \cdots & 0 & 0 & 0 \\
      0 & b_m & b_{m-1} &  \cdots & 0 & 0 & 0 \\
      \vdots & \vdots & \vdots & & \vdots & \vdots & \vdots \\
      0 & 0 & 0 & \cdots & b_1 & b_0 & 0 \\
      0 & 0 & 0 & \cdots & b_2 & b_1 & b_0  \\
\end{bmatrix}.
\]

\textbf{Trd:} $\rez(q, p) = (-1)^{mn} Rez(p, q)$ \\
\textbf{Trd:} $n = 0 \implies Rez(p, q) = a_0^m$ \\
\textbf{Trd:} $\exists r, s \in K[x], r \neq 0, s \neq 0$, tako da $\st r < \st q, \st s < \st p$ in $\rez(p, q) = pr + qs$ \\
\textbf{Izr:} Če $K$ Gaussov ali cel: $p$ in $q$ imata nekonstanten skupni faktor $\iff \rez(p, q) = 0$.  \\
\textbf{Trd:} $K$ cel, $p, q, r \in K[x]$, $\st p + \st r > 0$, $\st q + \st r > 0$, potem velja
$\rez(pq, r) = \rez(p, r)\rez(q, r)$ in $\rez(r, pq) = \rez(r, p)\rez(r, q)$.  \\
\textbf{Trd:} $K$ cel, $p(x) = a \prod(x-\alpha_i), q(x) = b \prod (x-\beta_j)$. \\ Potem je $\rez(p, q) = a^m \prod_i q(\alpha_i) = b^n \prod_j p(\beta_j) = a^mb^n \prod_i \prod_j (\alpha_i - \beta_j)$.\\
\textbf{Lema o homomorfizmu:} $K, K_1$ cela, $\varphi\colon K \to K_1$ homomorfizem, $\varphi(\vk(p)) \neq 0$. Potem $\varphi(\rez(p, q)) = \varphi(\vk(p))^{\st q - \st \varphi (q)} \rez(\varphi(p), \varphi(q))$ in $\varphi(\rez(p, q)) = 0 \iff \rez(\varphi(p), \varphi(q)) = 0$.

Uporaba: eliminacija spremenljivk\\
\textbf{Izr:} $K$ Gaussov, $p, q \in K[x_1, ...,x_d], R = \rez_{x_d}(p, q) \in K[x_1, \ldots x_{d-1}]$, $F$ alg. zaprtje ulomkov $K$. Potem lahko spremenljivko eliminiramo \\
$\forall c \in F^d \colon p(c_1, \ldots, c_d) = q(c_1, \ldots, c_d) \implies R(c_1, \ldots, c_{d-1}) = 0$ in \\
in rešitev prenesemo nazaj: $\forall c \in F^{d-1} \colon R(c_1, \ldots, c_{d-1}) = 0 \neq \vk_{x_d}(p)(c_1, \ldots, c_{d-1})\implies \exists c_d \in F \colon p(c_1, \ldots, c_d) = q(c_1, \ldots,c_d) = 0$\\
Tako lahko namesto enačb $p(x_1, \ldots, x_d) = 0$ in $q(x_1, \ldots, x_d) = 0$ pišemo enao samo enačbo $\rez_{x_d}(p, q) = 0$. Pri tem se lahko pojavijo parazitske rešitve, če $\vk_{x_d}(p)(c_1, \ldots, c_{d-1}) = 0$.

Uporaba: če imamo algebraični števili $a$ in $b$, potem s pomočjo rezultante dobimo polinome, ki uničijo $a+b$, $ab$, $1/a$, $a^r$,
$r \in \Q^+$. Za vsako elgebraično število $a$ obstaja polinom z
racionalnimi koeficienti (in z celimi), ki ga uniči. Takemu z
minimalno stopnjo in vodilnim koeficientom 1, se reče
\textbf{minimalni} polinom. Vsak drug polinom, ki uniči $a$, je
deljiv z minimalnim.

\begin{algorithm}[!ht]
\caption{Evklidov algoritem (tudi za polinome) \newline
\textbf{Vhod:} $a, b \in K$. \newline
\textbf{Izhod:} Največji skupni delitelj $\gcd(a, b) \in K$}
\label{alg:evk}
\begin{algorithmic}[1]
\Procedure{GCD}{$a, b$}
\While{$b\neq 0$}
\State $r \gets a \text{ mod } b$
\State $a \gets b$
\State $b \gets r$
\EndWhile
\State \Return $a$
\EndProcedure
\end{algorithmic}
\end{algorithm}

\subsection*{Razstavljanje polinomov}
$K$ ima karakteristiko 0. Polinom $f \in K[x]$ želimo razstaviti na
$f = g \circ h$, $g, h \in K[x]$ in $1 < \st g,\st h < \st f$.

\textbf{Trd: } Če je $\st f$ praštevilo, razstavitev ne obstaja.  \\
Poljubno si lahko izberemo prosti člen (sicer: $g_1(x) = g(x-c), h_1(x) = h(x) + c$) in vodilni koeficient $h$ (sicer: $g_2(x) = g(x/c), h_2(x) = ch(x)$), izberemo $h(0) = 0$ in $h$ moničen.
Glej algoritem~\ref{alg:razst}.

\begin{algorithm}[!ht]
\caption{Razstavljanje polinomov (Kozen, Landau). \newline
\textbf{Vhod:} $f \in K[x]$, $f = x^{rs} + a_{rs-1} x^{rs- 1} + \cdots + a_0$, $r, s \geq 2$. \newline
\textbf{Izhod:} $h, g \in K[x]$, tako da $f = g \circ h$,
$g = x^r + b_{r-1} x^{r-1} + \cdots + b_0$, $h = x^s + c_{s-1} x^{s-1} + \cdots + c_0$,  ali ``Ni mogoče.''.}
\label{alg:razst}
\begin{algorithmic}[1]
\Procedure{Razstavi}{$f, r, s$}
\State $h_0 \gets x^s$ \Comment{Izračun $h$.}
\For{k}{1}{s-1}
    \State $h_k \gets h_{k-1} + \frac{1}{r}(a_{rs-k} - [x^{rs-k}]h_{k-1}^r) x^{s-k}$
\EndFor
\State $h \gets h_{s-1}$
\State $b_r \gets 1$ \Comment{Izračun $g$.}
\For{i}{r-1}{1} \Comment{Gaussova eliminacija za $Ub = a$.}
\State $b_i \gets a_{is} - \sum_{j=i+1}^r ([x^{is}]h^j(x))b_j$
\EndFor
\State $b_0 \gets a_0$
\State $g(x) \gets \sum_{i=0}^r b_ix^i$
\If{$f(x) = g(h(x))$}
\State \Return $(g, h)$ \Comment{Preverimo, če se izide.}
\Else
\State \Return ``Ni mogoče.''
\EndIf
\EndProcedure
\end{algorithmic}
\end{algorithm}

\subsection*{Razcep polinomov v $\Z_p$, $p$ praštevilo}
Lastnosti $\Z_p$: $a^{p-1} = 1$, $(a+b)^p = a^p + b^p$, $v(x^p) = v(x)^p$, $u' = 0 \iff \exists v \in \Z[x]\colon v(x^p) = u(x)$.

Želimo razcepiti $u$. Izračunamo $d = \gcd(u, u')$. Če je $0 < \st d < \st u$, potem je $u = d \cdot u/d$ netrivialen razcep, ki ga lahko razcepimo naprej.
Če je $\st d = \st u > \st u'$, potem $d | u'$ torej $u'(x) =0$ in
$u(x) = v(x^p) = v(x)^p$ in razcepimo $v$ naprej.
Če je $\st d = 0$, potem je $u$ brez kvadratov in uporabimo algoritem~\ref{alg:razc}.

\begin{algorithm}[!ht]
\caption{Razcep polinomov (Berlekamp). \newline
\textbf{Vhod:} $u \in \Z_p[x]$ brez kvadratov, $n = \st u \geq 2$. \newline
\textbf{Izhod:} Netrivialen razcep polinoma $u$, če obstaja.}
\label{alg:razc}
\begin{algorithmic}[1]
\Procedure{Razcepi}{$u, p$}
\State Izračunaj $Q \in \Z_p^{n\times n}$ s koeficienti danimi z
$x^{kp} = \sum_{i=0}^{n-1} q_{ik} x^k \pmod{u(x)}$.
\State Poišči bazo $\{ v^{(1)} = [1,0,\ldots,0] \cong 1 + 0x + \cdots, v^{(2)}, \ldots, v^{(r)}\}$ jedra $Q - I$ nad $\Z_p$.
\If{$r=1$}
\State \Return $u$ nerazcepen.
\Else
\For{a}{0}{p-1}
\State $d_a(x) = \gcd(u(x), v^{(2)}(x) - a)$
\If{$\st d_a > 0$}
\State \Return $(d_a(x), u(x) / d_a(x))$ je razcep.
\Comment{Če želimo vseh $r$ faktorjev, potem ne}
\EndIf
\Comment{vrnemo takoj, ampak nadaljujemo}
\EndFor
\Comment{do $p-1$ in nato naprej z $v^{(3)}$ itd.,}
\EndIf
\Comment{ dokler jih nimamo $r$.}
\EndProcedure
\end{algorithmic}
\end{algorithm}

\subsection*{Razcep polinomov nad $\Z$}
\textbf{Def:} Polinom $p = \sum_i a_i x^i$ je primitiven, če je $\gcd(a_0, \ldots, a_n) = 1$. \\
\textbf{Lema:} Produkt primitivnih je primitiven.

Ideja 1: razcepimo polinom po modulu $p$, kjer je $p$ pračtevilo večje od $2M$ in $M$ je tak, da so vsi koeficienti v razcepu po abs.\ manjši od $M$ (obstajajo ocene).

Ideja 2: Razcep po nekem manjšem modulu $p$ lahko dvignemo do razcepa
nad $p^k < M$. Tukaj uporabimo Henslov dvig.
Ko imamo faktorje nad $Z_{p^k}$. Potem gremo čez vse podmnožice faktorjev in poskusimo, če kakšen produkt deli $p$.

\section*{Rešetke}
\textbf{Celoštevilska ogrinjača} vektorjev $v_1, \ldots v_k \in \R^m$ je množica $L(v_1, \ldots, v_k) = \{ \sum \lambda_i v_i; \lambda_i \in \Z \}$. \\
Množica $\Lambda \subseteq \R^m$ je $n$-razsežna \textbf{rešetka}, če obstajajo $b_1, \ldots, b_n \in \R^m$, tako da $\Lambda = L(b_1, \ldots, b_n)$.
Množica $\{b_i\}$ je baza $\Lambda$, $m\times n$ matrika $B$ s stolpci $b_i$ pa bazna matrika.  \\
Dve rešetki $\Lambda_1, \Lambda_2$ dim.\ $n$ sta \textbf{enaki} $\iff$ obstaja $U \in \Z^{n\times n}$, da $B_2 = B_1 U$ in $\det U = \pm 1$.\\
\textbf{Determinanta rešetke} $\Lambda$ z bazo $B$ je $d(\Lambda) = \sqrt{\det(B^\T B)}$.\\
\textbf{Pravokotnost} baze meri $\delta(b_1, \ldots, b_n) = \prod \|b_i\| / d(\Lambda)$. Vedno je $\delta(\Lambda) \geq 1$. \\
\textbf{Lema: } Vsaka omejena podmnožica rešetke je končna.

\textbf{SVP} je problem iskanja najkrajšega vektorja v rešetki.
SVP v 2D je iskanje najbližjega vektorja pravokotni projekciji:
$(u, v) \to (u, v-ku)$, optimalni $k$ je $\left[ \frac{\langle u, v \rangle}{\langle v, v \rangle} \right]$, kjer je $[x]$ najbližje celo število $x$. Od tod sledi algoritem~\ref{alg:60}. Posplošitev je algoritem~\ref{alg:theta}.

\begin{algorithm}[!ht]
\caption{Algoritem $60^\circ$ za iskanje najkrajšega vektorja v rešetki dimenzije 2. \newline
\textbf{Vhod:} Baza $(u, v) \in \R^m$. \newline
\textbf{Izhod:} Nova baza $(u, v)$, da je $\cos\measuredangle(u, v) \leq \frac{1}{2}$. Krajši izmed obeh je najkrajši vektor v rešetki.}
\label{alg:60}
\begin{algorithmic}[1]
\Procedure{60deg}{$u, v$}
\Repeat
\State \textbf{swap}$(u, v)$
\State $k \gets \left[ \frac{\langle u, v \rangle}{\langle v, v \rangle} \right]$
\State $u \gets u - kv$.
\Until{$\|v\|^2 \leq \|u\|^2$}
\State \Return $(u, v)$.
\EndProcedure
\end{algorithmic}
\end{algorithm}

\begin{algorithm}[!ht]
\caption{Algoritem $\theta^\circ$. \newline
\textbf{Vhod:} Baza $(u, v) \in \R^m$ in $t \in [1, 2)$. \newline
\textbf{Izhod:} Nova baza $(u, v)$, da je $\cos\measuredangle(u, v) \leq \frac{t}{2}$.}
\label{alg:theta}
\begin{algorithmic}[1]
\Procedure{thetadeg}{$u, v, t$}
\Repeat
\State \textbf{swap}$(u, v)$
\State $k \gets \left[ \frac{\langle u, v \rangle}{\langle v, v \rangle} \right]$
\State $u \gets u - kv$.
\Until{$\|v\|^2 \leq t^2\|u\|^2$}
\State \Return $(u, v)$.
\EndProcedure
\end{algorithmic}
\end{algorithm}

\textbf{Def:} $(b_1,\ldots, b_n)$ baza rešetke $\Lambda$. Za \textbf{Gram-Schmidtovo bazo} $(b_1^*, \ldots, b_n^*)$ velja $b_i^* \perp b_j^*$ (za $i \neq j$) in $\LL(b_1, \ldots,b_i) = \LL(b_1^*, \ldots, b_i^*)$ za $i = 1, \ldots, n$. \\
\textbf{Trd:} $(b_1,\ldots, b_n)$ baza rešetke $\Lambda$ potem je $\forall v \in \Lambda \colon \min\limits_{1 \leq i \leq n} \|b_i^*\| \leq \|v\|.$ \\
\textbf{Def:} $(b_1,\ldots, b_n)$ baza rešetke $\Lambda$ potem je $b_j(i)$ \textbf{pravokotna projekcija vektorja} $b_j$ vzoldž $\LL(b_1^*, \ldots, b_i^*)$ na $\LL(b_i, \ldots, b_n)$.\\
\textbf{Def:} Baza rešetke je \textbf{$t$-reducirana}, če velja $|\mu_{ij}| \leq \frac12$ in $\|b_i^\ast\|^2 \leq t \|b_{i+1}(i)\|^2$ za $1 \leq i \leq n$. \\
\textbf{Trd:} Baza, ki jo vrne algoritem~\ref{alg:theta} je $t$-reducirana. Za splošno dimenzionalno rešetko uporabljamo algoritem~\ref{alg:lll}.


\begin{algorithm}[!ht]
\caption{Algoritem Lenstra–Lenstra–Lovász. \newline
\textbf{Vhod:} Baza $b_1, \ldots,b_n \in \R^m$ rešetke $\Lambda$ in $t \in [1, 2)$. \newline
\textbf{Izhod:} $t$-reducirana baza $(b_1, \ldots, b_n)$ rešetke $\Lambda$}
\label{alg:lll}
\begin{algorithmic}[1]
\Procedure{LLL}{$\{b_i\}, t$}
    \State $\mu_{i,k} := \frac{\left<b_i, b_k^*\right>}{\|b_k^*\|^2}$
    \State $k \gets 1$
    \While{$k < n$}
        \State $b_{k+1} \gets b_{k+1} - [\mu_{k+1, k}]b_k$
        \If{$\|b_k^*\|^2 > t^2 \|b_{k+1}(k)\|^2$}
            \State zamenjaj $b_{k+1}$ in $b_k$
            \State $k \gets \max(k-1, 1)$
        \Else \For{j}{k-1}{1}
            \State $b_{k+1} \gets b_{k+1} - [\mu_{k+1, j}]b_k$
            \State $k \gets k+1$
        \EndFor \EndIf
    \EndWhile
    \State \Return $(b_1, \ldots, b_n)$.
\EndProcedure
\end{algorithmic}
\end{algorithm}

\section*{Vsote in rekurzivne enačbe}

\textbf{Def:} Zaporedje $a \in K^\N$ je \textbf{hipergeometrično}, če obstajata $n_0 \in \N$ in $r \in K(n)$ tako da $\forall n \geq n_0 \colon a_n \neq 0 \land \frac{a_{n+1}}{a_n} = r(n)$. Množico vseh hipergeometričnih zaporedij označimo s $\mathcal{H}(K)$.

\textbf{Izr:} $\forall r \in K(k)\ \exists a, b, c \in K[k] \colon \\
r(k) = \frac{a(k)}{b(k)}\frac{c(k+1)}{c(k)} \land
\left(\forall i \in \N_0 \colon a(k) \perp b(k+i)\right) \land
a(k) \perp c(k) \land b(k) \perp c(k+1)$

Za dano hipergeometrično zaporedje $t \in K^\N $ želimo njegovo vsoto $\sum_{k=k_0}^{n} t_k$ izraziti v zaključeni obliki.

\begin{algorithm}[!ht]
\caption{Gosperjev algoritem. \newline
\textbf{Vhod:} Hipergeometrično zaporedje $t_k$, tako da $\frac{t_{k+1}}{t_k} = r(k) \in K(k)$. \newline
\textbf{Izhod:} Hipergeometrično zaporedje $s_k$, tako da $s_{k+1} - s_k = t_k$ s.p., če obstaja.}
\label{alg:gosper}
\begin{algorithmic}[1]
\Procedure{GA}{$t$}
    \State izračunaj $r(k) \gets \frac{t_{k+1}}{t_k}$
    \State najdi $a, b, c \in K[k] \colon r(k) = \frac{a(k)}{b(k)}\frac{c(k+1)}{c(k)}$, tako da $a(k) \perp b(k+i), i = 0, 1, 2, \ldots$
    \State najdi $x(k) \in K[k]$, ki reši $a(k) x(k+1) - b(k-1)x(k) = c(k)$.
    \Comment{Lahko uporabimo poli.}
    \If{$x(k)$ ne obstaja}
        \State \Return rešitev ne obstaja
    \Else
        \State \Return $s_k = \frac{b(k-1)x(k)}{c(k)} t_k$
    \EndIf
\EndProcedure
\end{algorithmic}
\end{algorithm}

Z Gosperjevim algoritmom lahko zapišemo vsoto hipergeometričnega zaporedja v zaključeni obliki $\sum_{k=k_0}^{n} t_k = s_{n+1} - s_{k_0}$. Gosperjev algoritem nam odpre nov problem, ki je reševanje linearne rekurzivne enačbe s polinomskimi koeficienti. Tega se bomo lotili z algoritmom Poli (alg.~\ref{alg:poli}). \\
\textbf{Hint:} Včasih $s_{n+1}$ (ali kateri drugi člen) ni definiran, takrat uporabimo $\sum_{k=k_0}^{n} t_k = s_{n} - s_{k_0} + t_n$. V splošnem velja $\sum_{k=k_0}^{k_1} t_k = s_{k_1} - s_{k_0} + t_{k_1}$.

\textbf{Def:} Operator $E \colon K^\N \rightarrow K^\N$ se imenuje \textbf{operator pomika} in premakne člene zaporedja za eno mesto $(Ea)_n = a_{n+1}$. \\
\textbf{Def:} Operator $\lap = E-1$ je \textbf{diferenčni operator}, $(\lap a)_n = a_{n+1} - a_n$.\\
\textbf{Def:} Množica $L[n; E] = \left\{\sum\limits_{i=0}^{r} p_i E^i \mid \ r \in \N, p_i \in K[n] \right\}$ je kolobar linearnih rekurzivnih operatorjev (LRO) s polinomskimi koeficienti.

\begin{algorithm}[!ht]
\caption{Algoritem poli. \newline
\textbf{Vhod:} LRO $L = \sum_{j=0}^{r}q_j(n) \lap^j \in K[n;\lap]$ in polinom $f \in K[n]$. \newline
\textbf{Izhod:} Baza $B$ afinega prostora $\{x \in K[n] \mid Lx = f \}$.}
\label{alg:poli}
\begin{algorithmic}[1]
\Procedure{Poli}{$t$}
    \State $b \gets \max\limits_{0 \leq j \leq r}(\st q_j - j)$
    \State $p(z) \gets \sum\limits_{0 \leq j \leq r}\vk(q_j) z^{\underline{j}}$
    \State $\alpha \gets \max(\{k \in \N\rvert\ p(k) = 0\} \cup \{-1\})$
    \State $d \gets \max(\st f - b, \alpha)$
    \If{$d < 0$}
        \If{$f=0$}
            \State \Return $\{0\}$
        \Else
            \State \Return $\{\}$
        \EndIf
    \Else
        \State \Return $B = \{x \in K[n] \mid \ \st x \leq d, Lx = f\}$
        \Comment z metodo nedoločenih koeficientov
    \EndIf
    \Comment za $c_i$ in nastavkom $x(n) = \sum_{i=0}^d c_i n^i$
\EndProcedure
\end{algorithmic}
\end{algorithm}

\textbf{Def:} $F\colon \N\times\N \to K$ je dvorazsežno hipergeomertijsko zaporedje, če obstajajo $p_1, p_2, q_1, q_2 \in K[n, k] \setminus \{0\}$, tako da $p_1(n, k)F(n+1, k) = q_1(n, k)F(n, k)$ in $p_2(n, k)F(n, k+1) = q_2(n, k)F(n, k)$.
\textbf{Def:} $F\colon \N\times\N \to K$ je \emph{pravo} dvorazsežno hipergeometrijsko zaporedje, če je oblike $F(n,k) = P(n,k)y^nz^k\prod_{i=0}^p(a_i n+b_i k+\alpha_i)! /  \prod_{j=0}^q(c_j n+d_j k+\beta_j)!$ kjer $P\in \C[n,k]$, $y,z\in\C^*$, $a_i,b_i,c_j,d_j\in\Z$, $p,q\in\N$, $\alpha_i,\beta_j\in\C$, tako da $a_i n +b_i k+\alpha_i$ niso negativna cela števila.

\begin{algorithm}[!ht]
\caption{Zeilbergerjev algoritem (ni nujno, da se konča, razen za \emph{pravo}). \newline
\textbf{Vhod:} Dvorazsežno hipergeometrijsko zaporedje $F(n,k)$. \newline
\textbf{Izhod:} $L \in K(n)[E_n]$ in $G(n, k)$, tako da $(LF)(n, k) = G(n, k+1) - G(n, k)$.}
\label{alg:za}
\begin{algorithmic}[1]
\Procedure{ZA}{$F$}
    \For{d}{0}{\infty}
        \State Uporabi GA nad $K(n)$ na produktu $t_k = F(n+d, k) - \sum_{i=0}^{d-1} r_i(n) F(n+i, k)$,
        kjer so $r_0, \ldots, r_{d-1} \in K(n)$ nedoločene.
        \If{GA uspe}
            \State Naj GA vrne $s_k$ in $r_0, \ldots, r_{d-1}$, potem vrni $L = E_n^d - \sum_{i=0}^{d-1} r_i(n) F(n+i, k)$, $G(n, k) = s_k$.
        \EndIf
    \EndFor
\EndProcedure
\end{algorithmic}
\end{algorithm}

Na koncu ZA moramo rešiti hipergeometrijsko LRE. Za to imamo algoritem Hiper (alg.~\ref{alg:hiper}).\\
\textbf{Hint:} Pogosto se splača čez celotno enačbo narediti $\sum_{k=0}^\infty$, saj tako dobimo hipergeometrijsko LRE za $S(n)$ (vsota $F$), pri tem, ko (npr. v primeru binomskih simbolov v $G(n,k)$) na eni strani dobimo 0. Velja tudi $\binom{0}{0}=1$ (koristno pri $S(0)$) in če $m>n$ potem $\binom{n}{m}=0$ (zato je $G(n,\infty)$ pogosto 0).

\begin{algorithm}[!ht]
\caption{Algoritem Hiper (Petkovškov algoritem). \newline
\textbf{Vhod:} $p, q, r \in K[n], pr \neq 0$. \newline
\textbf{Izhod:} Vse hipergeometrične rešitve LRE $p(n)y_{n+2} + q(n)y_{n+1} + r(n)y_n$.}
\label{alg:hiper}
\begin{algorithmic}[1]
\Procedure{Hiper}{$p, q, r$}
\ForAll{monične $a(n)\mid r(n)$ in $b(n) \mid p(n-1)$}
\State $P(n) \gets a(n+1) p(n) / b(n+1)$
\State $Q(n) \gets q(n)$
\State $R(n) \gets b(n) r(n) / a(n)$
\State $\rho \gets \max\{\st P, \st Q, \st R\}$
\State $\alpha \gets [n^\rho]P(n)$
\State $\beta \gets [n^\rho]Q(n)$
\State $\gamma \gets [n^\rho]R(n)$
\ForAll{rešitve $z \neq 0$ enačbe $\alpha z^2 + \beta z + \gamma = 0$}
\ForAll{poli. rešitve $c(n)$ enačbe $z^2P(n)c(n+2) + zQ(n)c(n+1) + R(n)c(n)$}
\State $f(n) \gets z\frac{a(n)}{b(n)} \frac{c(n+1)}{c(n)}$
\State med vse rešitve dodaj rešitve LRE $y_{n+1} = f(n) y_n$.
\EndFor
\EndFor
\EndFor
\EndProcedure
\end{algorithmic}
\end{algorithm}

\subsection*{Znižanje reda enačbe}
Če imamo neko hipergeometrično rešitev $La = 0$, potem lahko znižamo red enačbe, tako da gremo noter z nastavkom $y_n = a_n z_n$. To nam da enačbo za $z_n$: $L'z_n = 0$. Linearni operator $L'$ zapišemo po potencah $\Delta$, prosti člen se mora pokrajšat in dobimo enačbo za $(\Delta z)_n$, ki je enega reda manjša. Potem to rešimo in dobimo rešitev $(\Delta z)_n = b_n$, od tod pa dobimo $z_n = c_1 \sum_{k=k_0}^{n-1} b_k + c_2$, $c \in K$ in od tod končno $y_n = a_n (c_1 \sum_{k=k_0}^{n-1} b_k + c_2)$.

\newpage

\section*{Računanje v polinomskih idealih}

\textbf{Oznake:} $x := (x_1, \ldots, x_n)$, $\alpha := (\alpha_1, \ldots, \alpha_n)$, $|\alpha| := \alpha_1 + \ldots + \alpha_n$, $x^\alpha = x_1^{\alpha_1}\ldots x_n^{\alpha_n}$.\\
% vm, vc, mst ze obstajajo kot komande
\textbf{Def:} Za $\alpha, \beta \in \N^n$ definiramo relacijo $\subseteq$ kot $\alpha \subseteq \beta \iff \forall i \ \alpha_i \leq \beta_i$. Velja $\alpha \subseteq \beta \implies x^\alpha | x^\beta$.\\
\textbf{Def:} Relacija $\leq$ v $\N^n$ je \textbf{monomska urejenost}, če linearno ureja elemente, so vsi elementi večji od 0 in je tranzitivna. Primeri: $\leq_{LEX}, \leq_{TLEX}, \leq_{TRLEX}$. \\
\textbf{Def:} Množica monomov $\Mon = \{x^\alpha \mid \alpha \in \N^n\}$. Z dano monomsko urejenostjo $\leq$ za $\N^n$, na $\Mon$ definiramo relacijo $\leq$ s predpisom $x^\alpha \leq x^\beta \iff \alpha \leq \beta$. \\
\textbf{Lema (DL):} $\forall M \subseteq \Mon \ \exists B \subseteq M \colon |B| < \infty \land (\forall m \in M \ \exists b \in B\colon b|m)$.

\textbf{Def:} Naj bo $\leq$ monomska urejenost na $\N^n$, $f \in K[x]\setminus \{0\}$ in pripadajoča $A \subseteq \N^n$ končna, tako da $f = \sum\limits_{\alpha \in A} c_\alpha x^\alpha$ in $\forall \alpha \in A \colon c_\alpha \neq 0$. Potem definiramo naslednje oznake: \\
\vspace{-0.5cm}
\begin{enumerate}[(i)]
    \vspace{-0.3cm}
    \item skupna stopnja $f$: $\st f := \max\limits_{\alpha \in A} |\alpha|$
    \vspace{-0.3cm}
    \item multistopnja $f$: $\mst f := \max_{\leq}A$
    \vspace{-0.3cm}
    \item vodilni koeficient $f$: $\vk_{\leq}f := c_{\mst_{\leq}f}$
    \vspace{-0.3cm}
    \item vodilni monom $f$: $\vm_{\leq}f := x^{\mst_{\leq}f}$
    \vspace{-0.3cm}
    \item vodilni člen $f$: $\vc_{\leq}f := \vk_{\leq}f \vm_{\leq}f$
\end{enumerate}
\textbf{Izr:} Naj bo $\leq$ monomska urejenost in $f_1, \ldots f_k \in K[x] \setminus \{0\}$, $g \in K[x]$. Potem obstajajo $h_1, \ldots, h_k, r \in K[x]$, tako da $g = \sum h_if_i + r$, noben člen $r_i$ ni deljiv z nodebim izmed členov $f_1, \ldots, f_k$ in, če $h_i \neq 0$, je $\mst (h_if_i) \leq \mst(g)$. Za algoritem za deljenje glej algoritem~\ref{alg:del}.

\begin{algorithm}[!ht]
\caption{Algoritem za deljenje v $K[x_1, \ldots, x_n]$. \newline
\textbf{Vhod:} $\leq$ monomska urejenost, $f_1, \ldots, f_k, g$ kot zgoraj. \newline
\textbf{Izhod:} $h_1, \ldots, h_k, r$ kot zgoraj.}
\label{alg:del}
\begin{algorithmic}[1]
\Procedure{deli}{$g, f_1, \ldots, f_k$}
\For{i}{1}{k} \State $h_i \gets 0$ \EndFor
\State $r\gets 0$
\State $p \gets g$
\While{$p \neq 0$}
    \State $i \gets 1$
    \State $uspeh \gets 0$
    \While{$i \leq k$ \textbf{ and } $uspeh \neq 1$}
        \If{$\vc{f_i} \mid \vc{p}$}
        \State $p \gets p - g \vc(p) / \vc(f_i)$
        \State $h_i \gets h_i + \vc(p) / \vc(f_i)$
        \State $uspeh \gets 1$
        \Else
        \State $i\gets i+1$
        \EndIf
    \EndWhile
    \If {$uspeh = 0$}
    \State $p \gets p - \vc(p)$
    \State $r \gets r + \vc(p)$
    \EndIf
\EndWhile
\EndProcedure
\end{algorithmic}
\end{algorithm}


\textbf{Def: } Naj bo $I \subseteq K[x]$ ideal $G\subseteq I$ končna.
Če za vsak $f \in I \setminus \{0\}$ obstaja $g \in G$, tako da $\vm(g) \mid \vm(f)$, potem je $G$ \textbf{Gr\"obnerjeva baza} za $I$.  \\
\textbf{Trd: } Naj bo $\{g_1, \ldots, g_m\}$ Gr\"obnerjeva baza za $I$.
Potem je $f \in I \iff f \mod (g_1, \ldots, g_m) = 0$ in $I = \langle g_1, \dots, g_m\rangle$. \\
\textbf{Trd:} Vsak ideal ima Gr\"obnerjevo bazo. \\
\textbf{Izr:} Vsak polinomski ideal je končno generiran. \\
\textbf{Def:} Za dva neničelna polinoma $f, g \in K[x]$ in monomsko urejenost $\leq$ naj bo $S$-polinom polinomov $f$ in $g$ polinom
$S(f, g) = \frac{m}{\vc(f)} f - \frac{m}{\vc(g)}g$, kjer je $m = \operatorname{lcm}(\vm f, \vm g)$. \\
\textbf{Izrek:} $I$ ideal v $K[x]$ z bazo $G = \{g_1, \ldots, g_k\}$.
Potem je $G$ Gr\"obnerjeva $\iff$ $\forall g_i, g_j \in G$ velja $S(g_i, g_j) \mod G = 0$. To je temeljni izrek za algoritem~\ref{alg:grob}, ki poišče Gr\"obnerjevo bazo.

\begin{algorithm}[!ht]
\caption{Buchbergerjev algoritem iskanje Gr\"obnerjeve baze. \newline
\textbf{Vhod:} $\leq$ monomska urejenost, $f_1, \ldots, f_k \in K[x]$. \newline
\textbf{Izhod:} Gr\"obnerjeva baza ideala $\langle f_1, \dots, f_k\rangle$ glede na $\leq$.}
\label{alg:grob}
\begin{algorithmic}[1]
\Procedure{Groebner}{$f_1, \ldots, f_k$}
\State $G' \gets \{f_1, \ldots, f_k\}$
\Repeat
    \State $G \gets G'$
    \ForAll{pairs $(f, g) \in G^2$}
        \State $r\gets S(f, g) \mod G$
        \If{$r \neq 0$}
            \State $G' \gets G' \cup \{r\}$
        \EndIf
    \EndFor
\Until{$G' = G$}
\EndProcedure
\end{algorithmic}
\end{algorithm}

\textbf{Def:} Gr\"obnerjeva baza je reducirana, če za vse $g \in G$ velja, da za vsak $h \in G \setminus \{g\}$ noben člen $h$ ni deljiv
z $\vc(g)$ in $\vk(g) = 1$. Vsak ideal ima enolično reducirano Gr\"obnerjevo bazo.

\hfill Avtorji: Jure Slak, Maks Kolman, Žiga Lukšič


\end{document}
% vim: syntax=tex:
% vim set: spell spelllang=sl:
% vim: foldlevel=99

% Latex template: Jure Slak, jure.slak@gmail.com
