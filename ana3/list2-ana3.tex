\documentclass[8pt,a4paper]{amsart}
% ukazi za delo s slovenscino -- izberi kodiranje, ki ti ustreza
\usepackage[slovene]{babel}
%\usepackage[cp1250]{inputenc}
%\usepackage[T1]{fontenc}
\usepackage[utf8]{inputenc}
\usepackage{amsmath,amssymb,amsfonts}
\usepackage{url}
%\usepackage[normalem]{ulem}
\usepackage{enumerate}
\usepackage[dvipsnames,usenames]{color}


\usepackage[
top    = 1cm,
bottom = 1cm,
left   = .5cm,
right  = 0.5cm]{geometry}
%
%% ne spreminjaj podatkov, ki vplivajo na obliko strani
%\textwidth 19cm
%\textheight 27cm
%\oddsidemargin-1.5cm
%\evensidemargin-1.5cm
%\topmargin-30mm
%%\addtolength{\footskip}{10pt}
%\pagestyle{plain}
%%\overfullrule=15pt % oznaci predlogo vrstico


% ukazi za matematicna okolja
\theoremstyle{definition} % tekst napisan pokoncno
\newtheorem{definicija}{Definicija}[section]
\newtheorem{primer}[definicija]{Primer}
\newtheorem{opomba}[definicija]{Opomba}
\newtheorem{zgled}[definicija]{Zgled}

\theoremstyle{plain} % tekst napisan posevno
\newtheorem{lema}[definicija]{Lema}
\newtheorem{izrek}[definicija]{Izrek}
\newtheorem{trditev}[definicija]{Trditev}
\newtheorem{posledica}[definicija]{Posledica}


\newenvironment{itemize*}%
{
\begin{itemize}
\setlength{\itemsep}{0pt}
\setlength{\parskip}{1pt}
}
{\end{itemize}}

\newenvironment{enumerate*}%
{
\begin{enumerate}
\setlength{\itemsep}{0pt}
\setlength{\parskip}{1pt}
}
{\end{enumerate}}


\newcommand{\dx}{\ensuremath{\,\mathrm{d}x}}
\newcommand{\dt}{\ensuremath{\,\mathrm{d}t}}
\let\oldint\int
\renewcommand{\int}{\oldint \!}


\newcommand{\R}{\mathbb R}
\newcommand{\N}{\mathbb N}
\newcommand{\Z}{\mathbb Z}
\newcommand{\C}{\mathbb C}
\newcommand{\Q}{\mathbb Q}

\pagestyle{empty}

\begin{document}
\thispagestyle{empty}
\setlength{\parindent}{0pt}
\section*{\textbf{Linearni sistemi}} %%%%% LINEARNI SISTEMI

\textsc{Definicija: }Naj bo $\dot{\vec{x}}(t) = A(t) \vec{x}(t)$ sistem in $A(t)
\in \R^{n \times n} \forall t$. Rešitev enačbe $\dot{X} = AX$, ki je obrnljiva,
se imenuje \textbf{fundamentalna rešitev}. Dve fundamentalni rešitvi se
razlikujeta za obrnljivo matriko. \textbf{Splošna rešitev}
$\dot{\vec{x}} = A \vec{x}$ je $X\vec{c}$, kjer je $\vec{c}$ konstantni vektor. \\

$X = [x_1,x_2,\ldots x_n] \qquad X\vec{c} = c_1\vec{x_1} + c_2\vec{x_2} + \cdots
c_n\vec{x_n}$ \\

Naj bo $A$ matrika konstant. $\dot{\vec{x}} = A \vec{x}$ ima splošno rešitev
$\vec{x} = Pe^{Jt}\vec{c}$, kjer je $A=PJP^{-1}$ jordanska kanonična forma.  \\

\textsc{Postopek:}
\begin{enumerate*}
  \item Izračunaj lastne vrednosti
    matrike $A$.
    \begin{itemize*}
      \item Če so vse lastne vrednosti različne,
        izračunaj lastne vektorje za vse lastne vrednosti in določi $J$ in $P$ (pazi,
        da vrstni red lastnih vrednosti v $J$ sovpada z vrstnim redom lastnih vektorjev
        v $P$)
      \item Če je lastna vrednost $\lambda$ večkratna in zanjo obstaja le en
        lastni vektor, izračunaj korenski vektor in ga preslikaj z $A-\lambda I$
    \end{itemize*}
  \item Zapiši rešitev $\vec{x} = Pe^{Jt}\vec{c} = c_1
    e^{\lambda_1t} v_1 + c_2 e^{\lambda_2t} v_2 + \cdots c_n e^{\lambda_nt} v_n$,
    kjer $P = [v_1,v_2,\ldots ,v_n]$. Iz tega dobiš $X = [\vec{x_1},\vec{x_2},
    \ldots, \vec{x_n}]$, kjer $x_i = e^{\lambda_it} v_i$.
\end{enumerate*}

\textsc{Opomba:} Pri kompleksnih lastnih vrednostih in vektorjih z uporabo
$e^{\lambda + i\mu} = e^{\lambda}(\cos{\mu}+i\sin{\mu})$ loči realni in
imaginarni del (hočemo realne lastne vektorje, konstante pred njimi pa so lahko
kompleksne).

% \textsc{Trditev:} Naj bo dan sistem  $\dot{\vec{x}} = A \vec{x}$, kjer $A:[a,b] \rightarrow \R^{n\times n}$ zvezna in $X_1,X_2$ fundamentalni rešitvi. Potem velja
%
% $$
% X_1 = X_2 P,
% $$
%
% za neko obrnljivo matriko $P$.
% \\
Ko rešuješ sistem
$$
\begin{bmatrix} \dot{x} \\ \dot{y} \end{bmatrix} =
\begin{bmatrix} a(t) & b(t) \\ c(t) & d(t) \end{bmatrix}
\begin{bmatrix} x \\ y \end{bmatrix},
$$

iz prve enačbe izrazi $y$, enačbo odvajaj in vstavi dobljeni $\dot{y}$ v drugo. Dobiš diferencialno enačbo za $x$, iz katere izračunaš $x$ in $y$. Splošno rešitev sistema $X$ dobiš tako, da razpišeš $x$ in $y$ po bazi konstant $C_1,C_2$ (prvi stolpec: $C_1 = 1, C_2 = 0$ in drugi stolpec: $C_1 = 0, C_2 = 1$).
\\

\textsc{Liouvilleov izrek: }Naj bo $\dot{\vec{z}} = A \vec{z}$ in $A:[a,b] \rightarrow \R^{n\times n}$ zvezna za vsak $t$. Potem obstaja fundamentalna rešitev sistema, ki je obrnljiva na $[a,b]$ $X(t)$ taka, da velja:

$$
\det{X(t)} = \det{X(t_0)}e^{\int_{t_0}^t \text{sl}A(\xi)d\xi}.
$$

\section*{\textbf{Linearne DE višjega reda}}  %%%%%%%%%%%%% LINEARNE DE VIŠJEGA REDA

$y^{(n)}+a_{n-1}(x)y^{(n-1)}+\cdots + a_1(x)y' + a_0(x)y = f(x)$

Naj bo $y$ rešitev enačbe. Potem:

$\vec{z} = \begin{bmatrix} y \\ y' \\ y'' \\ \vdots \\ y^{(n-1)} \end{bmatrix}$,
$ \quad \vec{z}' =  \begin{bmatrix} y' \\ y'' \\ y''' \\ \vdots \\ y^{(n)} \end{bmatrix}
=
\begin{bmatrix}
  0 & 1 & 0 &  \cdots & 0 \\
  0 & 0 & 1 &  \cdots & 0 \\
  \vdots & \vdots & \vdots & \ddots & \vdots \\
  0 & 0 & 0 &  \cdots & 1 \\
  -a_0 & -a_1 & -a_2 & \cdots & -a_{n-1}
\end{bmatrix}
\begin{bmatrix} y \\ y' \\ \vdots \\ y^{(n-2)} \\ y^{(n-1)} \end{bmatrix}
+
\begin{bmatrix} 0 \\ 0 \\ 0 \\ \vdots \\ f \end{bmatrix}$

$\vec{x} = X\vec{c} = X \int X^{-1}\vec{f}dt$, rabimo samo prvo vrstico ($y$).

\textsc{Linearna NDE s konstantnimi koeficienti:}
$y^{(n)}+a_{n-1}y^{(n-1)}+\cdots + a_0y = b(t)$, $a_0,\ldots ,a_{n-1} \in \R$.

Homogen del: substitucija $y = e^{\lambda x}$. Dobimo $\lambda_1, \ldots \lambda_k$ paroma
različne.

\hspace{3em} Rešitev: $y = y_1+\cdots +y_k$, kjer $y_j$ ustreza $\lambda_j$ in $y_j = c_1
e^{\lambda_j x}+c_2 x e^{\lambda_j x}+\cdots + c_{k_j} x^{k_j -1} e^{\lambda_j
x}$, kjer je $k_j$ večkratnost $\lambda_j$.

Partikularni del:
za $b$ oblike $e^{\mu x}q(x)$, $\mu \in \C$ in $q(x)$ polinom stopnje $m$.
$y_p = p(x)e^{\mu x}x^k$, kjer $k$ večkratnost $\mu$ kot $\lambda$ in $p$ stopnje $m$.

\textsc{Cauchy-Eulerjeva enačba:} $x^n y^{(n)}+a_{n-1}x^{n-1}y^{(n-1)}+\cdots +
a_1xy' + a_0 y = b$. Uvedemo novo spremenljivko $x = e^t$, enačba se prevede na
enačbo s konstantnimi koeficienti. V praksi porabimo nastavek $y = x^{\lambda}$.

\hspace{3em} Rešitev: $y = y_1+\cdots +y_k$, kjer $y_j$ ustreza $\lambda_j$ in $y_j = c_1
x^{\lambda_j}+c_2 \log(x) x^{\lambda_j}+\cdots + c_{k_j} \log^{k_j -1}(x)
x^{\lambda_j}$, kjer je $k_j$ večkratnost $\lambda_j$.

Partikularni del: za $b$ oblike $(\ln{t})t^{\mu}$. Nastavek $y_p =
q(\ln{t})t^{\mu}\ln^k{t}$, kjer $\text{st}(q)= \text{st}(p)$ in $k$ večkratnost
$\mu$ med $\lambda_i$, ki rešijo ``karakteristični polinom''.


\section*{\textbf{Variacijski račun}}

\textsc{Definicija: }Naj bo $A:X \longrightarrow Y$ linearna preslikava med
normiranima vektorskima prostoroma. $A$ je omejena, če obstaja $M \geq 0$, da
velja $\| Ax \|_Y \leq M \|x \|_X$. Če je $A$ omejena, potem $\| A \| := \sup_{x
\neq 0}{\frac{\| Ax \|}{\| x \|}} := \sup_{\| x\| \leq 1}{\| Ax \|} := \sup_{\|
x\| = 1}{\| Ax \|}$.

\textsc{Trditev:} $A$ je omejena $\Longleftrightarrow A$ zvezna. $\| Ax \|_Y
\leq \|A\| \|x\|_X$ ($\|A\|$ je norma v prostoru operatorjev)

Za Banachova prostora $(C[a,b],\|.\|_\infty)$ in $(C^1[a,b],\|.\|_1)$ je $\|f\|_1 := \|f\|_\infty + \|f'\|_\infty$

\textsc{Definicija: }Naj bosta $X$ in $Y$ Banachova, $\mathcal{U}^{odp} \subseteq X$ in $F:\mathcal{U}\longrightarrow Y$. $F$ je:
\begin{enumerate}[a)]
\item \textbf{krepko (Fréchetjevo) odvedljiva} v točki $x$, če obstaja $(DF)(x)$ omejen linearni operator iz $X$ v $Y$, da velja:
$$ \frac{\|F(x+h)-F(x)-(DF)(x)(h)\|}{\|h\|} \xrightarrow[h \to 0]{} 0 $$

\item \textbf{šibko (Gâuteauxjevo) odvedljiva}, če obstaja
$ g_x(h):= \lim_{t \to 0} \frac{F(x+th)-F(x)}{t} $
\end{enumerate}

\textsc{Izrek:} Če je $F$ krepko odvedljiva $\Longrightarrow$ $F$ šibko odvedljiva in odvoda sta enaka.

\textsc{Opomba:} Izračunamo šibkega, dobimo kandidata za krepkega. Po definiciji
izračunamo limito krepkega. Dokažemo omejenost $g_x$.

Uporabno: $| \int (f+f')dx| \leq \int (|f|+|f'|)dx \leq \int (\|f\|_\infty + \|f'\|_\infty)dx = \int \|f\|_1dx$

\textsc{Trditev:} Naj bo $X$ Banachov prostor in $F,G:X \longrightarrow \R$
krepko odvedljivi v $x_0 \in X$. Potem je $F G:X \longrightarrow \R $ krepko
odvedljiva v $x_0$ in velja $D(FG)(x_0)=F(x_0)DG(x_0)+G(x_0)DF(x_0)$

\textsc{Razmadzejev izrek:} Naj bosta $M$ in $N$ zvezni funkciji na $[a,b]$.
Denimo, da za vsako testno funkcijo $\varphi \in \mathcal{D}([a,b])$ velja:

$
\int_a^b (M\varphi + N\varphi')dx = 0.
$
Tedaj je $N$ odvedljiva in velja $N'=M$.

\textsc{Euler-Lagrangeeva enačba:} $L_y - \frac{d}{dx}L_{y'}=0$

Naj velja $L_y=\frac{d}{dx}L_{y'}$ in $L_{y'}h |_a^b = 0$ za vse dopustne variacije.
\begin{itemize}
\item če poznamo $y(a),y(b) \Longrightarrow h(a)=h(b)=0$ za vse dopustne variacije.
\item če poznamo $y(a) \Longrightarrow h(a)=0$ za vse dopustne variacije ($L_{y'}(b)=0$).
\item če poznamo $y(b) \Longrightarrow h(b)=0$ za vse dopustne variacije ($L_{y'}(a)=0$).
\item če $y(a),y(b)$ ne poznamo $\Longrightarrow$ ...
\end{itemize}

\textsc{Klasični problem:} $I[y]=\int_a^bL(x,y,y')dx$.
\begin{enumerate}
\item Če $L=L(x,y') \Longrightarrow L_{y'}=C$ (Iz te enačbe izrazi $y' = f(x)$, z integracijo izračunaj $y = \int f(c) $ in vstavi robne pogoje, da določiš konstante.)
\item $L=L(y,y') \Longrightarrow L-y'L_{y'}=C$ (ločljive spremenljivke, verjetno potrebna obravnava glede na vrednost $C$)
\end{enumerate}

\textsc{Izoperimetrični polinom:} $A:$ $I[y]=\int_a^bL(x,y,y')dx +$ robni pogoji + dodatni pogoji: $\int_a^bG_1(x,y,y')dx = A_1, \ldots \int_a^bG_n(x,y,y')dx=A_n \Longrightarrow $ obstajajo $\lambda_1,\ldots \lambda_n: \int_a^b(L-\lambda_1G_1-\cdots \lambda_nG_n)dx = \tilde{I}[y]$. Ekstremali za $A$ so ekstremali za $\tilde{I}$ z nekaterimi robnimi pogoji. (Da dobiš $\lambda $ izračunaj najprej $y (x,\lambda)$, nato pa uporabi dodatni pogoj. Dodatno: Če $y(a)$ ni podan: $\tilde{L}_{y'}(a)=0$)
\\

$I[y]=\int_a^bL(x,y,y',\ldots ,y^{(n)})dx \Longrightarrow L_y -\frac{d}{dx}L_{y'} - \cdots + (-1)^n \frac{d^n}{dx^n}L_{y^{(n)}}=0$
\\

$I[y_1,y_2,\ldots ,y_n] = \int_a^b L(x,y_1,\ldots y_n, y_1',\ldots y_n')dx$ in podani $y_1(a),\ldots y_n(a), y_1(b),\ldots y_n(b)$. Potem za $i = 1,\ldots ,n$ velja: $L_{y_i} = \frac{d}{dx}L_{y_i'}$.

\textsc{Nelinearni sistemi:} Če $\dot{x} = f(x,y), \dot{y} = g(x,y) \Longrightarrow y'(x)=\frac{g(x,y)}{f(x,y)}$, $(\frac{y}{x})\dot{ } = \frac{x\dot{y}-y\dot{x}}{x^2}$, $(xy)\dot{ } = \dot{x}y+x\dot{y}$
%
%Naloge tipa znajdi-se-sam ali bolje rečeno manage-it-on-your-own. Malo množi/deli, seštevaj/odštevaj enačbe s pravimi stvarmi in poskusi zagledati odvod produkta/kvocienta:
%
%\begin{itemize}
%\item $(xy)\dot{ } = \dot{x}y+x\dot{y}$
%\item $(\frac{y}{x})\dot{ } = \frac{x\dot{y}-y\dot{x}}{x^2}$
%\item Če $\dot{x} = f(x,y), \dot{y} = g(x,y) \Longrightarrow y'(x)=\frac{g(x,y)}{f(x,y)}$
%\end{itemize}


%%%%%% PRVI LIST


\section*{\textbf{NDE 1. reda}}
Ločljive spremeljivke: $y' = f(x)g(y)$ \\
Linearna: $y' = a(x)y + b(x)$, rešujemo $y_s = y_h + y_p$ \\
Trik: $y(x) \leftrightarrow x(y) \implies y' = 1/\dot{x}$ \\
Homogena: $f(tx, ty) = t^\alpha f(x, y)$, v posebnem $f(x, y) = f(1, x/y)
  \implies z = y/x, y' = z + xz' \implies$ linearna \\
Bernoullijeva: $y' = a(x)y + b(x)y^\alpha$, rešujemo $z = y^{1-\alpha}$,
  $\implies \frac{1}{1-\alpha}z' = a(x)z + b(x)$ \\
Ricattijeva: $y' = a(x)y^2 + b(x)y + c(x)$, ena rešitev $y_1$. Nova spr. $y = y_1 + \frac{1}{u}$\\
$y'=y_1'-\frac{u'}{u^2}$ po pretvorbi $u'=-u(2ay_1+b)-a$
\\
Integrirajoči množitelj: $Pdx + Qdy = 0$, iščemo $\mu$: $(\mu P)_y = (\mu Q)_x$.
Rešitev $u(x, y) = \int Pdx = \int Qdy = 0$ \\
\hspace*{20pt} $\mu = \mu(x) \iff \frac{\mu_x}{\mu} = \frac{P_y - Q_x}{Q}$ odvisno samo od $x$.
               $\mu = \mu(y) \iff \frac{\mu_y}{\mu} = \frac{Q_x - P_y}{P}$ odvisno samo od $y$. \\
\hspace*{20pt} Če $\mu = f(x, y)$, pazi, da odvajaš kot kompozitum.\\
Parametrično: $x = X(u, v), y = Y(u, v), y' = Z(u, v)$. Rešujemo: $dY = Z \, dX$ \\
\hspace*{20pt} Triki: $\cos^2 + \sin^2 = 1, ch^2 - sh^2 = 1, y' = tx.$\\
Clairautova: $y = xy' + b(y')$. Rešitev: $y = C x + b(C)$. Tudi singularna
rešitev (ogrinjača). \\
Lagrangeeva: $y = a(y')x + b(y')$. Rešujemo parametrično: $X = u, Z = y' = v, Y =
a(v)u + b(v) \implies$ linearna. \\
%\hspace*{20pt} Singularna rešitev: poiščemo fiksne točke $a$. Če $a(t_0) = t_0 \implies y = a(t_0) x + b(t_0)$ je singularna rešitev.\\
%Singularna rešitev: če $G(x, y, c) = 0$ splošna rešitev, sing. rešitev dobimo:
%$G(x, y, c) = 0, G_c(x, y, c) = 0$. \\
%\hspace*{20pt} Druga možnost: če $F(x, y, y') = 0$ dana enačba, sing. rešitev
%    dobimo: $F(x, y, y') = 0, F_{y'}(x, y, y') = 0$. Preveriti moramo, če rešitev res reši DE!!! \\
%Iskanje ortogonalne trajektorije družine krivulj:
%\begin{enumerate}
%  \item odvajaj enačbo krivulje (če se znebiš konstante, nadaljuj s korakom 3.))
%  \item eliminiraj konstanto iz enačbe krivulje in odvajane enačbe
%  \item v novi enačbi zamenjaj $y'$ z $-1/y'$ in reši dobljeno DE.
%  \item (Če je potrebno poljuben kot izrazi nov $y'$ iz enačbe $\tan\alpha=\frac{k_1-k_2}{1+k_1k_2}$)
%\end{enumerate}

Če je enačba podana eksplicitno in je desna stran polinom lihe stopnje v $y$ s koeficienti funkcijami v $x$ uvedeš $u=y^2$.


\section*{\textbf{NDE višjih redov}}
Ne nastopa $y$: uvedemo $z = y'$. \\
Obe strani sta odvoda nečesa: integriramo in dodamo konstanto. \\
\hspace*{20pt} Odvodi: $y'/y = (\log(y))', x y' + y = (xy)', \frac{y'' y - y'^2}{y^2} = (\frac{y'}{y})', \frac{y' x - y}{x^2} = (\frac{y}{x})'$.\\
Ne nastopa $x$: uvedemo $z(y) = y'$, $y$ neodvisna spr. $y'' = \dot{z}z$, $y''' =
  \ddot{z}z^2 + \dot{z}^2z$. \\
Homogena: $F(x, ty, ty', \dots, ty^{(n)})$ = $t^k F(x, y, y', \dots,  y^{(n)})$. Vpeljemo $z(x) = y'/y$. $y''/y = z' + z^2$.\\
Z utežjo: $F(kx, k^my, k^{m-1}y', \dots, k^{m-n}y^{(n)}) = k^pF(x, y, y', \dots,  y^{(n)})$. Uvedemo: $x = e^t, y = u(t)e^{mt}$.
%
%\subsubsection*{Geometrija}
%Tangenta v točki $(x,y)$: $Y-y=y'(X-x)$ \\
%Normala v točki $(x,y)$: $Y-y=-\frac{1}{y'}(X-x)$\\
%Ločna dolžina: $\int_a^b \sqrt{1 + y'(x)^2} \dx$ \\
%$d(T_0,p)=\frac{|ax_0+by_0+c|}{\sqrt{a^2+b^2}}$ $ \quad T_0=(x_0,y_0), ax+by+c=0$

%\begin{tabular}{ll}
%Abscisa tangente: $X = x - y/y'$ & Ordinata tangente: $Y = y - xy'$ \\
%Abscisa normale: $X = x+yy'$ & Ordinata normale: $Y = y + x/y'$
%\end{tabular}


\section*{\textbf{Integrali in formule}}
\begin{tabular}{llll}
$\int \ln{x} \dx = x \ln{x} - x + C$ & $\int \frac{1}{\sin(x)} \dx = \ln{\tan(x/2)} + C$ & $t \mapsto (\alpha (t), \beta (t), \gamma (t))$:$\int_{t_o}^{t_1}\sqrt{\dot{\alpha}^2+\dot{\beta}^2+\dot{\gamma}^2}dt$\\

$ \int x^m\log(x) \dx = x^{m+1}\left(\frac{\log x}{m+1} - \frac{1}{(m+1)^2}\right) + C$ & $ \int \frac{1}{\cos(x)} \dx = -\log(\cot(x/2)) + C$  & $G(y)=\int_0^1(y^2+y'^2)dx \Longrightarrow DG(y)(h)= \int_0^1(2yh+2y'h')dx$\\

$ \int p(x) e^{k x} \dx = q(x) e^{k x} + C$, st($q$) = st($p$) & $ \int \frac{1}{\tan(x)} \dx = \log(\sin(x)) + C$ & $\sin{x} = \frac{e^{ix}-e^{-ix}}{2i}$\\

$ \int e^{a x} \sin(b x) \dx = \frac{e^{a x} }{ a^2 + b^2} (a \sin(b x) - b \cos(b x)) + C$ & $ \int \tan(x) \dx = - \log(\cos(x)) + C$ & $\cos{x} = \frac{e^{ix}+e^{-ix}}{2}$\\

$ \int e^{a x} \cos(b x) \dx = \frac{e^{a x} }{ a^2 + b^2} (a \cos(b x) + b \sin(b x)) + C$ & $ \int x/(1 + x) \dx = x - \log(x + 1) + C$ & $\sinh{x} = \frac{e^{x}-e^{-x}}{2}$\\

$ \int \frac{1}{\sqrt{a^2 + x^2}} \dx =\text{arsh}\frac{x}{a} + C = \log|x + \sqrt{x^2 + a^2}| + C$  & $ \int x/(1 + x) \dx = x - \log(x + 1) + C$ & $\cosh{x} = \frac{e^{x}+e^{-x}}{2}$\\

$\int \frac{1}{\sqrt{a^2 - x^2}} \dx =\arcsin\frac{x}{a} + C$ & $ \int \sin^2(x) \dx = \frac{1}{2} (x - \sin x \cos x) + C$ & $\cosh^2{x}-\sinh^2{x}=1$ & $1+\tan^2{x} = \frac{1}{\cos^2{x}}$\\

$\int \frac{1}{a^2+x^2} \dx = \frac{1}{a}\arctan\frac{x}{a} + C$ & $ \int \cos^2(x) \dx = \frac{1}{2} (x + \sin x \cos x) + C$  & $\tan^2{x} = \tan'{x}-1$\\

$\sin^2(x/2) = (1 - \cos(x))/2$ & $\cos^2(x/2) = (1 + \cos(x))/2$ \\

\end{tabular}



$\displaystyle \int \frac{1}{a x^2 + bx + c} \dx =
\begin{cases}
\frac{1}{\sqrt{a}}\log|2ax + b + 2 \sqrt{a} \sqrt{ax^2 + bx + c}|+C, & a >0\\
 \frac{-1}{\sqrt{-a}} \arcsin((2ax + b)/\sqrt{D})+C, & a<0
\end{cases} $\\
$\int \frac{p(x)}{(x-a)^n (x^2 + bx + c)^m} \, \dx = A \log|x - a| + B \log|x^2 + bx + c| + C \arctan(\frac{2x + b}{\sqrt{-D}}) + \frac{\text{polinom st. ena manj kot spodaj}}{(x-a)^{n-1} (x^2 + bx + c)^{m-1}}$ \\

Substitucija: $t = \tan x, \sin^2 x = t^2 /(1 + t^2), \cos^2 x = 1/(1 + t^2), \dx = \dt/(1 + t^2)$\\
Substitucija: $u = \tan (x/2), \sin x = 2 u /(1 + u^2), \cos x = (1-u^2)/(1 + u^2), \dx = 2 du/(1 + u^2)$\\

\end{document}
