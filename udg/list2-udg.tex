\documentclass[8pt,a4paper]{amsart}
% ukazi za delo s slovenscino -- izberi kodiranje, ki ti ustreza
\usepackage[slovene]{babel}
%\usepackage[cp1250]{inputenc}
%\usepackage[T1]{fontenc}
\usepackage[utf8]{inputenc}
\usepackage{amsmath,amssymb,amsfonts}
\usepackage{url}
%\usepackage[normalem]{ulem}
\usepackage{enumerate}
\usepackage[dvipsnames,usenames]{color}


\usepackage[
top    = 1cm,
bottom = 1cm,
left   = .5cm,
right  = 0.5cm]{geometry}
%
%% ne spreminjaj podatkov, ki vplivajo na obliko strani
%\textwidth 19cm
%\textheight 27cm
%\oddsidemargin-1.5cm
%\evensidemargin-1.5cm
%\topmargin-30mm
%%\addtolength{\footskip}{10pt}
%\pagestyle{plain}
%%\overfullrule=15pt % oznaci predlogo vrstico


% ukazi za matematicna okolja
\theoremstyle{definition} % tekst napisan pokoncno
\newtheorem{definicija}{Definicija}[section]
\newtheorem{primer}[definicija]{Primer}
\newtheorem{opomba}[definicija]{Opomba}
\newtheorem{zgled}[definicija]{Zgled}

\theoremstyle{plain} % tekst napisan posevno
\newtheorem{lema}[definicija]{Lema}
\newtheorem{izrek}[definicija]{Izrek}
\newtheorem{trditev}[definicija]{Trditev}
\newtheorem{posledica}[definicija]{Posledica}




\newcommand{\R}{\mathbb R}
\newcommand{\N}{\mathbb N}
\newcommand{\Z}{\mathbb Z}
\newcommand{\C}{\mathbb C}
\newcommand{\Q}{\mathbb Q}

\begin{document}
\thispagestyle{empty}
\setlength{\parindent}{0pt}
\section*{\textbf{Druga fundamentalna forma}} %%%%% 2FF in ukrivljenosti

\textsc{Definicija: }Naj bo $\sigma (u,v)$ karta ploskve $S$ in naj bo $\textbf{N}(u,v)$ zvezno vektorsko polje, ki predstavlja normalo na $S$. (torej je $N = \sigma _u \times \sigma _v)/\| \sigma _u \times \sigma _v \|$) \textbf{Drugo fundamentalno formo} ($IIFF$) $S$ glede na karto $\sigma $ predstavljajo skalarni produkti:

\begin{itemize}

\item $L=\sigma _{uu} \cdot \textbf{N}$;

\item $M=\sigma _{uv} \cdot \textbf{N}$;

\item $N=\sigma _{vv} \cdot \textbf{N}$.

\end{itemize}



\textsc{Definicija: }Naj bo $\gamma$ pot na ploskvi $S$ in naj bo $\varphi $ kot med normalo krivulje $\textbf{n}$ in normalo ploskve $\textbf{N}$. \textbf{Normalno ukrivljenost} poti $\gamma$ na $S$ je

$$
\kappa_n = \gamma ''\textbf{N} = \kappa \cos \varphi .
$$

\textbf{Geodetska ukrivljenost }poti $\gamma$ na $S$ je

$$
\kappa_g = \gamma ''(\textbf{N} \times \gamma ')=\pm \kappa \sin \varphi .
$$

\textsc{Opomba: }$\kappa_n$ in $\kappa_g$ spremenita predznak, če izberemo drugo normalo. Prav tako $\kappa_g$ spremeni predznak ($\kappa_n$ pa se ohrani), če pot $\gamma$ parametriziramo v `drugo smer'.

\textsc{Trditev: }Za poljubno ploskev $S$ in poljubno pot $\gamma (t)=(u(t),v(t))$ na $S$ veljajo naslednje formule:
\begin{itemize}
\item $\kappa ^2 = \kappa _g^2+\kappa_n^2$,
\item $\kappa_n = Lu'^2+2Mu'v'+Nv'^2$,
\item $\kappa_n=\frac{L\dot{u}^2+2M\dot{u}\dot{v}+N\dot{v}^2}{E\dot{u}^2+2F\dot{u}\dot{v}+G\dot{v}^2}$
\end{itemize}

\textsc{Definicija: }Naj bosta

$$
F_I=\left( \begin{array}{cc} E & F \\ F & G \end{array} \right)
\qquad
F_{II}=\left( \begin{array}{cc} L & M \\ M & N \end{array} \right)
$$
matriki, ki predstavljata fundamentalni formi ploskve $S$ v bazi $\{ \sigma_u,\sigma_v\}$. Vrednosti $\kappa_1,\kappa_2$, za kateri ima determinanta $|F_{II}-\kappa F_I|$ vrednost $0$, se imenujeta \textbf{glavni ukrivljenosti} ploskve $S$, bazna vektorja iz $TS$ pripadajočih ničelnih prostorov matrike pa \textbf{glavna vektorja}. Matrika $\mathcal{W}=-F_I^{-1}F_{II}$ se imenuje \textbf{Weingartnova matrika}.

\textsc{Opomba:} V nekaterih primerih dobimo le eno glavno ukrivljenost (glavna ukrivljenost vedno obstaja), kar pomeni, da so vsi vektorji tangentnega prostora v omenjeni točki glavni.
\\

\textsc{Trditev:}
\begin{enumerate}
\item Glavna vektorja v vsaki točki na ploskvi tvorita bazo tangentnega prostora.
\item Če velja $\kappa_1 \neq \kappa_2$, sta glavna vektorja pravokotna.
\item $K = \det(\mathcal{W})$.
\item Glavni ukrivljenosti sta lastni vrednosti $\mathcal{W}$, glavna vektorja sta lastna vektorja $\mathcal{W}$.
\end{enumerate}

\textsc{Definicija:} \textbf{Gaussova ukrivljenost} ploskve je enaka

$$
K = \kappa_1\kappa_2 = \frac{LN-M^2}{EG-F^2}.
$$

\textbf{Povprečna ukrivljenost} ploskve je enaka

$$
H = \frac{\kappa_1+\kappa_2}{2} = \frac{LG-2MF+NE}{2(EG-F^2)}.
$$

Ploskev je \textbf{minimalna}, če zanjo velja $H=0$.
\\

Gaussova ukrivljenost sfere: $K = \frac{1}{R^2}$

\section*{\textbf{Geodetke}} %%%%%%%GEODETKE

\textsc{Definicija: }Pot $\gamma$ na ploskvi $S$ je \textbf{geodetka}, če velja $\ddot{\gamma} \perp S$ (to je ekvivalentno pogoju $\ddot{\gamma} \parallel \textbf{N}$).
\\

\textsc{Trditev:} Vsaka geodetka ima konstantno hitrost: $\frac{d}{dt} \| \dot{\gamma} \| = 2\ddot{\gamma}\dot{\gamma} = 0$, (ker $\dot{\gamma} \in TS$).
\\

\textsc{Opomba: }V definiciji je geodetka parametrizirana pot, trditev pa pove, da mora imeti ta parametrizacija konstantno hitrost. Izkaže se: pot $\gamma$ je geodetka natanko tedaj, ko je njena reparametrizacija z naravnim parametrom geodetka.

Za poljuben $k \neq 0$ velja tudi sledeče: $s \mapsto \gamma (s)$ je geodetka natanko tedaj, ko je $s \mapsto \gamma (ks)$ geodetka.
\\

\textsc{Trditev: }Geodetke imajo naslednje lastnosti:

\begin{enumerate}

\item linearne poti so geodetke;

\item geodetke so lokalno najkrajše poti med dvema točkama;

\item $\forall x \in S, \forall v \in T_xS$ obstaja natanko ena parametrizirana geodetka $\gamma$ na $S$, za katero velja $\gamma (0) = x, \dot{\gamma} (0)=v$;

\item naj bo $\sigma (u,v)$ karta ploskve $S$. Pot $\gamma (t) = \sigma (u(t),v(t))$ je parametrizirana geodetka na $S$ natanko tedaj, ko zadošča sistemu

$$
\frac{d}{dt}(E\dot{u} + F\dot{v}) = \frac{1}{2}(E_u \dot{u}^2+2F_u\dot{u}\dot{v} + G_u\dot{v}^2)
$$
$$
\frac{d}{dt}(F\dot{u} + G\dot{v}) = \frac{1}{2}(E_v \dot{u}^2+2F_v\dot{u}\dot{v} + G_v\dot{v}^2).
$$

\item izometrije ploskve slikajo geodetke v geodetke.

\end{enumerate}

\textsc{Lastnosti geodetk: }

\begin{itemize}

\item $\gamma$ je geodetka natanko tedaj, ko je $\kappa_g = 0$

\item Normalni presek ploskve $S$ in ravnine je vedno geodetka.

\item Na stožcu poteka skozi poljubni dve točki geodetka (ni samo ena), poljubni dve geodetki se ne sekata nujno v eni točki, obstajata geodetki, ki se ne sekata, geodetka lahko seka samo sebe.

\end{itemize}

\section*{\textbf{Risanje geodetk na vrteninah}} %%%%%% RISANJE

\textsc{Trditev: }
\begin{itemize}

\item Naj bo $\sigma (u,v) = (f(u)\cos v, f(u) \sin v, g(u))$ parametrizacija vrtenine, pri čemer je $u$ naravni parameter poti $(f,g)$, tj. $f'^2+g'^2=1$.

\item Označimo oddaljenost točke $\sigma (u,v)$ od osi vrtenja kot $\rho = \rho (u) := f(u).$

\item Naj bo $\psi (s)$ kot med geodetko $\gamma$ in poldnevnikom $\sigma (t,v(s))$ skozi $\gamma (s) = \sigma (u(s),v(s))$ in definirajmo $\Omega := \rho \sin \psi.$
\end{itemize}

Tedaj za geodetke na vrtenini veljajo naslednje lastnosti:

\begin{enumerate}

\item Poldnevnik (pot pri konstantnem $v$) je vedno geodetka.

\item Vzporednik (pot pri konstantnem $u$) je geodetka natanko tedaj, ko je $\rho'(u) = 0$.

\item Clairotov princip: $\Omega := \rho \sin \psi$ je konstanta vzdolž vsake geodetke.

\item Če je $\Omega$ konstanta vzdolž poti $\gamma$ in $\gamma$ ni vzporednik, potem je $\gamma$ geodetka.

\item Zaradi fleksibilnosti pri izbiri orientacije je $\sin \psi$ določen le do predznaka natančno. Z upoštevanjem simetričnosti je dovolj obravnavati le nenegativne vrednosti $\Omega$.

\item Vzdolž geodetke velja $(u')^2 = 1-\Omega^2 \rho^{-2}$. Torej velja $\rho \geq \Omega$.

\item Če v točki $\gamma (s)$ velja $\rho (s) > \Omega (s)$, potem je $u'(s) \neq 0$ in geodetka zato seka vzporednik skozi $\gamma (s)$.

\item Če v točki $\gamma (s)$ velja $\rho (s) = \Omega (s)$, potem je $u'(s) = 0$, zato se geodetka v tej točki dotika vzporednika. Če poleg tega velja $\rho'(s) = 0$, je zaradi enoličnosti geodetk $\gamma$ kar omenjeni vzporednik.

\end{enumerate}

\section*{\textbf{Površina ploskve}} %%%% POVRŠINE

Če je $\sigma (u,v)$ parametrizacija ploskve, njeno površino izračunamo po naslednji formuli:

$$
P(A) = \iint_{A} \| \sigma_u \times \sigma_v \| du dv = \iint_A \sqrt{EG-F^2} dudv
$$
V zgornji formuli ne rabiš Jacobijeve determinante!
\\

\textsc{Trditev: }Naj bo $\textbf{N}$ normala ploskve $\sigma (u,v)$. Potem velja:

\begin{itemize}

\item $\textbf{N}_u \cdot \sigma_u = -L$
\item $\textbf{N}_u \cdot \sigma_v = -M$
\item $\textbf{N}_v \cdot \sigma_v = -N$
\end{itemize}

\textsc{Trditev: }Če velja $IIFF = 0$, potem je ploskev vsebovana v ravnini.
\\

\textsc{Definicija:} Pot $\gamma$ na ploskvi $S$ je \textbf{pot ukrivljenosti}, če je $\dot{\gamma}$ glavni vektor za vsak $t$ (neodvisno od parametrizacije).

\section*{\textbf{Gauss-Bonnet}}  %%%%%%%%% GAUSS BONNET

\textsc{Izrek:} Naj bo $\gamma (s)$ enostavna (omejuje disk oz. nekaj disku homeomorfnega), sklenjena, orientirana enotska pot na orientabilni ploskvi $S$. Potem velja:

$$
\int_0^{l(\gamma)}\kappa_g ds = 2 \pi - \iint_{int \gamma} K dA,
$$

kjer je $\kappa_g$ geodetska ukrivljenost, $l(\gamma)$ dolžina krivulje $\gamma$, $int \gamma$ območje, ki ga $\gamma$ omejuje in $K$ Gaussova ukrivljenost.
\\

Pri uporabi izreka pazi, da izbereš pravo smer $\gamma$ in glede na to smer še normalo na ploskev tako, da bo $int \gamma$ na levi strani.
\\

\textsc{Trditev: }Če je Gaussova ukrivljenost $K \leq 0$, potem na $S$ ne obstaja enostavna sklenjena geodetka.
\\

\textsc{Izrek: }Naj bo $\gamma (s)$ enostaven sklenjen orientiran enotski poligon na orientirani ploskvi $S$ z notranjimi koti $\alpha_1, \alpha_2, \ldots ,\alpha_n \in (0,2 \pi)$. Potem velja:

$$
\int_0^{l(\gamma)} \kappa_g ds = \sum_{i=1}^n \alpha_i - (n-2)\pi - \iint_{int \gamma}K dA.
$$

\textsc{Izrek (Eulerjeva karakteristika): }Naj bo $S$ orientirana, sklenjena ploskev. Tedaj velja:

$$
\iint_S KdA = 2\pi \chi ,
$$

kjer je $\chi = T_E+V$, $T$ je število trikotnikov v simplicialnem kompleksu, $E$ število stranic in $V$ število točk (oglišč).

$\chi (\text{sfera}) = 2, \quad \chi(\text{torus}) = 0$

Gaussova ukrivljenost sfere: $K = \frac{1}{R^2}$

\section*{\textbf{Frenetove formule in podobno}} %%%%%%%% FRENET

Naj bo $\vec{t}$ tangentni vektor poti $\gamma$ in $\vec{N}$ normala na ploskev. Definiramo $\vec{B}:=\vec{t} \times \vec{N}$.
\\

Frenetove formule:

\begin{enumerate}

\item $\vec{t}' = \kappa \vec{n}$

\item $\vec{n}'= -\kappa \vec{t} + \tau \vec{b}$

\item $\vec{b}' = -\tau \vec{n}$

\end{enumerate}

Veljajo naslednje enakosti:

\begin{enumerate}

\item $\vec{t}' = \kappa_n \vec{N} - \kappa_g \vec{B}$

\item $\vec{N}' = -\kappa_n \vec{t} + \tau_g \vec{B}$

\item $\vec{B}' = \kappa_g \vec{t} - \tau_g \vec{N}$

\item $\gamma$ je pot ukrivljenosti $\Longleftrightarrow \tau_g = 0$

\end{enumerate}

$\tau_g = \tau + \varphi' $, kjer je $\varphi$ kot med normalo krivulje $\vec{n}$ in normalo ploskve $\vec{N}$ (odvod po naravnem parametru) se imenuje \textbf{geodetska torzija}.
\\

\textsc{Trditev: }Naj bosta $S_1,S_2$ ploskvi in $\gamma = S_1 \cap S_2$ pot, ki je tudi pot ukrivljenosti na $S_1$. Potem je $\gamma$ pot ukrivljenosti na $S_2 \Longleftrightarrow$ kot med ploskvama je konstanten vzdolž preseka.

\vfill \hfill Avtor: Klemen Sajovec

\end{document}

