\documentclass[a4paper, oneside, 10pt]{article}
\usepackage[slovene]{babel}
\usepackage[utf8]{inputenc}
\usepackage[T1]{fontenc}
\usepackage{url}
\usepackage{graphicx}
\usepackage[usenames]{color}
\usepackage[reqno]{amsmath}
\usepackage{amssymb, amsthm}
\usepackage{enumerate}
\usepackage{array}
\usepackage[bookmarks, colorlinks=true, %
linkcolor=black, anchorcolor=black, citecolor=black, filecolor=black, %
menucolor=black, runcolor=black, urlcolor=black%
]{hyperref}
\usepackage[
    paper=a4paper,
    top=1.5cm,
    bottom=1.5cm,
%    textheight=24cm,
    textwidth=17cm,
    ]{geometry}

\usepackage{icomma}
\usepackage{units}

\newtheorem{izrek}{Izrek}
\newtheorem{posledica}{Posledica}

\theoremstyle{definition}
\newtheorem{definicija}{Definicija}
\newtheorem{opomba}{Opomba}
\newtheorem{zgled}{Zgled}

\def\R{\mathbb{R}}
\def\N{\mathbb{N}}
\def\Z{\mathbb{Z}}
\def\C{\mathbb{C}}
\def\Q{\mathbb{Q}}

\newcommand{\mytitle}{Teorija kodiranja in kriptografija}
\title{\mytitle}
\author{Jure Slak}
\date{\today}
\hypersetup{pdftitle={\mytitle}}
\hypersetup{pdfauthor={Jure Slak}}
\hypersetup{pdfsubject={}}

\setlength{\parindent}{0pt}
\setlength{\parskip}{8pt}

\begin{document}
\pagestyle{empty}

\begin{center}
  \bf \Large Teorija kodiranja in kriptografija
\end{center}

\textbf{Znane šifre.} (vse gledamo po modulu, vse so bločne, gledamo bloke različnih dolžin) \\
Cezarjeva šifra: $E_k(b) = b + k$ \\
Substitucijska šifra: $E_\pi(b) = \pi(b)$ \\
Vigenerjeva šifra: $E_{\underline{k}}(\underline{b}) = \underline{b} + \underline{k}$ \\
Hillova šifra: $E_A(\underline{b}) = A\underline{b}$ \\
Afina šifra: $E_{(A,\underline{b})}(\underline{x}) = A\underline{x}+\underline{b}$ \\
Vernamova šifra: Vigenerjeva, samo da je ključ enake dolžine kot besedilo.

\textbf{SPN.} ($B = C = \Sigma^{lm}$)
substitucije $\sigma \in S(\Sigma^{lm})$, permutacije $\pi \in S_{lm}$.
Postopek kodiranja: primešamo ključ, premešamo zloge, premešamo bite, ponovimo po želji.\\

\textbf{Feistlova šifra.} Razdelimo besedilo na dva polovici, v vsakem krogu levo nastavimo na
prejšnjo desno, desno pa je $L_{i-1}\oplus f(R_{i-1})$.

\textbf{Tokovne šifre.} Vsak blok zašifriramo s svojim ključem.
\textbf{LFSR.} V registru je na začetku IV, shiftamo v levo, novega izračunamo s pomočjo prejšnjih
$z_n = \sum_{i=1}^mc_iz_{n-i}$. Karakteristični polinom je $c(x) = 1+\sum c_ix^i$.
LFSR ima periodo $q^m-1$, $q$ je ponavadi 2. Če je karakteristični polinom LFSR-ja nerazcepen, je perioda enaka
njegovemu redu, tj. najmanjši tak $t$, da $c\,|\,x^t-1$.

\textbf{RSA.} (Asimetričen, iz kodirnega ključa ne znamo dobiti dekodirnega ključa)
Izberemo $p$, $q$, $n=pq$. $m = \varphi(n) = (p-1)(q-1)$. Izberemo $e\cdot d \equiv 1 \pmod{m}$. Javni ključ
$(n,e)$. Kodiranje $E_{(n,e)}(x) = x^e \pmod{n}$, dekodiranje $D_{(n,d)}(y) = y^d \pmod{n}$.

\textbf{Izmenjava ključev.} Veliko praštevilo $p$ in $\alpha \in \Z_p^\ast$ z velikim redom $n$.
Alenka izbere naključen $a \in \{1,2,\dots,n-1\}$ in Bojan $b \in \{1,2,\dots,n-1\}$. Alenka pošlje
$\alpha^a$, Bojan pa $\alpha^b$. Skupni ključ $\alpha^{ab}$. Napadalec pozna samo $\alpha$,
$\alpha^a$ in $\alpha^b$. Varnost na problemu diskretnega logaritma. Man in the middle attack.

\textbf{ElGamalov kriptosistem.} $p$ praštevilo in $\alpha$ primitiven element, $A = \alpha^a, B =
\alpha^b$, $B = C = \Z_p^\ast$, $K = \Z_p^\ast \times \Z_p^\ast$, $E_{(a,B)} = B^a \cdot x
\pmod{p}$, $D_{(b,A)}(y) = A^{p-b-1} \cdot y \pmod{p}$. Javni ključ $p,\alpha, b)$. Alenka pošlje
$A$ in besedilo. $a$ je vsakič drug, preprečimo napad z znanim besedilom.

\textbf{Lastnost popolne tajnosti.} $B, K, C$ slučajne spremenljivke z zalogo vrednosti $\mathcal B,
\mathcal K, \mathcal C$. Če sta $B$ in $K$ neodvisni, in $P(B = b) > 0$ za vsak $b$, potem ima
kriptosistem $\mathcal S$ z dano verjetnostno porazdelitvijo na $\mathcal B \times \mathcal K$
lastnost popolne tajnosti, če za vsak $b$ in $c$ velja $P(B=b|C=c) = P(B=b)$.
Če ima kriptosistem to lastnost, potem velja, $|\mathcal B| \leq |\mathcal C| \leq |\mathcal K|$. Če
veljajo enakosti, potem ima sistem lastnost popolne tajnosti natanko tedaj, ko za vsak $b$, $c$
obstaja natanko en ključ, ki ju zakodira in $K$ je enakomerno razporejena.
Velja $LPT \iff  P(C=c|B=b) = P(C=c)$. $P(C=c|B=b) = \sum_{E_k(b)=c} P(K=k)$.

\textbf{Verjetnost.}
Bayesova formula: $P(A|B) = \frac{P(B|A) \cdot P(A)}{P(B)}$.
Popoln sistem dogodkov je tak, da so vsi neodvisni in velja $P(B) = \sum P(B|A_i)P(A_i)$.

\textbf{Zgoščevalne funkcije.} $h: \{0,1\}^\ast \to \{0,1\}^n$. Odpornost praslik: za vsak $z$ se ne da
najti $x$, da $h(x) = z$ (enosmerna funkcija), odpornost drugih praslik: za dani $x$ ne moremo najti $x'$, da $h(x) =
h(x')$. Odpornost na trke: najti $x$ in $x'$ da $h(x) = h(x')$. Najbolj pogost način je iterativno
klicanje kompresijske funkcije. $H_i = f(H_{i-1}||x_i)$. Če besedilo ni primerno dolgo, dodamo še
primerno število ničel, potem pa še za en blok, število za dolžino besedila dodamo na konec, po
blokih dolgih $r-1$, z enicami na začetku.

\textbf{Bločno kodiranje.}
Naivno. Verižno: $c_0 = IV$, $c_j = E(b_j\oplus c_{i-1})$. S števcem: $c_j = b_j \oplus E(cnt + j
\pmod{2^n})$. Izhodna povratna zanka: $c_j = b_j \oplus E^j(IV)[:r]$. Kodna povratna zanka:
$l_0 = IV$. Pri vsakem $j$ $O_j = E(l_j), c_j = b_j \oplus O_j[:r], l_j = O_j\cdot 2^r \oplus c_j
\pmod{2^n}$.

\textbf{Digitalni podpisi.} Podpišemo izvleček z $RSA$. Najprej podpišemo, potem šifriramo.
ElGamalov sistem podpisovanja. Naj bo $p$ praštevilo in $\alpha^a=\beta$ v $\Z+p^\ast$. Znani so
$(p, \alpha, \beta)$. Podpisnik izbere naključen $k$ iz $\Z_p^\ast$ in izračuna podpis $(\gamma,
\delta), \gamma = \alpha^k \pmod{p}, \delta = (x-a\gamma)k^{-1} \pmod{p-1}$. Preverimo, če je podpis
veljaven $\beta^\gamma\gamma^\delta = \alpha^x \pmod{p}$.

\textbf{Razširjeni evklidov algoritem.} Zadnji neničelen $r$ je ostanek, v tisti vrstici tudi
$s$, $t$. \\
\begin{tabular}[h]{c|c|c|c}
        & $r_0$ & 1 & 0 \\
  $q_1$ & $r_1$ & 0 & 1 \\ \hline
  $q_2$ & $r_2 = r_0 - q_1r_1$ & $s_2 = s_0-q_1s_1$ & $t_2 = t_0 - q_1t_1$ \\
\end{tabular}

\textbf{Kitajski izrek o ostankih.} $x = a_i \pmod{m_i}$, $m = \prod m_i$. $M_i = m/m_i$. $\exists
y_i: y_iM_i = 1 \pmod{m_i}$. $x = \sum a_iy_iM_i \pmod{m}$.


\end{document}
% vim: spell spelllang=sl
% vim: foldlevel=99
