\documentclass[a4paper,10pt]{article}
\usepackage[slovene]{babel}
\usepackage[utf8]{inputenc}
\usepackage[T1]{fontenc}
\usepackage{lmodern}
\usepackage{url}
\usepackage{graphicx}
\usepackage[usenames]{color}
\usepackage[reqno]{amsmath}
\usepackage{amssymb,amsthm}
\usepackage{enumerate}
\usepackage{array}
\usepackage[bookmarks, colorlinks=true, %
linkcolor=black, anchorcolor=black, citecolor=black, filecolor=black,%
menucolor=black, runcolor=black, urlcolor=black, pdfencoding=unicode%
]{hyperref}
\usepackage[
  paper=a4paper,
  top=1.5cm,
  bottom=1.5cm,
%    textheight=24cm,
  textwidth=18cm,
  ]{geometry}

\usepackage{icomma}
\usepackage{units}

\newtheorem{izrek}{Izrek}
\newtheorem{posledica}{Posledica}

\theoremstyle{definition}
\newtheorem{definicija}{Definicija}
\newtheorem{opomba}{Opomba}
\newtheorem{zgled}{Zgled}

\def\R{\mathbb{R}}
\def\N{\mathbb{N}}
\def\Z{\mathbb{Z}}
\def\C{\mathbb{C}}
\def\Q{\mathbb{Q}}

\newenvironment{itemize*}%
{
\vspace{-6pt}
\begin{itemize}
\setlength{\itemsep}{0pt}
\setlength{\parskip}{2pt}
}
{\end{itemize}}

\newenvironment{enumerate*}%
{
\vspace{-6pt}
\begin{enumerate}
\setlength{\itemsep}{0pt}
\setlength{\parskip}{2pt}
}
{\end{enumerate}}

\newcommand{\mytitle}{UNM list}
\title{\mytitle}
\author{Jure Slak}
\date{\today}
\hypersetup{pdftitle={\mytitle}}
\hypersetup{pdfauthor={Jure Slak}}
\hypersetup{pdfsubject={}}

\pagestyle{empty}
\setlength{\parindent}{0pt}
\setlength{\parskip}{5pt}

\newcommand{\mybox}[1]{\frame{\rule{0pt}{10pt}\ #1\ }}

\newcommand{\T}{\mathsf{T}\!}
\renewcommand{\H}{\mathsf{H}\!}

\begin{document}

% \subsection*{Zapis števil in napake}
% Števila predstavimo kot elemente $P(b, t, L, U)$, to so vsa decimalna števila
% $0.c_1c_2\ldots c_t \cdot b^e$, $L \leq e \leq U, c_1 \neq 0$. Osnovna zaokrožitvena
%  napaka je $u = \frac12b^{-t}$.
%
% Standard IEEE single: $\mybox{s}\mybox{e}\mybox{f}$, $s$ predznak, 1 bit, $e$ je eksponent,
% 8 bitov, $f$ je mantisa, 23 bitov. Število $x$ zapišemo kot $x = (-1)^s(1+f)2^{e-127}$.
% Denormalizirano število: $e = 0, f \neq 0, x = (-1)^s(0+f)2^{-126}$
%
% Za elementarne operacije velja $\operatorname{fl}(a \oplus b)$ se v praksi izračuna
% z relativno napako $|\delta| < u$ v $(a \oplus b)(1+\delta)$. Za zaporednje $n$ operacij
% je napaka manjša od $nu$.
%
% Direktna slabilnost: vedno majhna relativna napaka. \\
% Obratna stabilnost: izračunan rezultat je točen rezultat malo spremenjenih začetnih vrednosti.
%
% \subsection*{Nelinearne enačbe}
% Iščemo ničle $\alpha$ funkcije $f$. Občutljivost $\frac{1}{f'(\alpha)}$, za
% dvojno ničlo velja ocena$\sqrt{\frac{2 \epsilon}{|f''(\alpha)|}}$, kjer $|f(\alpha)| \leq \epsilon$.
%
% \textsc{Bisekcija:} razpolavljamo interval, na katerem imamo ničlo. Št korakov za
% natančnost $\varepsilon$: $k \geq \log\left(\frac{|b-a|}{\varepsilon}\right)$.
%
% \textsc{Navadna iteracija:} Iščemo fiksno točno $g(\alpha) = \alpha$. Metoda: $x_{r+1} =
% g(x_r)$. Če je $|g'(\alpha)| < 1$ je točka privlačna, če $|g'(\alpha)| > 1$ je
% odbojna. Red konvergence je $p$, če je $g^{(p)}(\alpha) \neq 0$, nižji odvodi pa so $0$ v $\alpha$. Ocene za napako:
% $|x_r - \alpha| \leq m^r|x_0 - \alpha|$, $|x_{r+1} - \alpha|  \leq \frac{m}{1-m} |x_r - x_{r+1}|$, kje je $m$ Lipscitzeva
% konstanta za $g$ ($m = \max g'$).
%
% \textsc{Tangentna metoda:} $x_{r+1} = x_r - \frac{f(x_r)}{f'(x_r)}$. Konvergenca je za
% enojne ničle kvadratična, za večkratne ničle linearna. Če za enostavno ničlo
% velja $f''(\alpha) = 0$ je konvergenca kubična, itn\dots Vse ničle so privlačne.
%
% \textsc{Sekantna metoda:} $x_{r+1} = x_r - \frac{f(x_r)(x_r - x_{r-1})}{f(x_r) -
% f(x_{r-1})}$. Red konvergence: $\frac{1+\sqrt{5}}{2}$.
%
% \textsc{Laguerrova metoda} za iskanje ničel polinomov: $z_{r+1} = z_r -
% \frac{np(z_r)}{p'(z_r) \pm \sqrt{(n-1)((n-1)p'^2(z_r) - np(z_r)p''(z_r))}}$ \\
% Pri stabilni metodi izberemo predznak tako, da je absolutna vrednost imenovalca
% največja. Če izbiramo vedno $-$ ali  + skonvergiramo k levi oz. desni ničli, če
% so vse ničle realne. Konvergenca v bližini enostavne ničle je kubična. Metoda
% najde tudi kompleksne ničle.
%
% \textsc{Redukcija polinoma:} Imamo eno ničlo, radi bi jo faktorizirali ven.
% Poznamo obratno in direktno redukcijo, pri katerih je stabilno izločati ničle v
% padajočem in naraščajočem vrstnem redu po absolutni vrednosti. V praksi
% uporabimo kombinirano metodo: do nekega $r$ uporabimo z ene strani obratno, z
% druge pa direktno. Ta $r$ izberemo tako, da je $|\alpha^ra_{n-r}|$ maksimalen.
%
% \textsc{Durand-Kernerjeva metoda:} Iščemo vse ničle naenkrat: $x_k^{(r+1)} =
% x_k^{(r)} - \frac{p(x_k^{(r)})}{\prod_{\substack{j=1 \\ j \neq
% k}}^n (x_k^{(r)} - x_j^{(r)})}$. Kvadratična konvergenca. Za kompleksne ničle je
% treba začeti s kompleksnimi približki.


%%%%%%%%%%%%%%%%                   LINEARNI SISTEMI
\subsection*{Linearni sistemi}

\textsc{Norme:} $\|A\|_1 =
\max_{j\in\{1..n\}}\left(\sum_{i=1}^n|a_{ij}|\right)$ = največji stolpec,
$\|A\|_\infty = \|A^\T\|_1$ = največja vrstica \\
$\|A\|_2 = \sigma_1 = \sqrt{\lambda_{max}(A^\H A)}$ = največja singularna vrednost,
$\|A\|_F = \sqrt{\sum_{ij}a_{ij}^2}$ = gledamo kot vektor \\
Operatorska norma: $\|A\| = \max_{x\neq 0}\frac{\|Ax\|}{\|x\|}$.
Neenakosti: $\lambda \leq \|A\|$. $\|Ax\| \leq \|A\|\|x\|$.
\begin{alignat*}{4}
\textstyle \frac{1}{\sqrt{n}}\|A\|_F &\leq \|A\|_2 &&\leq \|A\|_F \\
\textstyle \frac{1}{\sqrt{n}}\|A\|_1 &\leq \|A\|_2 &&\leq \sqrt{n} \|A\|_1 \\
\textstyle \frac{1}{\sqrt{n}}\|A\|_\infty &\leq \|A\|_2 &&\leq \sqrt{n} \|A\|_\infty \\
N_\infty(A) &\leq \|A\|_2 &&\leq nN_\infty(A) \\
            &\leq \|A\|_2 &&\leq \sqrt{\|A\|_1\|A\|_\infty} \\
\|a_i\|_2, \|\alpha_i\|_2 &\leq \|A\|_2 &&
\end{alignat*}

Rešujemo sistem $Ax=b$. Za napako $x$ velja ocena:
\[ \frac{\|\Delta x\|}{\|x\|}  \leq \frac{\kappa(A)}{1-\kappa(A) \frac{\|\Delta
A\|}{\|A\|}} \left( \frac{\|\Delta A\|}{\|A\|} + \frac{\|\Delta
b\|}{\|b\|}\right) \]
Količina $\kappa(A)$ se imenuje občutljivost matrike. $\kappa(A) =
\|A\|\|A^{-1}\|$. Velja $\kappa_2(A) = \frac{\sigma_1(A)}{\sigma_n(A)} \geq 1$.

\textsc{LU razcep} s kompletnim pivotiranjem: matriko $A$ zapišemo kot $PAQ =
UL$, $L$ sp.\ trikotna z $1$ na diagnoali in $U$ zg.\ trikotna, ter $P, Q$
permutacijski matriki stolpcev in vrstic. Algoritem:
\scriptsize
\begin{verbatim}
Q = I, P = I
for j = 1 to n:
    r, q taka, da a_rq največji v podmatriki A(j+1:n)
    zamenjaj vrstici r in j v A, L, P // za delno pivotiranje
    zamenjaj stolpca q in j v A, L, Q // za kompletno pivotiranje
    for i  = j+1 to n:
        l_ij = a_ij / a_jj
        for k = j+1 to n:
            a_ik = a_ik - l_ij * a_jk
\end{verbatim}
\normalsize

Postopek na roke:
\begin{enumerate*}
  \item * Če delamo pivotiranje zamenjamo primerne vrstice in stolpce v
    $A, P, Q$, da je $a_{00}$ največji. \label{enum:lu:pivot}
  \item Prvi stolpec delimo z $a_{00}$, razen $a_{00}$, ki ga pustimo na miru.
  \item Za vsak element v podmatriki $A(2:n, 2:n)$: $a_{ij} = a_{ij} - a_{i1} \cdot
    a_{1j}$ (odštejemo produkt  $\leftarrow$ in $\uparrow$).
  \item Ponovimo postopek na matriki $A(2:n, 2:n)$.
\end{enumerate*}

Delno pivotiranje uporablja samo matriko $P$, za LU razcep brez pivotiranja pa
preskočimo~\ref{enum:lu:pivot}.

Skalarni produkt potrebuje $2n$ operacij. Reševanje s premimi substitucijami
potrebuje $n^2$, z obratnimi $n^2+n$. Reševanje z LU razcepom (brez pivotiranja) potrebuje
$\frac23n^3 + \frac32n^2 + \frac56n$ operacij.

Za izračunani LU razcep $\hat{L}\hat{U} = A + E$ velja $|E| \leq
nu|\hat{L}||\hat{U}|$.

Pivotna rast: $g = \frac{\max u_{ij}}{\max a_{ij}}$. Pri delnem pivotiranju $g < 2^n$.

\textsc{Razcep Choleskega:}
Za spd matriko $A$ obstaja razcep $A = VV^\T$.
\scriptsize
\begin{verbatim}
for k = 1 to n:
    v_kk = sqrt(a_kk - sum(v_kj^2, j=1 to k))
    for i = k+1 to n:
        v_ik = 1/v_kk * (a_ik - sum(v_ij * v_kj, j = 1 to k))
\end{verbatim}
\normalsize
Postopek na roke po stolpcih:
\begin{enumerate*}
  \item Če sem diagonalen element: odštejem od sebe skalarni produkt vrstice na
    levo same s sabo in se korenim.
  \item Če nisem diagonalni: od sebe odštejem skalarni produkt vrstice levo od
    sebe z vrstico levo od mojega diagonalnega. Nato se delim z
    diagonalnim.
\end{enumerate*}

Razcep stane $\frac13n^3$ operacij. Je obratno stabilno. Je enoličen.

%%%%%%%%%%%%%%%%                   NELINEARNI SISTEMI
\subsection*{Nelinearni sistemi}

\textsc{Jacobijeva iteracija:}
Posplošitev navadne iteracije. Naj velja $G(\alpha)= \alpha$. Metoda: $x^{(r+1)}
= G(x^{(r)})$. Točka $\alpha$ je privlačna, če velja $\rho(DG(\alpha)) < 1$.
Dovolj je $\|DG(\alpha)\| < 1$.  Konvergenca je linearna.

\textsc{Newtonova metoda:}
Posplošitev tangentne metode. Metoda: reši sistem $DF(x^{(r)})\Delta x^{(r)} =
-F(x^{(r)})$. \\ $x^{(r+1)} = x^{(r)} + \Delta x^{(r)}$. Konvergenca je
kvadratična.

%%%%%%%%%%%%%%%%                   PROBLEM NAJMANJŠIH KVADRATOV
\subsection*{Problem najmanjših kvadratov}

\textsc{Reševanje predoločenih sistemov:} Za dan predoločen sistem $Ax=b$
rešujemo normalni sistem $A^\T Ax=A^\T b$. Če je $A$ polnega ranga, je $x$ enoličen.
Rešujemo z razcepom Choleskega. Število operacij: $n^2m + \frac13n^3$.

QR razcep je bolj stabilen. Za $A \in \R^{m\times n}$ obstaja enoličen razcep $A
= QR$, $Q^\T Q = I$ in $R$ zg. trikotna s pozitivnimi diagonalci. Za predoločen
sistem rešimo $Rx = Q^\T b$.

\textsc{CGS in MGS} Klasična GS ortogonalizacija. Od vsakega stolpca $a_k$
odštejemo pravokotne projekcije $a_i, i < k$. Algoritem ni najbolj stabilen.
\scriptsize
\begin{verbatim}
for k = 1 to n:
    q_k = a_k
    for i = 1 to k-1:
        r_ik = q_i' * a_k (CGS) ALI = q_i' * q_k (MGS)
        q_k = q_k - r_ik q_i
    r_kk = ||q_k||
    q_k = q_k / r_kk
\end{verbatim}
\normalsize
Za večjo natančnost izračunamo $[A b] = [Q q_{n+1}] [R z; 0 p]$ in rešimo $Rx =
z$. Porabi $2nm^2$ operacij.

Razširjeni QR razcep: $A = \tilde{Q}\tilde{R}$, $Q \in \R^{m \times m}$
ortogonalna, $R$ zgornje trapezna. $\tilde{Q} = [Q\ Q_1]$, $\tilde{R} = [R; 0]$.

\textsc{Givensove rotacije}

Elemente v $A$ po stolpcih enega po enega ubijamo z rotacijami. Rotacija, ki ubije element
$a_{ki}$ je $R_{ik}^\T([i k],[i,k]) = [c\ s; -s\ c]$, in ostalo identiteta.
Parametre nastavimo: $c = x_{ii}/r$, $s =
x_{ki}/r$, $r = \sqrt{x_{ii}^2+x_{ki}^2}$. $\tilde{Q}$ dobimo kot prokdukt vseh rotacij, potrebnih za genocid
elementov $A$. Rotacija spremeni samo $i$-to in $k$-to vrstico.

Število operacij: $3mn^2 -n^3$. Če potrebujemo $\tilde{Q}$, potem rabimo še dodatnih
$6m^2n - 3mn^2$ operacij.

\scriptsize
\begin{verbatim}
Q = I_m
for i = 1 to n:
    for k = i+1 to m:
      r = sqrt(a_ii^2 + a_ki^2)
      c = a_ii/r, s = a_ki/r
      A([i,k], i:n) = [c s; -s c] A([i k], i:n)
      b([i, k]) = [c s; -s c] b([i, k]) // za predoločen sistem
      Q(i, [i k]) = Q(i, [i k]) [c -s; s c] // za matriko Q
Q = Q'
\end{verbatim}
\normalsize

\textsc{Hausholderjeva zrcaljenja}
Definiramo $P = I - \frac{2}{w^\T w}ww^\T$. $P$ je zrcaljenje prek ravnine z normalo
$w$. $Px$ = $x - \frac{1}{m}(x^\T w)w$, $m = \frac12w^\T w$.

Da vektor $x$ prezrcalimo tako, da mu uničimo vse razen prve komponente,
uporabimo $w = [x_1 + \operatorname{sign}(x_1)\|x\|_2; x_2; \dots x_n]$ in $m =
\|x\|_2(\|x\|_2 + |x_1|)$. Število operacij za $Pz$ je $4n$m za $w$ in $m$ pa
potrebujemo $2n$ operacij.

\scriptsize
\begin{verbatim}
Q = I_m
for i = 1 to n:
    w_i iz R^{m-i+1}, ki prezrcali A(i:m, i) v +-k e_1
    A(i:m,i:n) = P_i * A(i:m, i:n)
    b(i:m) = P_i * b(i:m)           // za predoločen sistem
    Q(i:m, 1:n) = P_i * Q(i:m, 1:n) // za matriko Q
Q = Q'
\end{verbatim}
\normalsize

Reševanje predoločenega sistema tako stane $2mn^2 - \frac23n^3$. Za $\tilde{Q}$
potrebujemo še $4m^2n-2mn^2$ operacij. Za kvadratne sisteme je stabilnejši, a
rabimo $\frac43n^3$ operacij.

Za napako pri reševanju predoločenega sistema velja:
$\frac{\|\Delta x\|}{\|x\|} \leq \frac{\varepsilon\kappa_2(A)}{1-\varepsilon\kappa_2(A)}
\left(2+(\kappa_2(A) + 1)\frac{\|r\|}{\|A\|\|x\|} \right), r = Ax -b$.

%%%%%%%%%%%%%%%%                   LASTNE VREDNOSTI
\subsection*{Lastne vrednosti}
Desni in levi lastni vektorji: $y^\H A = \mu y^\H$ in $Ax = \lambda x$. Levi in
desni vektorji za različne l. vrednosti so pravokotni. Občutljivost lastne
vrednosti je $\frac{1}{y^\H x}$, kjer sta $x$ in $y$ normirana levi in desni
lastni vektor.

\textsc{Potenčna metoda:} Pravzaprav navadna iteracija. Izberemo si začetni
vektor $z$ in tolčemo č\'{e}z matriko $A$ in normiramo, dokler ne postane lastni vektor. Ta
metoda ima linearno konvergenco k lastnemu vektorju za dominantno lastno
vrednost, če je le ena sama lastna vrednost največja.
Hitrost konvergence je $\lambda_1 / \lambda_2$, kjer sta to dve
največji lastni vrednosti.
\scriptsize
\begin{verbatim}
z = ones(n, 1)  // naključen neničeln vektor
for k = 1 to m: // m je veliko število
    y = A * z
    z = y / ||y||
\end{verbatim}
\normalsize
Če imamo lastni vektor $v$, potem želimo imeti lastno vrednost $\lambda$.
Najboljši približek je Raylegihov kvocient: $\rho(A, v) = \frac{z^\H\!\!Az}{z^\H
z}$.
Kot kriterij v potenčni metodi uporabimo \verb'|A * z - p(A, z)| < eps'.

Če imamo dober približek $\tilde{\lambda}$ za lastno vrednost vrednost
$\lambda_i$ uporabimo inverzno iteracijo. Iščemo največjo vrednost matrike
$(A-\tilde{\lambda}I)^{-1}$, ki
ima lastne vrednosti $\frac{1}{\lambda_i - \lambda}$.
\scriptsize
\begin{verbatim}
z = ones(n, 1)
for k = 0 to m:
    reši (A - lambda I)y = z
    z = y / ||y||
\end{verbatim}
\normalsize

\textsc{Schurova forma:} Za vsako matriko $A$ obstaja Schurova forma $S$, da je
$A = USU^\H$,
kjer je $U$ unitarna in $S$ zgornje trikotna. Na diagnoali so lastne vrednosti.
V primeru kompleksnih lastnih vrednosti imamo 2x2 bloke.

\textsc{Otrogonalna iteracija:} Za izračun Schurove forme. $Z$ je lahko $n\times
p$ matrika z ortonormiranimi stolpci, tako dobimo prvih $p$ stolpcev Schurove
forme. Za $p = 1$ je to potenčna metoda, za $p = n$, pa dobimo celo schurovo
formo.
\scriptsize
\begin{verbatim}
Z = eye(n) // naključna matrika z otronormiranimi stolpci
for k = 0 to m:
    Y = A * Z
    [Q, R] = qr(Y)
    Z = Q
\end{verbatim}
\normalsize

\textsc{QR iteracija:} Najboljša metoda za izračun lastnih vrednosti $A$.
\scriptsize
\begin{verbatim}
for k = 0 to m:
    [Q, R]= qr(A)
    A = R * Q
\end{verbatim}
\normalsize

\textsc{Gerschgorinov izrek:} Naj bo $A \in \C ^{n\times n}$, $C_i =
\overline{K}(a_{ii}, r=\sum_{j=1, j \neq i}^n | a_{ij}|), i=1,2,\ldots ,n$.
Potem vsaka lastna vrednost leži v vsaj enem Gerschgorinovem krogu. Če $m$
krogov $C_i$ sestavlja povezano množico, ločeno od ostalih $n-m$ krogov, potem
ta množica vsebuje natanko $m$ lastnih vrednosti.

Diagonalno dominantna matrika ($|a_{ij}| > \sum_{j=1, j\neq i}^n|a_{ij}|$) je obrnljiva.

%%%%%%%%%%%%%%%%                    INTERPOLACIJA
% \subsection*{Interpolacija}

% \textsc{Lagrangeev interpolacijski polinom:}

% \[ \ell_{n,j}(x)=\frac{\prod_{i=0,i\neq j}^n(x-x_i)}{\prod_{i=0,i\neq
% j}^n(x_j-x_i)} \]

% Polinom: $p(x) = \sum_{i=0}^nf(x_i)\ell_{n,i}(x)$. Definiramo $\omega(x) =
% (x-x_0)\dots(x-x_n)$. Velja $\ell_{n,i}(x) = \frac{\omega(x)}{(x-x_i)\omega'(x)}$.
%
% Ocena napake: $f(x) = p(x) + \frac{f^{(n+1)}(\xi)}{(n+1)!}\omega(x)$, kjer je
% $\xi$ nekje na intervalu, ki ga določajo $x_i$ in $x$. To se prevede na:
% $|f(x) - p(x)|  = |\omega(x) f[x_0, \ldots, x_n, x]| =
% |\omega(x) \frac{f^{(n+1)}(\xi)}{(n+1)!}| \leq \|\omega\|_\infty
% \frac{\|f^{(n+1)}\|_\infty}{(n+1)!}$


% Izračun vrednosti polinoma, ki je dan z $d_i = f[x_0, \ldots, x_i]$:
% \scriptsize
% \begin{verbatim}
% v = d_n
% for i = n-1:-1:0
%     v = d_i + (x - x_i) v
% \end{verbatim}
% \normalsize

% \textsc{Deljene diference:}
%
% Če so točke paroma različne: $f[x_i] = y_i$, ostalo izračunamo po
% rekurzivni formuli: $f[x_0, \ldots, x_k] = \frac{f[x_1, \ldots, x_k] -f[x_0, \ldots, x_{k-1}] }{x_k-x_0}$. Če
% so točke $x_0$ do $x_k$ enake, je $f[x_0, \ldots, x_k] = \frac{f^{(k)}(x_0)}{k!}$.
%
% Polinom: $p(x) = f[x_0] + f[x_0, x_1] (x-x_0) + f[x_0, x_1, x_2] (x-x_0)(x-x_1) +
% f[x_0, \dots, x_n](x-x_0)\cdots (x-x_{n-1})$
%
% %%%%%%%%%%%%%%%%                    INTEGRIRANJE
% \subsection*{Integriranje}
%
% Ekvidistančne točke $a = x_0 < x_1 < \dots < x_n = b$, $x_i = x_0 + ih$.
%
% \textsc{Sest. trapezno pravilo:} $\int_{a}^{b} f(x) dx = \frac{h}{2} (f(x_0)+2f(x_1)
% +2f(x_2)+\cdots + 2f(x_{m-1})+f(x_m))-\frac{h^2}{12}(b-a)f''(\xi)$
%
% \textsc{Sest. Simpsonovo:} $\int_{a}^{b} f(x) dx = \frac{h}{3} (f(x_0)+4f(x_1)
% +2f(x_2)+\cdots + 2f(x_{2m-2})+4f(x_{2m-1})+f(x_{2m}))-\frac{h^4}{180}(b-a)f^{(4)}(\xi)$
%
% \textsc{3/8 pravilo:} $\int_a^b f(x) dx = \frac{3}{8} h(f(x_0) + 3f(x_1)
% + 3f(x_2) + f(x_3)) - \frac{3}{80} h^5f^{(4)}(\xi)$, Vedno $\xi \in (a, b)$.
% \\[-8pt]
%
% \textsc{Richardsonova ekstrapolacija:} $I = T_h(f) + R_h(f)$, kjer je $T$
% pravilo in $R$ napaka, npr. $h^2/12 f''(\xi)$.
% S pomočjo izračuna za $h$ in $h/2$ določimo $I(f) = T_h(f) + R_h(f) = T_{h/2}(f) +
% R_{h/2}(f)$ in ocenimo napako (če so odvodi baš enaki) in boljše izračunamo integral:
% $R_{h/2}(f) \approx \frac{S_{h/2}(f) - S_h(f)}{15}, \quad I(f) \approx
% \frac{16S_{h/2}(f) - S_h(f)}{15}$.

\end{document}
% vim: syntax=tex
% vim: spell spelllang=sl
% vim: foldlevel=99
