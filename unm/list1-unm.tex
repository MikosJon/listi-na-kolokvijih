\documentclass[a4paper,10pt]{article}
\usepackage[slovene]{babel}
\usepackage[utf8]{inputenc}
\usepackage[T1]{fontenc}
\usepackage{lmodern}
\usepackage{url}
\usepackage{graphicx}
\usepackage[usenames]{color}
\usepackage[reqno]{amsmath}
\usepackage{amssymb,amsthm}
\usepackage{enumerate}
\usepackage{array}
\usepackage[bookmarks, colorlinks=true, %
linkcolor=black, anchorcolor=black, citecolor=black, filecolor=black,%
menucolor=black, runcolor=black, urlcolor=black, pdfencoding=unicode%
]{hyperref}
\usepackage[
  paper=a4paper,
  top=1.5cm,
  bottom=1.5cm,
%    textheight=24cm,
  textwidth=18cm,
  ]{geometry}

\usepackage{icomma}
\usepackage{units}

\newtheorem{izrek}{Izrek}
\newtheorem{posledica}{Posledica}

\theoremstyle{definition}
\newtheorem{definicija}{Definicija}
\newtheorem{opomba}{Opomba}
\newtheorem{zgled}{Zgled}

\def\R{\mathbb{R}}
\def\N{\mathbb{N}}
\def\Z{\mathbb{Z}}
\def\C{\mathbb{C}}
\def\Q{\mathbb{Q}}

\newenvironment{itemize*}%
{
\vspace{-6pt}
\begin{itemize}
\setlength{\itemsep}{0pt}
\setlength{\parskip}{2pt}
}
{\end{itemize}}

\newenvironment{enumerate*}%
{
\vspace{-6pt}
\begin{enumerate}
\setlength{\itemsep}{0pt}
\setlength{\parskip}{2pt}
}
{\end{enumerate}}

\newcommand{\mytitle}{UNM list}
\title{\mytitle}
\author{Jure Slak}
\date{\today}
\hypersetup{pdftitle={\mytitle}}
\hypersetup{pdfauthor={Jure Slak}}
\hypersetup{pdfsubject={}}

\pagestyle{empty}
\setlength{\parindent}{0pt}

\begin{document}

\subsection*{Napake}
Jih je veliko in so nasploh zelo depresivne in vse metode so slabe.

\subsection*{Nelinearne enačbe}
Iščemo ničle $\alpha$ funkcije $f$. Občutljivost $\frac{1}{f'(\alpha)}$, za
dvojno ničlo $\sqrt{\frac{2}{f''(x)}}$.

\textsc{Bisekcija:} razpolavljamo interval, na katerem imamo ničlo. Št korakov za
natančnost $\varepsilon$: $k \geq \log\left(\frac{|b-a|}{\varepsilon}\right)$.

\textsc{Navadna iteracija:} Iščemo fiksno točno $g(\alpha) = \alpha$. Metoda: $x_{r+1} =
g(x_r)$. Če je $|g'(\alpha)| < 1$ je točka privlačna, če $|g'(\alpha)| > 1$ je
odbojna. Red konvergence je $p$, če je $\alpha$ $p$-kratna ničla $g$.

\textsc{Tangentna metoda:} $x_{r+1} = x_r - \frac{f(x_r)}{f'(x_r)}$. Konvergenca je za
enojne ničle kvadratična, za večkratne ničle linearna. Če za enostavno ničlo
velja $f''(\alpha) = 0$ je konvergenca kubična, itn\dots Vse ničle so privlačne.

\textsc{Sekantna metoda:} $x_{r+1} = x_r - \frac{f(x_r)(x_r - x_{r-1})}{f(x_r) -
f(x_{r-1})}$. Red konvergence: $\frac{1+\sqrt{5}}{2}$.

\textsc{Laguerrova metoda} za iskanje ničel polinomov: $z_{r+1} = z_r -
\frac{np(z_r)}{p'(z_r) \pm \sqrt{(n-1)((n-1)p'^2(z_r) - np(z_r)p''(z_r))}}$ \\
Pri stabilni metodi izberemo predznak tako, da je absolutna vrednost imenovalca
največja. Če izbiramo vedno $-$ ali  + skonvergiramo k levi oz. desni ničli, če
so vse ničle realne. Konvergenca v bližini enostavne ničle je kubična. Metoda
najde tudi kompleksne ničle.

\textsc{Redukcija polinoma:} Imamo eno ničlo, radi bi jo faktorizirali ven.
Poznamo obratno in direktno redukcijo, pri katerih je stabilno izločati ničle v
padajočem in naraščajočem vrstnem redu po absolutni vrednosti. V praksi
uporabimo kombinirano metodo: do nekega $r$ uporabimo z ene strani obratno, z
druge pa direktno. Ta $r$ izberemo tako, da je $|\alpha^ra_{n-r}|$ maksimalen.

\textsc{Durand-Kernerjeva metoda:} Iščemo vse ničle naenkrat: $x_k^{(r+1)} =
x_k^{(r)} - \frac{p(x_k^{(r)})}{\prod_{\substack{j=1 \\ j \neq
k}}^n (x_k^{(r)} - x_j^{(r)})}$. Kvadratična konvergenca. Za kompleksne ničle je
treba začeti s kompleksnimi približki.


%%%%%%%%%%%%%%%%                   LINEARNI SISTEMI
\subsection*{Linearni sistemi}

\textsc{Norme:} $\|A\|_1 =
\max_{j\in\{1..n\}}\left(\sum_{i=1}^n|a_{ij}|\right)$ = največji stolpec,
$\|A\|_\infty = \|A^T\|_1$ = največja vrstica \\
$\|A\|_2 = \sigma_1 = \sqrt{\lambda_{max}(A^HA)}$ = največja singularna vrednost,
$\|A\|_F = \sqrt{\sum_{ij}a_{ij}^2}$ = gledamo kot vektor \\
Neenakosti: $\lambda \leq \|A\|$. $\|Ax\| \leq \|A\|\|x\|$.

Rešujemo sistem $Ax=b$. Za napako $x$ velja ocena:
\[ \frac{\|\Delta x\|}{\|x\|}  \leq \frac{\kappa(A)}{1-\kappa(A) \frac{\|\Delta
A\|}{\|A\|}} \left( \frac{\|\Delta A\|}{\|A\|} + \frac{\|\Delta
b\|}{\|b\|}\right) \]
Količina $\kappa(A)$ se imenuje občutljivost matrike. $\kappa(A) =
\|A\|\|A^{-1}\| = \frac{\sigma_1(A)}{\sigma_n(A)} \geq 1$

\textsc{LU razcep} s kompletnim pivotiranjem: matriko $A$ zapišemo kot $PAQ =
UL$, $L$ sp.\ trikotna z $1$ na diagnoali in $U$ zg.\ trikotna, ter $P, Q$
permutacijski matriki stolpcev in vrstic. Algoritem:
\begin{verbatim}
Q = I, P = I
for j = 1 to n:
    r, q taka, da a_rq največji v podmatriki A(j+1:n)
    zamenjaj vrstici r in j v A, L, P
    zamenjaj stolpca q in j v A, L, Q
    for i  = j+1 to n:
        l_ij = a_ij / a_jj
        for k = j+1 to n:
            a_ik = a_ik - l_ij * a_jk
\end{verbatim}

Postopek na roke:
\begin{enumerate*}
  \item * Če delamo pivotiranje zamenjamo primerne vrstice in stolpce v
    $A, P, Q$, da je $a_{00}$ največji. \label{enum:lu:pivot}
  \item Prvi stolpec delimo z $a_{00}$, razen $a_{00}$, ki ga pustimo na miru.
  \item Za vsak element v podmatriki $A(2, n)$: $a_{ij} = a_{ij} - a_{i1} \cdot
    a_{1j}$ (odštejemo produkt  $\leftarrow$ in $\uparrow$).
  \item Ponovimo postopek na matriki $A(2, n)$.
\end{enumerate*}

Delno pivotiranje uporablja samo matriko $P$, za LU razcep brez pivotiranja pa
preskočimo~\ref{enum:lu:pivot}.

Skalarni produkt potrebuje $2n$ operacij. Reševanje s premimi substitucijami
potrebuje $n^2$, z obratnimi $n^2+n$. Reševanje z LU razcepov potrebuje
$\frac23n^3 + \frac32n^2 + \frac56n$ operacij.

Za izračunani LU razcep $\hat{L}\hat{U} = A + E$ velja $|E| \leq
nu|\hat{L}||\hat{U}|$.

\textsc{Razcep Choleskega:}
Za spd matriko $A$ obstaja razcep $A = VV^T$.
\begin{verbatim}
for k = 1 to n:
    v_kk = sqrt(a_kk - sum(v_kj^2, j=1 to k))
    for i = k+1 to n:
        v_ik = 1/v_kk * (a_ik - sum(v_ij * v_kj, j = 1 to k))
\end{verbatim}
Postopek na roke po stolpcih:
\begin{enumerate*}
  \item Če sem diagonalen element: odštejem od sebe skalarni produkt vrstice na
    levo same s sabo in se korenim.
  \item Če nisem diagonalni: od sebe odštejem skalarni produkt vrstice levo od
    sebe z vrstico levo od mojega diagonalnega. Nato se delim z
    diagonalnim.
\end{enumerate*}

Razcep stane $\frac13n^3$ operacij. Je obratno stabilno.

%%%%%%%%%%%%%%%%                   NELINEARNI SISTEMI
\subsection*{Nelinearni sistemi}

\textsc{Jacobijeva iteracija:}
Posplošitev navadne iteracije. Naj velja $G(\alpha)= \alpha$. Metoda: $x^{(r+1)}
= G(x^{(r)})$. Točka $\alpha$ je privlačna, če velja $\rho(DG(\alpha)) < 1$.
Dovolj je $\|DG(\alpha)\| < 1$.  Konvergenca je linearna.

\textsc{Newtonova metoda:}
Posplošitev tangentne metode. Metoda: reši sistem $DF(x^{(r)})\Delta x^{(r)} =
-F(x^{(r)})$. \\ $x^{(r+1)} = x^{(r)} + \Delta x^{(r)}$. Konvergenca je
kvadratična.

%%%%%%%%%%%%%%%%                   PROBLEM NAJMANJŠIH KVADRATOV
\subsection*{Problem najmanjših kvadratov}

\textsc{Reševanje predoločenih sistemov:} Za dan predoločen sistem $Ax=b$
rešujemo normalni sistem $A^TAx=A^Tb$.
Število operacij: $n^2m + \frac13n^3$.


%%%%%%%%%%%%%%%%                   LASTNE VREDNOSTI
\subsection*{Lastne vrednosti}

\textsc{Gerschgorinov izrek:} Naj bo $A \in \C ^{n\times n}$, $C_i =
\overline{K}(a_{ii}, r=\sum_{j=1, j \neq i}^n | a_{ij}|), i=1,2,\ldots ,n$.
Potem vsaka lastna vrednost leži v vsaj enem Gerschgorinovem krogu. Če $m$
krogov $C_i$ sestavlja povezano množico, ločeno od ostalih $n-m$ krogov, potem
ta množica vsebuje natanko $m$ lastnih vrednosti.

Diagonalno dominantna matrika ($|a_{ij}| > \sum_{j=1, j\neq i}^n|a_{ij}|$) je obrnljiva.

%%%%%%%%%%%%%%%%                    INTERPOLACIJA
\subsection*{Interpolacija}

\textsc{Lagrangeev interpolacijski polinom:}

$$l_{n,j}(x)=\frac{\prod_{i=0,i\neq j}^n(x-x_i)}{\prod_{i=0,i\neq j}^n(x_j-x_i)}$$

Polinom: $p(x) = \sum_{i=0}^nf(x_i)l_{n,i}(x)$

\textsc{Deljene diference:}

\begin{itemize}

\item Če so točke paroma različne: $D_{i,0} = y_i$, ostalo izračunamo po
rekurzivni formuli: $D_{i,j} = \frac{D_{i,j-1}-D_{i-1,j-1}}{x_i-x_{i-j}}$. Če
sta dve točki na $j$-tem koraku enaki, je $D_{i,j} = \frac{f^{(j)}(x_i)}{j!}$.

Polinom: $p(x) = D_{1,1} + D_{2,2}(x-x_0) + D_{3,3}(x-x_0)(x-x_1) + \cdots D_{n,n}(x-x_0)\cdots (x-x_n)$

\end{itemize}

%%%%%%%%%%%%%%%%                    INTEGRIRANJE
\subsection*{Integriranje}

Ekvidistančne točke $a = x_0 < x_1 < \dots < x_n = b$, $x_i = x_0 + ih$.

\textsc{Sest. trapezno pravilo:} $\int_{a}^{b} f(x) dx = \frac{h}{2} (f(x_0)+2f(x_1)
+2f(x_2)+\cdots + 2f(x_{m-1})+f(x_m))-\frac{h^2}{12}(b-a)f''(\xi)$

\textsc{Sest. Simpsonovo:} $\int_{a}^{b} f(x) dx = \frac{h}{3} (f(x_0)+4f(x_1)
+2f(x_2)+\cdots + 2f(x_{2m-2})+4f(x_{2m-1})+f(x_{2m}))-\frac{h^4}{180}(b-a)f^{(4)}(\xi)$

\textsc{3/8 pravilo:} $\int_a^b f(x) dx = \frac{3}{8} h(f(x_0) + 3f(x_1)
+ 3f(x_2) + f(x_3)) - \frac{3}{80} h^5f^{(4)}(\xi )$, $\quad \xi \in (x_0,x_3)$


\end{document}
% vim: syntax=tex
% vim: spell spelllang=sl
% vim: foldlevel=99
