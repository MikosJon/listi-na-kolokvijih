\documentclass[a4paper,10pt]{article}
\usepackage[slovene]{babel}
\usepackage[utf8]{inputenc}
\usepackage[T1]{fontenc}
\usepackage{lmodern}
\usepackage{url}
\usepackage{graphicx}
\usepackage[usenames]{color}
\usepackage[reqno]{amsmath}
\usepackage{amssymb,amsthm}
\usepackage{enumerate}
\usepackage{array}
\usepackage{algpseudocode}
\usepackage[bookmarks, colorlinks=true, %
linkcolor=black, anchorcolor=black, citecolor=black, filecolor=black,%
menucolor=black, runcolor=black, urlcolor=black, pdfencoding=unicode%
]{hyperref}
\usepackage[
  paper=a4paper,
  top=1.5cm,
  bottom=1.5cm,
%    textheight=24cm,
  textwidth=18cm,
  ]{geometry}

\usepackage{icomma}
\usepackage{units}

\newtheorem{izrek}{Izrek}
\newtheorem{posledica}{Posledica}

\theoremstyle{definition}
\newtheorem{definicija}{Definicija}
\newtheorem{opomba}{Opomba}
\newtheorem{zgled}{Zgled}

\def\R{\mathbb{R}}
\def\N{\mathbb{N}}
\def\Z{\mathbb{Z}}
\def\C{\mathbb{C}}
\def\Q{\mathbb{Q}}

\newenvironment{itemize*}%
{
\vspace{-6pt}
\begin{itemize}
\setlength{\itemsep}{0pt}
\setlength{\parskip}{2pt}
}
{\end{itemize}}

\newenvironment{enumerate*}%
{
\vspace{-6pt}
\begin{enumerate}
\setlength{\itemsep}{0pt}
\setlength{\parskip}{2pt}
}
{\end{enumerate}}

\newcommand{\mytitle}{UNM list}
\title{\mytitle}
\author{Jure Slak}
\date{\today}
\hypersetup{pdftitle={\mytitle}}
\hypersetup{pdfauthor={Jure Slak}}
\hypersetup{pdfsubject={}}

\pagestyle{empty}
\setlength{\parindent}{0pt}

\begin{document}

\subsection*{Napake}
Jih je veliko in so nasploh zelo depresivne in vse metode so slabe.

\subsection*{Nelinearne enačbe}
Iščemo ničle $\alpha$ funkcije $f$. Občutljivost $\frac{1}{f'(\alpha)}$, za
dvojno ničlo $\sqrt{\frac{2}{f''(x)}}$.

\textsc{Bisekcija:} razpolavljamo interval, na katerem imamo ničlo. Št korakov za
natančnost $\varepsilon$: $k \geq \log\left(\frac{|b-a|}{\varepsilon}\right)$.

\textsc{Navadna iteracija:} Iščemo fiksno točno $g(\alpha) = \alpha$. Metoda: $x_{r+1} =
g(x_r)$. Če je $|g'(\alpha)| < 1$ je točka privlačna, če $|g'(\alpha)| > 1$ je
odbojna. Red konvergence je $p$, če je $\alpha$ $p$-kratna ničla $g$.

\textsc{Tangentna metoda:} $x_{r+1} = x_r - \frac{f(x_r)}{f'(x_r)}$. Konvergenca je za
enojne ničle kvadratična, za večkratne ničle linearna. Če za enostavno ničlo
velja $f''(\alpha) = 0$ je konvergenca kubična, itn\dots Vse ničle so privlačne.

\textsc{Sekantna metoda:} $x_{r+1} = x_r - \frac{f(x_r)(x_r - x_{r-1})}{f(x_r) -
f(x_{r-1})}$. Red konvergence: $\frac{1+\sqrt{5}}{2}$.

\textsc{Laguerrova metoda} za iskanje ničel polinomov: $z_{r+1} = z_r -
\frac{np(z_r)}{p'(z_r) \pm \sqrt{(n-1)((n-1)p'^2(z_r) - np(z_r)p''(z_r))}}$ \\
Pri stabilni metodi izberemo predznak tako, da je absolutna vrednost imenovalca
največja. Če izbiramo vedno $-$ ali  + skonvergiramo k levi oz. desni ničli, če
so vse ničle realne. Konvergenca v bližini enostavne ničle je kubična. Metoda
najde tudi kompleksne ničle.

\textsc{Redukcija polinoma:} Imamo eno ničlo, radi bi jo faktorizirali ven.
Poznamo obratno in direktno redukcijo, pri katerih je stabilno izločati ničle v
padajočem in naraščajočem vrstnem redu po absolutni vrednosti. V praksi
uporabimo kombinirano metodo: do nekega $r$ uporabimo z ene strani obratno, z
druge pa direktno. Ta $r$ izberemo tako, da je $|\alpha^ra_{n-r}|$ maksimalen.

\textsc{Durand-Kernerjeva metoda:} Iščemo vse ničle naenkrat: $x_k^{(r+1)} =
x_k^{(r)} - \frac{p(x_k^{(r)})}{\prod_{\substack{j=1 \\ j \neq k}}^n (x_k^{(r)} -
x_j^{(r)})}$. Kvadratična konvergenca. Za kompleksne ničle je treba začeti s kompleksnimi približki.

\subsection*{Linearni sistemi}
Rešujemo sistem $Ax=b$. Za napako $x$ velja ocena:
\[ \frac{\|\Delta x\|}{\|x\|}  \leq \frac{\kappa(A)}{1-\kappa(A) \frac{\|\Delta
A\|}{\|A\|}} \left( \frac{\|\Delta A\|}{\|A\|} + \frac{\|\Delta
b\|}{\|b\|}\right) \]
Količina $\kappa(A)$ se imenuje občutljivost matrike. $\kappa(A) =
\|A\|\|A^{-1}\| = \frac{\sigma_1(A)}{\sigma_n(A)} \geq 1$

\textsc{LU razcep} s kompletnim pivotiranjem: matriko $A$ zapišemo kot $PAQ =
UL$, $L$ sp.\ trikotna z $1$ na diagnoali in $U$ zg.\ trikotna, ter $P, Q$
permutacijski matriki stolpcev in vrstic. Algoritem:
\begin{verbatim}
Q = I, P = I
for j = 1 to n:
    r, q taka, da a_rq največji v podmatriki A(j+1:n)
    zamenjaj vrstici r in j v A, L, P
    zamenjaj stolpca q in j v A, L, Q
    for i  = j+1 to n:
        l_ij = a_ij / a_jj
        for k = j+1 to n:
            a_ik = a_ik - l_ij * a_jk
\end{verbatim}

Postopek na roke:
\begin{enumerate*}
  \item * Če delamo pivotiranje zamenjamo primerne vrstice in stolpce v
    $A, P, Q$, da je $a_{00}$ največji. \label{enum:lu:pivot}
  \item Prvi stolpec delimo z $a_{00}$, razen $a_{00}$, ki ga pustimo na miru.
  \item Za vsak element v podmatriki $A(2, n)$: $a_{ij} = a_{ij} - a_{i1} \cdot
    a_{1j}$ (odštejemo produkt  $\leftarrow$ in $\uparrow$).
  \item Ponovimo postopek na matriki $A(2, n)$.
\end{enumerate*}

Delno pivotiranje uporablja samo matriko $P$, za LU razcep brez pivotiranja pa
preskočimo~\ref{enum:lu:pivot}.

\subsection*{Nelinearni sistemi}
\subsection*{Problem najmanjših kvadratov}
\subsection*{Lastne vrednosti}
\subsection*{Interpolacija}
\subsection*{Integriranje}


\end{document}
% vim: syntax=tex
% vim: spell spelllang=sl
% vim: foldlevel=99
