\documentclass[a4paper,oneside,10pt]{article}

\usepackage[slovene]{babel}    % slovenian language and hyphenation
\usepackage[utf8]{inputenc}    % make čšž work on input
\usepackage[T1]{fontenc}       % make čšž work on output
\usepackage[reqno]{amsmath}    % basic ams math environments and symbols
\usepackage{amssymb,amsthm}    % ams symbols and theorems
\usepackage{mathtools}         % extends ams with arrows and stuff
\usepackage{url}               % \url and \href for links
\usepackage{icomma}            % make comma a thousands separator with correct spacing
\usepackage{units}             % \unit[1]{m} and unitfrac
\usepackage{enumerate}         % enumerate style
\usepackage{array}             % mutirow
\usepackage[usenames]{color}   % colors with names
\usepackage{graphicx}          % images
\usepackage[all]{xy}

\usepackage[bookmarks, colorlinks=true, linkcolor=black, anchorcolor=black,
  citecolor=black, filecolor=black, menucolor=black, runcolor=black,
  urlcolor=black, pdfencoding=unicode]{hyperref}  % clickable references, pdf toc
\usepackage[
  paper=a4paper,
  top=2cm,
  bottom=2cm,
  textwidth=15cm,
% textheight=24cm,
]{geometry}  % page geomerty

\newtheorem{izrek}{Izrek}
\newtheorem{posledica}{Posledica}

\theoremstyle{definition}
\newtheorem{definicija}{Definicija}
\newtheorem{opomba}{Opomba}
\newtheorem{zgled}{Zgled}

% basic sets
\newcommand{\R}{\ensuremath{\mathbb{R}}}
\newcommand{\N}{\ensuremath{\mathbb{N}}}
\newcommand{\Z}{\ensuremath{\mathbb{Z}}}
\renewcommand{\C}{\ensuremath{\mathbb{C}}}
\newcommand{\Q}{\ensuremath{\mathbb{Q}}}
\newcommand{\F}{\ensuremath{\mathcal{F}}}
\newcommand{\PP}{\ensuremath{\mathcal{P}}}
\newcommand{\TT}{\ensuremath{\mathbb{T}}}
\newcommand{\V}{\ensuremath{\mathbb{V}}}
\newcommand{\Mon}{\ensuremath{\textrm{Mon}}}

% linearna orgrinjača
\newcommand{\LL}{\ensuremath{\mathcal{L}}}

% vectors
\newcommand{\vv}{\vec{v}}
\newcommand{\vu}{\vec{u}}
\newcommand{\vr}{\vec{r}}
\newcommand{\vn}{\vec{n}}
\newcommand{\va}{\vec{a}}
%\newcommand{\vb}{\vec{b}}
% \newcommand{\vc}{\vec{c}}
\newcommand{\vt}{\vec{t}}
\newcommand{\vf}{\vec{f}}
\newcommand{\vF}{\vec{F}}
\newcommand{\vE}{\vec{E}}
\newcommand{\ve}{\vec{e}}
\newcommand{\vx}{\vec{x}}
\newcommand{\vi}{\vec{\imath}}
\newcommand{\vj}{\vec{\jmath}}
\newcommand{\vS}{\vec{S}}
\newcommand{\vw}{\vec{w}}
\newcommand{\vom}{\vec{\omega}}
\newcommand{\vzeta}{\vec{\zeta}}
\newcommand{\er}{\vec{e}_r}
\newcommand{\ef}{\vec{e}_{\varphi}}
\newcommand{\et}{\vec{e}_{\vartheta}}

% greek letters
\let\oldphi\phi
\let\oldtheta\theta
\newcommand{\eps}{\varepsilon}
\renewcommand{\phi}{\varphi}
\renewcommand{\theta}{\vartheta}

% vector operators
\newcommand{\grad}{\operatorname{grad}}
\newcommand{\rot}{\operatorname{rot}}
\renewcommand{\div}{\operatorname{div}}
\newcommand{\lap}{\Delta}

% transpose
\newcommand{\T}{\ensuremath{\mathsf{T}}}
\renewcommand{\sl}{\ensuremath{\operatorname{sl}}}

% partial derivatives
\newcommand{\dpar}[2]{\ensuremath{\frac{\partial #1}{\partial #2}}}
\newcommand{\dpr}[1]{\dpar{#1}{r}}
\newcommand{\dpt}[1]{\dpar{#1}{t}}
\newcommand{\dpx}[1]{\dpar{#1}{x}}
\newcommand{\dpy}[1]{\dpar{#1}{y}}
\newcommand{\dpz}[1]{\dpar{#1}{z}}
\newcommand{\dpth}[1]{\dpar{#1}{\theta}}
\newcommand{\dpfi}[1]{\dpar{#1}{\varphi}}

% total derivatives
\newcommand{\dd}[2]{\ensuremath{\frac{d #1}{d #2}}}
\newcommand{\ddr}[1]{\dd{#1}{r}}
\newcommand{\ddt}[1]{\dd{#1}{t}}
\newcommand{\ddx}[1]{\dd{#1}{x}}
\newcommand{\ddy}[1]{\dd{#1}{y}}
\newcommand{\ddz}[1]{\dd{#1}{z}}
\newcommand{\ddth}[1]{\dd{#1}{\theta}}

% material derivatives
\newcommand{\D}[1]{\ensuremath{\frac{D#1}{Dt}}}
\newcommand{\Dn}[2]{\ensuremath{\frac{D^{#1}#2}{Dt^{#1}}}}
\newcommand{\cor}[1]{\ensuremath{#1^\circ}}
\newcommand{\con}[1]{\ensuremath{#1^\diamond}}

% tensors
\renewcommand{\t}[1]{\ensuremath{\underline{\underline{#1}}}}
\newcommand{\ta}{\t{a}}
\newcommand{\tl}{\t{\ell}}
\newcommand{\tF}{\t{F}}
\newcommand{\td}{\t{d}}
\newcommand{\tw}{\t{w}}
\newcommand{\tI}{\t{I}}
\renewcommand{\tt}{\t{t}}

% lists with less vertical space
\newenvironment{itemize*}{\vspace{-6pt}\begin{itemize}\setlength{\itemsep}{0pt}\setlength{\parskip}{2pt}}{\end{itemize}}
\newenvironment{enumerate*}{\vspace{-6pt}\begin{enumerate}\setlength{\itemsep}{0pt}\setlength{\parskip}{2pt}}{\end{enumerate}}
\newenvironment{description*}{\vspace{-6pt}\begin{description}\setlength{\itemsep}{0pt}\setlength{\parskip}{2pt}}{\end{description}}

\newcommand{\Title}{List ANA2 ISRM}
\newcommand{\Author}{Rok Kos}
\title{\Title}
\author{\Author}
\date{\today}
\hypersetup{pdftitle={\Title}, pdfauthor={\Author}, pdfcreator={\Author}, pdfproducer={\Author}, pdfsubject={}, pdfkeywords={}}  % setup pdf metadata

% \pagestyle{empty}              % vse strani prazne
\setlength{\parindent}{0pt}    % zamik vsakega odstavka
\setlength{\parskip}{10pt}     % prazen prostor po odstavku
% \setlength{\overfullrule}{30pt}  % oznaci predlogo vrstico z veliko črnine

\usepackage{titlesec}
\titlespacing*{\section}{0px}{0px}{-2px}
\titleformat*{\section}{\Large\bf}
\titlespacing*{\subsection}{0px}{0px}{-2px}
\titleformat*{\subsection}{\large\bf}

\newcommand{\lstar}{\overset{*}{\gets}}
\newcommand{\rstar}{\overset{*}{\to}}
\newcommand{\istar}{\overset{*}{\leftrightarrow}}
\newcommand{\red}{\downarrow}
\newcommand{\st}{\operatorname{st}}
\newcommand{\rez}{\operatorname{Rez}}
\newcommand{\vk}{\operatorname{vk}}
\newcommand{\mst}{\operatorname{mst}}
\newcommand{\vc}{\operatorname{\text{vč}}}
\newcommand{\vm}{\operatorname{vm}}

\usepackage{algpseudocode}
\usepackage{algorithm}

\floatname{algorithm}{Algoritem}
\algnewcommand\algorithmicto{\textbf{to}}
\algrenewtext{For}[3]{$\algorithmicfor\ #1 \gets #2\ \algorithmicto\ #3\ \algorithmicdo$}


\let\oldtextbf\textbf
\renewcommand{\textbf}[1]{\oldtextbf{\boldmath #1}}


\usepackage{multicol}  % Za razdeliti stran na pol
\setlength{\columnseprule}{1pt}
\def\columnseprulecolor{\color{black}}

% Hiperbolcne funkcije (obstajajo tudi simboli ampak imajo durgacen zapis)
\DeclareMathOperator\ch{ch}
\DeclareMathOperator\sh{sh}
\DeclareMathOperator\Th{th}  % th komanda ze zavzeta z crko thorn (staro angleska crka)
\DeclareMathOperator\cth{cth}
\DeclareMathOperator\arsh{arsh}
\DeclareMathOperator\arth{arth}

% Absolutna vrednost
\newcommand\abs[1]{\left|#1\right|}


\begin{document}

\begin{multicols}{2}
	\section*{Odvodi}
	\begin{tabular}[h]{|c|c|}
		\hline
		$f(x)$ & $f'(x)$ \\
		\hline
		$x^n$ & $nx^{n-1}$\\
		$a^x$ & $a^x\ln(x)$\\
		$e^x$ & $e^x$\\
		$\ln{x}$ & $\frac{1}{x}$\\
		$\sin{x}$ & $\cos{x}$\\
		$\cos{x}$ & $-\sin{x}$\\
		$\tan{x}$ & $\frac{1}{\cos^2{x}}$\\
		$\cot{x}$ & $-\frac{1}{\sin^2{x}}$\\
		$\arcsin{x}$ & $\frac{1}{\sqrt{1-x^2}}$\\
		$\arccos{x}$ & $-\frac{1}{\sqrt{1-x^2}}$\\
		$\arctan{x}$ & $\frac{1}{1+x^2}$\\
		$\sh{x}$ & $\ch{x}$\\
		$\ch{x}$ & $-\sh{x}$\\
		$\Th{x}$ & $\frac{1}{\ch^2{x}}$\\
		$\cth{x}$ & $-\frac{1}{\sh^2{x}}$\\
		$\arsh{x}$ & $\frac{1}{\sqrt{x^2+1}}$\\
		$\arsh{x}$ & $\frac{1}{\sqrt{x^2-1}}$\\
		\hline 

	\end{tabular}
	% Mozni dodatki pravila, logaritemsko odvajanje tangenta/odvajanje

	\section*{Integrali}
	\begin{tabular}[h]{|c|c|}
		\hline
		$f(x)$ & $\int f(x)dx$ \\
		\hline
		$x^n$ & $\frac{x^{n+1}}{n+1} (n \neq 1)$ \\
		$\frac{1}{x}$ & $\ln{\abs{x}}$ \\
		$e^x$ & $e^x$\\
		$\sin{x}$ & $-\cos{x}$\\
		$\cos{x}$ & $\sin{x}$\\
		$\frac{1}{\cos^2{x}}$ & $\tan{x}$\\
		$\frac{1}{\sin^2{x}}$ & $-\cot{x}$\\
		$\frac{1}{\sqrt{1-x^2}}$ & $\arcsin{x}$\\
		$\frac{1}{1+x^2}$ & $\arctan{x}$\\

		\hline 

	\end{tabular}

	\subsection*{Per Partes}
	\begin{align*}
		\int f(x)g'(x)dx &= f(x)g(x) - \int g(x)f'(x)dx\\
		\int udv &= uv - \int vdu
	\end{align*}
	\subsection*{Racionalne funkcije}
	$\int\frac{p(x)}{q(x)}dx$, \qquad $p(x)$, $q(x)$ sta polinoma
	\begin{enumerate}
		\item Če je $st(p(x)) <= st(p(x))$ polinoma delimo
		\item $q(x)$ razdelimo na linearne in kvadratne faktorje
		\item Izraz pod integralom razcepimo na parcialne ulomke
		\item Integriramo vsakega zase
	\end{enumerate}
	
	\begin{align*}
		k \geq 2 & \qquad st(p(x)) \leq 2k - 1\\
		st(q(x)) \leq 2k - 3 & \qquad (ax^2 + bx + c) \qquad \text{nerazcepen v \R}\\
		I = \int \frac{p(x)}{(ax^2 + bx + c)^k} &= \int \frac{Ax+B}{ax^2 + bx + c} + \frac{q(x)}{(ax^2 + bx + c)^{k-1}}
	\end{align*}
	A,B, $q(x)$ poiščemo tako da enačbo odvajamo.
	\subsection*{Korenske funkcije}

	\begin{enumerate}
		\item $\int f(\sqrt{ax + b})dx$ \qquad $t = \sqrt{ax + b}$\\
		\item $\int f(\sqrt{ax^2 + bx + c})dx$\\
		\begin{enumerate}[a]
			\item $\int \frac{dx}{\sqrt{ax^2 + bx + c}}$ ga prevedemo na oblike:
			\begin{itemize}
				\item Če je $a < 0: \int \frac{dx}{\sqrt{1-x^2}} = \arcsin{x}$
				\item Če je $a > 0: \int \frac{dx}{\sqrt{x^2 + c}} = \ln{\abs{x + \sqrt{x^2 + c}}}$
			\end{itemize}
			\item $\int \frac{p(x)}{\sqrt{ax^2 + bx + c}} = q(x)\sqrt{ax^2 + bx + c} + A\int\frac{dx}{\sqrt{ax^2 + bx + c}}$\\
			$st(p(x)) - 1 = st(q(x))$ \quad A, q(x) poiščemo z odvanjanjem
		
		\end{enumerate}
	\end{enumerate}


	\section*{Kotne funkcije}
	\begin{align*}
		\cos^2{x} &= \frac{\cos{2x} + 1}{2} & \sin^2{x} &= \frac{1 - \cos{2x}}{2}
	\end{align*}	


\end{multicols}
\end{document}
% vim: syntax=tex:
% vim set: spell spelllang=sl:
% vim: foldlevel=99

% Latex template: Jure Slak, jure.slak@gmail.com
% Dodatki k templatu: Rok Kos, kosrok97@gmail.com

