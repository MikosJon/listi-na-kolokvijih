\documentclass[8pt,a4paper]{amsart}
% ukazi za delo s slovenscino -- izberi kodiranje, ki ti ustreza
\usepackage[slovene]{babel}
%\usepackage[cp1250]{inputenc}
%\usepackage[T1]{fontenc}
\usepackage[utf8]{inputenc}
\usepackage{amsmath,amssymb,amsfonts}
\usepackage{url}
%\usepackage[normalem]{ulem}
\usepackage{enumerate}
\usepackage[dvipsnames,usenames]{color}
\usepackage{dsfont}

\usepackage[
top    = 1cm,
bottom = 1cm,
left   = .5cm,
right  = 0.5cm]{geometry}
%
%% ne spreminjaj podatkov, ki vplivajo na obliko strani
%\textwidth 19cm
%\textheight 27cm
%\oddsidemargin-1.5cm
%\evensidemargin-1.5cm
%\topmargin-30mm
%%\addtolength{\footskip}{10pt}
%\pagestyle{plain}
%%\overfullrule=15pt % oznaci predlogo vrstico


% ukazi za matematicna okolja
\theoremstyle{definition} % tekst napisan pokoncno
\newtheorem{definicija}{Definicija}[section]
\newtheorem{primer}[definicija]{Primer}
\newtheorem{opomba}[definicija]{Opomba}
\newtheorem{zgled}[definicija]{Zgled}

\theoremstyle{plain} % tekst napisan posevno
\newtheorem{lema}[definicija]{Lema}
\newtheorem{izrek}[definicija]{Izrek}
\newtheorem{trditev}[definicija]{Trditev}
\newtheorem{posledica}[definicija]{Posledica}


\newenvironment{itemize*}%
{
\vspace{-6pt}
\begin{itemize}
\setlength{\itemsep}{0pt}
\setlength{\parskip}{1pt}
}
{\end{itemize}}

\newenvironment{enumerate*}%
{
\vspace{-6pt}
\begin{enumerate}
\setlength{\itemsep}{0pt}
\setlength{\parskip}{1pt}
}
{\end{enumerate}}


\newcommand{\dx}{\ensuremath{\,\mathrm{d}x}}
\newcommand{\dt}{\ensuremath{\,\mathrm{d}t}}
\let\oldint\int
\renewcommand{\int}{\oldint \!}


\newcommand{\R}{\mathbb R}
\newcommand{\N}{\mathbb N}
\newcommand{\Z}{\mathbb Z}
\newcommand{\C}{\mathbb C}
\newcommand{\Q}{\mathbb Q}

\pagestyle{empty}

\begin{document}
\thispagestyle{empty}
\setlength{\parindent}{0pt}

\section*{Pogojna verjetnost}

\textsc{Izrek o polni verjetnosti:} Če $H_1, H_2, H_3,\ldots$ tvorijo \textbf{popoln sistem dogodkov} (tj. vedno se zgodi natanko eden izmed njih), velja:
$$
P(A) = P(H_1)P(A|H_1)+P(H_2)P(A|H_2)+P(H_3)P(A|H_3)+\cdots
$$

\textsc{Bayesova formula:}  Če $H_1, H_2, H_3,\ldots$ tvorijo popoln sistem dogodkov, velja:
$$
P(H_i|A)=\frac{P(H_i)P(A|H_i)}{P(H_1)P(A|H_1)+P(H_2)P(A|H_2)+P(H_3)P(A|H_3)+\cdots}
$$

\textsc{Definicija:} Dogodka $A$ in $B$ sta \textbf{neodvisna}, če velja $P(A\cap B)=P(A)P(B)$.

\section*{Kvantili} %%%%% KVANTILI

\textsc{Definicija: }Število $x_\alpha$ je \textbf{kvantil} slučajne spremenljivke $X$ za verjetnost $\alpha$, če velja:

$$
P(X < x_\alpha) \leq \alpha, \quad P(X \leq x_\alpha) \geq \alpha.
$$

Kvantilu za verjetnost $1/2$ pravimo \textbf{mediana}, kvantiloma za verjetnosti $1/3$ in $2/3$ pravimo prvi in drugi \textbf{tercil}, kvantili za verjetnosti $1/4, 2/4, 3/4$ so \textbf{kvartili}, kvantili za verjetnosti $0.1, 0.2,\ldots, 0.9$ so \textbf{decili}, kvantili za verjetnosti $0.01, 0.02,\ldots, 0.09$ pa so \textbf{centili} ali \textbf{percentili}.

Če je $X$ zvezno porazdeljena in je $x_\alpha$ kvantil za verjetnost $\alpha$, velja kar $F_X(x_\alpha)=\alpha$. Če ima $X$ v okolici točke $x_\alpha$ strogo pozitivno gostoto, je $x_\alpha$ edini kvantil za verjetnost $\alpha$. Brž, ko je torej gostota na nekem intervalu strogo pozitivna, izven tega intervala pa enaka noč, so kvantili za vse verjetnosti iz $(0,1)$ natančno določeni.

\section*{Slučajne spremenljivke}

\textsc{Izrek:} Naj bo $X$ zvezno porazdeljena slučajna spremenljivka z zalogo vrednosti v odprti množici $A \subseteq \R^n$ in gostoto $f_X$. Nadalje naj bo dana zvezno odvedljiva bijekcija $h: A \longrightarrow B$, pri čemer naj bo $h'(x) \neq 0$ za vse $x \in A$. Tedaj ima slučajni vektor $Y := h(X)$ gostoto:
$$
f_Y(y) = \begin{cases} f_X(h^{-1}(y))|(h^{-1})'(y)| & y\in B \\ 
0 & \mbox{sicer} \end{cases}
$$

\textsc{Izrek:} Naj bo $X$ zvezno porazdeljena slučajna spremenljivka z gostoto $f_X$, skoncentrirana na dovolj lepi množici $A$. Če je $h: A \longrightarrow \R$ lokalno Lipschitzeva:
$$
|h(x)-h(y)| \leq k |x-y|
$$
in $P(h\text{ v $X$ ni odvedljiva ali }h'(X)=0)=0$, je slučajna spremenljivka $Y$ porazdeljena zvezno z gostoto
$$
f_Y(y)= \sum_{x\in A;h(x)=y}\frac{f_X(x)}{|h'(x)|}.
$$

\textsc{Definicija:} Porazdelitev slučajnega vektorja $(X,Y)$ podamo z verjetnostmi $P((X,Y)=(x,y))=P(X=x,Y=y)$ (\textbf{skupna ali navzkrižna porazdelitev} slučajnih spremenljivk $X$ in $Y$). Porazdelitve komponent imenujemo \textbf{robne porazdelitve}: $P(X=x)=\sum_yP(X=x,Y=y)$, $P(Y=y)=\sum_xP(X=x, Y=y)$. $X$ in $Y$ sta \textbf{neodvisni}, brž ko za poljuvna $x$ in $y$ velja $P(X=x,Y=y)=P(X=x)P(Y=y)$.

\textsc{Posledica:} Će sta $S \sim Bin(m,p)$ in $T \sim Bin(n,p)$ neodvisni slučajni spremenljivki, je $U := S+T \sim Bin(m+n,p)$.

\textsc{Posledica:}Če sta $S \sim NegBin(m,p)$ in $T \sim NegBin(n,p)$ neodvisni slučajni spremenljivki, je $U := S+T \sim NegBin(m+n,p)$.

\textsc{Definicija:} Porazdelitev zveznega dvorazsežnega vektorja $(X,Y)$ lahko opišemo z \textbf{dvorazsežno gostoto} $f_{X,Y}$, za katero velja:
$$
P(a \leq X \leq b, c \leq Y \leq d) = \int_a^b\int_c^d f_{X,Y}(x,y=dy dx.
$$
Splošneje, porazdelitev zveznega slučajnega vektorja $X \in \R^n$ lahko opišemo z $n$-\textbf{razsežno gostoto} $f_X$, ki ima to lastnost, da za vsako merljivo množico $A \subseteq \R^n$ velja:
$$
P(X \in A) = \int_A f_X(x)dx.
$$
Seveda velja: $\int_{\R^n}f_X(x)dx=1.$

Če sta $X$ in $Y$ slučajna vektorja z vrednostmi v $\R^m$ in $\R^n$ in je slučajni vektor $(X,Y)$ porazdeljen zvezno z $(m+n)$-razsežno \textbf{skupno} gostoto $f_{X,Y}$, sta tudi njegovi komponenti $X$ in $Y$ porazdeljeni zvezno, in sicer z \textbf{robnima gostotama}:
$$
f_X(x) = \int_{\R^n}f_{X,Y}(x,y)dy \qquad f_Y(y) = \int_{\R^m}f_{X,Y}(x,y)dx.
$$
Zvezno porazdeljena slučajna vektorja $X$ in $Y$ sta neodvisna natanko tedaj, ko je tudi slučajni vektor $(X,Y)$ porazdeljen zvezno z gostoto: $f_{X,Y}(x,y) = f_X(x)f_Y(y).$

\textsc{Izrek:} Naj bo $X$ zvezno porazdeljen slučajni vektor z zalogo vrednosti v odprti množici $A \subseteq \R^n$ in gostoto $f_X$. Nadalje naj bo $h: A \longrightarrow B$ difeomorfizem razreda $C^(1)$. Tedaj ima slučajni vektor $Y:=h(X)$ gostoto:
$$
f_Y(y) = \begin{cases} f_X(h^{-1}(y))|J(h^{-1})(y)| & y\in B \\ 
0 & \mbox{sicer} \end{cases}
$$
\textsc{Izrek:} Naj bo $X$ zvezno porazdeljen $n$-razsežen slučajni vektor z gostoto $f_X$, skoncentriran na merljivi množici $A$. Če je $h: A \longrightarrow \R^n$ lokalno Lipschitzeva in $P(h\text{ v $X$ nidiferenciabilna ali }Jh(X)=0)=0$, je slučajni vektor $Y$ porazdeljena zvezno z gostoto
$$
f_Y(y)= \sum_{x\in A;h(x)=y}\frac{f_X(x)}{|Jh(x)|}.
$$

\textsc{Porazdelitev vsote in razlike:} 
\begin{itemize}
\item $Z=X+Y$: $f_Z(z) = \int_{-\infty}^{\infty}f_{X,Y}(x,z-x)dx = \int_{-\infty}^{\infty}f_{X,Y}(z-y,y)dy$

 \item  $W=X-Y$: $f_W(w) = \int_{-\infty}^{\infty}f_{X,Y}(x,x-w)dx = \int_{-\infty}^{\infty}f_{X,Y}(w+y,y)dy$
\end{itemize}

\section*{Matematično upanje (pričakovana vrednost)}

\textsc{Definicija:} Za diskretne slučajne spremenljivke: $E(X)=\sum_x xP(X=x)$, $E[h(X)]=\sum_x h(x)P(X=x).$ Če je vrednosti neskončno, matematično upanje obstaja, če je vrsta absolutno konvergentna. 

Velja: $E(aX+b)=aE(X)+b$.

\textsc{Definicija:} Za zvezne slučajne spremenljivke: $E(X) = \int_{-\infty}^{\infty}xf_X(x)dx$,x $E[h(X)] = \int_{-\infty}^{\infty}h(x)f_X(x)dx$. Spet le-to obstaja, če integral absolutno konvergira.

\textsc{Definicija:} \textbf{Varianca (disperzija)}: $\text{var}(X) = E((X-E(X))^2) = E(X^2)-(E(X))^2$. \textbf{Standardni odklon}: $\sigma (X) = \sqrt{\text{var}(X)}$.

Velja: $\text{var}(aX+b)=a^2\text{var}(X).$
$$
E[h(X,Y)] = \sum_x \sum_y h(x,y)P(X=x,Y=y)
$$
$$
E[h(X,Y)] = \iint_{\R^2}h(x,y)f_{X,Y}(x,y)dx dy.
$$
$$
E(X+Y)=E(X)+E(Y)
$$
\textsc{Definicija:} \textbf{Indikator} dogodka je slučajna spremenljivka, ki je na danem dogodku enaka 1, zunaj njega pa 0. Indikator dogodka $A$ bomo označevali z $\mathds{1}_\mathcal{A}.$ Indikator dogodka, da je izjava $\mathcal{A}$ pravilna, bomo označevali z $\mathds{1}(\mathcal{A}).$ Velja $E(\mathds{1}_\mathcal{A}) = P(\mathcal{A}).$

\textsc{Definicija: }Slučajni spremenljivki $X$ in $Y$ sta \textbf{nekorelirani}, če velja $E(XY) = E(X)E(Y).$ Poljubni neodvisni slučajni spremenljivki sta nekorelirani, obratno pa ni nujno res!

Slučajni spremenljivki $X$ in $Y$ sta zagotovo neodvisni v naslednjih treh primerih:
\begin{itemize}
\item če sta nekorelirani in posamezna slučajna spremenljivka lahko zavzame kvečjemu dve vrednosti;
\item če sta nekorelirani in je njuna skupna porazdelitev dvorazsežna normalna;
\item če za poljubni omejeni merljivi funkciji $g$ in $h$ velja, da sta slučajni spremenljivki $g(X)$ in $h(Y)$ nekorelirani.
\end{itemize}

Brž ko sta $X$ in $Y$ nekorelirani in imata varianco, velja: $\text{var}(X+Y) = \text{var}(X)+\text{var}(Y)$.

\textsc{Definicija:} \textbf{Kovarianca}: $\text{cov}(X,Y):=E((X-E(X))(Y-E(Y)))=E(XY)-E(X)E(Y)$. 

Velja $\text{cov}(X,X) = \text{var}(X)$ in $\text{cov}(X,Y) = \text{cov}(Y,X)$. Če sta $a$ in $b$ konstanti, velja $\text{cov}(X+a,Y+b)=\text{cov}(X,Y)$ in $\text{cov}(aX+bY,Z) = a\text{cov}(X,Z) + b\text{cov}(Y,Z)$. 

Slučajni spremenljivki $X$ in $Y$ sta nekorelirani natanko tedaj, ko je $\text{cov}(X,Y)=0$. 

\textsc{Definicija:} \textbf{Korelacijski koeficient}: $\text{corr}(X,Y)=\frac{\text{cov}(X,Y)}{\sigma (X) \sigma (Y)}$.

Velja: $-1 \leq \text{corr}(X,Y)\leq 1$. Če so $a,b,c$ in $d$ konstante ter $a,c > 0$, velja $\text{corr} (aX+b,cY+d) = \text{corr}(X,Y)$.

\section*{Pogojne porazdelitve}

\textsc{Definicija:} \textbf{Pogojno porazdelitev} slučajne spremenljivke $X$ glede na dogodek $B$ opišemo s pogojnimi verjetnostmi $P(X\in C | B)$, kjer $C$ preteče vse merljive množice. Če je $X$ diskretna, lahko njeno pogojno porazdelitev opišemo s \textbf{pogojno porazdelitveno shemo}:
$$
\begin{pmatrix}
a_1 & a_2 & \cdots \\
P(X=a_1 | B) & P(X=a_2 | B) & \cdots
\end{pmatrix}
$$

\textsc{Definicija:} Za vsako realno slučajno spremenljivko in vsak dogodek $B$ s pozitivno verjetnostjo lahko pogojno porazdelitev slučajne spremenljivke $X$ glede na $B$ opišemo s \textbf{pogojno kumulativno porazdelitveno funkcijo}:
$$
F_{X|B}(x) = P(X \leq x | B).
$$
Če je pogojna porazdelitev zvezna, obstaja tudi \textbf{pogojna porazdelitvena gostota}:
$$
f_{X|B}(x) = F_{X|B}'(x).
$$
Brž ko je $X$ zvezno porazdeljena, je tudi njena pogojna porazdelitev zvezna glede na vsak dogodek s pozitivno verjetnostjo.

Če je $X$ porazdeljena zvezno z gostoto $f_X$ in $P(X \in C) > 0$, je:
$$
f_{X|X\in C}(x) = \begin{cases} \frac{f_X(x)}{P(X \in C)}; & x\in C \\ 
0; & \mbox{sicer} \end{cases}
$$
Podobno velja tudi za zvezne slučajne vektorje.

\textsc{Definicija:} \textbf{Pogojno matematično upanje} slučajne spremenljivke $X$ glede na dogodek $B$ je matematično upanje, ki pripada ustrezni pogojni porazdelitvi, in ga označimo z $E(X | B)$. Tako velja:
$$
E(X | B) = \sum_x xP(X=x | B)
$$
in splošneje:
$$
E(h(X) | B) = \sum_x h(X) P(X=x | B)
$$
Pogojno matematično upanje ima vse lastnosti običajnega matematičnega upanja, npr. linearnost. 

\textsc{Izrek:} Za vsako slučajno spremenljivko $X$ z matematičnim upanjem in vsak popoln sistem dogodkov $H_1, H_2, H_3, \ldots $ velja \textbf{izrek o polnem matematičnem upanju}:
$$
E(X) = P(H_1)E(X|H_1) + P(H_2)E(X | H_2) + P(H_3)E(X | H_3) + \cdots
$$



\end{document}

