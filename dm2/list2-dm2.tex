\documentclass[8pt,a4paper]{amsart}
% ukazi za delo s slovenscino -- izberi kodiranje, ki ti ustreza
\usepackage[slovene]{babel}
%\usepackage[cp1250]{inputenc}
%\usepackage[T1]{fontenc}
\usepackage[utf8]{inputenc}
\usepackage{amsmath,amssymb,amsfonts}
\usepackage{url}
%\usepackage[normalem]{ulem}
\usepackage{enumerate}
\usepackage[dvipsnames,usenames]{color}
\usepackage{bbold}
\usepackage{stmaryrd}



\usepackage[
top    = 1cm,
bottom = 1cm,
left   = .5cm,
right  = 0.5cm]{geometry}
%
%% ne spreminjaj podatkov, ki vplivajo na obliko strani
%\textwidth 19cm
%\textheight 27cm
%\oddsidemargin-1.5cm
%\evensidemargin-1.5cm
%\topmargin-30mm
%%\addtolength{\footskip}{10pt}
%\pagestyle{plain}
%%\overfullrule=15pt % oznaci predlogo vrstico


% ukazi za matematicna okolja
\theoremstyle{definition} % tekst napisan pokoncno
\newtheorem{definicija}{Definicija}[section]
\newtheorem{primer}[definicija]{Primer}
\newtheorem{opomba}[definicija]{Opomba}
\newtheorem{zgled}[definicija]{Zgled}

\theoremstyle{plain} % tekst napisan posevno
\newtheorem{lema}[definicija]{Lema}
\newtheorem{izrek}[definicija]{Izrek}
\newtheorem{trditev}[definicija]{Trditev}
\newtheorem{posledica}[definicija]{Posledica}


\newcommand{\R}{\mathbb R}
\newcommand{\N}{\mathbb N}
\newcommand{\Z}{\mathbb Z}
\newcommand{\C}{\mathbb C}
\newcommand{\Q}{\mathbb Q}

\begin{document}
\thispagestyle{empty}
\setlength{\parindent}{0pt}

\vspace{-2ex}
\section{Permutacijske grupe}

Naj grupa $G$ deluje na $X$. Definiramo relacijo: $x \sim y \Longleftrightarrow
\exists g \in G: g(x)=y$.

\textsc{Trditev:} $\sim$ je ekvivalenčna relacija na $X$.

\textsc{Definicija:} \textbf{Orbite} (glede na delovanje $G$ na $X$) so
ekvivalenčni razredi relacije $\sim$, velja torej: $Gx = \{ y \in X; g(y) = x
\}$.

$G\cdot x$... orbita elementa $x$, $G(x \rightarrow y) = \{ g \in G; g(x)=y \}$,
$G_x$... stabilizator elementa $x$: $G(x \rightarrow x)$

\textsc{Izrek:} Če je $G$ končna permutacijska grupa, ki deluje na $X$, tedaj je
za vsak $x \in X$:  $|G| = |Gx| |G_x|$.

\textsc{Definicija}: Naj bo $G$ grupa, ki deluje na $X$. Za $g \in G$ je $F(g) =
\{ x\in X; g(x) = x \}$ množica negibnih točk permutacije $g$.

\textsc{Burnsideova lema:} Število orbit pri delovanju $G$ na $X$ je enako:  $\frac{1}{|G|}
\sum_{g \in G}|F(g)|$.

\textsc{Definicija:} Naj bo $G$ grupa in $X$ množica. \textbf{Reprezentacija}
$G$ s permutacijami nad $X$ je predpis $g \in G \mapsto \hat{g} \text{
permutacija } X$, tako da je $\widehat{g_1g_2} = \widehat{g_1}\widehat{g_2}$ za
vse $g_1,g_2 \in G$.

$\widehat{G} = \{ \widehat{g} ; g\in G \}$ je (permutacijska) grupa.

\textsc{Definicija:} Reprezentacija je \textbf{zvesta}, če je $\widehat{g_1} =
\widehat{g_2} \Longleftrightarrow g_1 = g_2$.

\textsc{Trditev:} Vsaka končna grupa premore zvesto reprezentacijo.


%%%%%%%% SIMETRIJE IN ŠTETJE
\vspace{-2ex}
\section{Simetrije in štetje}

Naj bo $\alpha_i$ število disjunktnih ciklov dolžine $i$ v $\pi$ zapisanem kot
produkt disjunktnih ciklov. ($\alpha_1 = $ število negibnih točk $\pi$.)

Če $|\pi| = n$, potem $\alpha_i + 2\alpha_2 + \ldots + n\alpha_n= n$.

$z(\pi ; x_1,\ldots ,x_n)=x_1^{\alpha_1}x_2^{\alpha_2}\cdots x_n^{\alpha_n}$
imenujemo \textbf{ciklični indeks permutacije $\pi$}

\textsc{Definicija:} $G$ permutacijska grupa, tedaj je \textbf{ciklični indeks
grupe} $G$:
$$ Z(G;x_1,\ldots ,x_n) = \frac{1}{|G|} \sum_{g\in G} z(g;x_1,\ldots ,x_n).$$

Vrtiljaku ustreza ciklična grupa, ogrlici pa diedrska. $D_{2n}$ je grupa
simetrij pravilnega $n$-kotnika.

\textsc{Izrek:} $Z(C_n; x_1,\ldots, x_n) = \frac{1}{n} \displaystyle \sum_{d
\mid n} \phi (d)x_d^{\frac{n}{d}}$, \quad $\phi(p^k) = p^k(p-1)$.

\begin{tabular}[h!]{*{21}{|c}|} \hline
$n$ & 1 & 2 & 3 & 4 & 5 & 6 & 7 & 8 & 9 & 10 & 11 & 12 & 13 & 14 & 15 & 16 & 17 & 18 &
19 & 20 \\ \hline
$\phi(n)$ & 1 & 1 & 2 & 2 & 4 & 2 & 6 & 4 & 6 & 4 & 10 & 4 & 12 & 6    & 8 & 8  & 16 & 6 &
18 & 8  \\ \hline
\end{tabular}


\textsc{Izrek:} $Z(D_{2n}; x_1,\ldots,x_n) = \frac{1}{2} Z(C_n;x_1,\ldots,x_n) +
\begin{cases} \frac{1}{4}(x_1^2x_2^{\frac{n}{2}-1}+x_2^{\frac{n}{2}});& n \text{
  sod} \\ \frac{1}{2}x_1x_2^{\frac{n-1}{2}};& n \text{ lih} \end{cases}$

Delovanje na ploskve nekega telesa je enako kot delovanje na oglišča dualnega
telesa. Telesa in njihovi duali:
\begin{itemize}
  \item kocka $\leftrightarrow$ oktaeder
  \item tetraeder $\leftrightarrow$ tetraeder
  \item ikozaeder (12 oglišč, 30 robov, 20 ploskev, ploskve trikotniki) $\leftrightarrow$ dodekaeder (20
    oglišč, 30 robov, 12 ploskev, ploskve petkotniki)
\end{itemize}

\begin{tabular}{| c || c | c | c |}\hline
  polieder & $|X|$ & $|G|$ & $Z$ \\ \hline\hline
  tetraeder & 4 & 12 & $\frac{1}{12}(x_1^4+8x_1x_3 + 3x_2^2$\\ \hline
  oktaeder & 6 & 24 & $\frac{1}{24}(x_1^6+6x_1^2x_4 + 3x_1^2x_2^2+6x_2^3+8x_3^2$\\ \hline
  kocka & 8 & 24 & $\frac{1}{24}(x_1^8+8x_1^2x_2^2 + 9x_2^4 + 6x_4^2)$ \\ \hline
  ikozaeder & 12 & 60 & $\frac{1}{60}(x_1^{12} + 24x_1^2x_5^2 + 15x_2^6 + 20x_3^4)$ \\ \hline
  dodekaeder & 20 & 60 & $\frac{1}{60}(x_1^{20} + 20x_1^2x_3^6 + 15x_2^{10} + 24x_5^4)$ \\ \hline
\end{tabular}

%%%%% BARVANJA
\vspace{-2ex}
\section{Število neekvivalentnih barvanj}

$G$ grupa, ki deluje na $X$, $|X|=n$, $K$ naj bo množica $r$-barv, $w: X
\longrightarrow K$ je $r$-barvanje $X$, $\Omega = \{ w: X \longrightarrow K \}$,
$|\Omega| = r^n$

$\widehat{g}: \Omega \longrightarrow \Omega$ ($w \mapsto \widehat{g}(w)$). $g$
je avtomorfizem grafa. Velja: $(\widehat{g}(w))(x) = w(g^{-1}(x))$.

\textsc{Lema:} Preslikava~~$\widehat{\cdot}$~~je zvesta reprezentacija grupe
$G$.

Grupi $G$ in $\widehat{G} = \{ \widehat{g}: \Omega \longrightarrow \Omega \}$
sta izomorfni.

\textsc{Definicija:} Barvanji sta \textbf{ekvivalentni}, če sta v isti orbiti
grupe $\widehat{G}$, oz. število neekvivalentnih barvanj $X$ glede na $G$ je
število orbit $G$.

\hfill

\textsc{Izrek:} Naj bo $G$ grupa, ki deluje na $X$ in $r\geq 2$ število barv. Tedaj je
število neekvivalentnih barvanj $X$ enako $Z(G;r,\ldots ,r)$.

\hfill

$K = \{ a,b,\ldots k\}$, $U(a,b,\ldots, k)$... rodovna funkcija za vsa
neekvivalentna barvanja glede na delovanje grupe $G$ na $n$-množico $X$.

\textsc{Izrek Polya:} Če $G$ deluje na $n$-množico $X$ in je $K = \{ a,b,\ldots
,k\}$ množica barv, tedaj je $$ U(a,b,\ldots, k) = Z(G;\sigma_1,\ldots
,\sigma_n),\text{ kjer je } \sigma_i = a^i + b^i + \cdots + k^i \quad (1\leq i
\leq n) $$

\vspace{-2ex}
\section{Ramseyeva teorija}

\textsc{Trditev:} Naj bodo povezave $K_n$ pobarvane z dvema barvama in naj bo
$r_i$ število povezav iz $i$-tega vozlišča barve 1. Tedaj je število
monokromatičnih trikotnikov enako $\binom{n}{3} - \frac{1}{2}\sum_{i=1}^n
r_i(n-1-r_i)$.

\textsc{Posledica:} V situaciji iz zadnje trditve imamo vsaj $\binom{n}{3} -
\lfloor \frac{n}{2} \lfloor (\frac{n-1}{2})^2 \rfloor \rfloor $ monokromatičnih
trikotnikov.

\textsc{Ramseyev izrek:} Naj bo $r \geq 1$ in $a_1, a_2 \geq r$. Tedaj obstaja
tako najmanjše naravno število $N(a_1,a_2;r)$, da velja naslednje: naj bo $S$
$n$-množica, kjer je $n \geq N(a_1,a_2;r)$ in recimo, da smo vse njene
$r$-podmnožice pobarvali z barvo 1 oz. barvo 2. Tedaj $S$ premore
$a_1$-podmnožico, tako da so vse njene $r$-podmnožice barve 1, ali pa $S$
premore $a_2$-podmnožico, da so vse njene $r$-podmnožice barve 2.

\textsc{Posledica:} $N(a_1,a_2;r)\leq N(N(a_1-1,a_2;r),N(a_1,a_2-1;r);r-1)+1$.

\textsc{Izrek:} $N(a_1,a_2;2) \leq \binom{a_1+a_2-2}{a_1-1}$.

$r=2$:
\begin{tabular}{|c||c|c|c|c|c|c|c|c|} \hline
$a_1 \backslash a_2$ & 3 & 4 & 5 & 6 & 7 & 8 & 9 & 10 \\ \hline \hline
3 & 6 & 9 & 14 & 18 & 23 & 28 & 36 & 40/42 \\ \hline
4 &   & 18 & 25 & 36/41 & 49/61 & 58/84 & 73/115 & 92/149 \\ \hline
5 & & & 43/49 & 58/87 & 80/143 & 101/216 & 126/316 & 144/442 \\ \hline
6 & & & & 102/165 & 113/298 & 132/495 & 169/780 & 179/1171 \\ \hline

\end{tabular}

\textsc{Izrek:} Če je $a \geq 3$, tedaj je $N(a,a;2) \geq 2^{\frac{a}{2}}$.

\textsc{Izrek (Erdős, Szekeres):} Za vsak $n \geq 3$ obstaja tako najmanjše
naravno število $N$, tako da če imamo $N$ točk v ravnini v splošni legi (nobene
3 niso kolinearne), potem med njimi obstaja $n$ točk, ki določajo konveksen
$n$-kotnik.

\textsc{Definicija:} Naj bodo $G_1,\ldots ,G_k$ grafi. \textbf{Grafovsko
Ramseyevo število} $N(G_1,\ldots ,G_k)$ je najmanjši tak $N$, da če povezave
polnega grafa $K_N$ pobarvamo poljubno z barvami $1, 2, \ldots ,k$, tedaj v tem
$K_N$ najdemo vsaj en $G_i$, ki je barve $i$.

To število obstaja!

\textsc{Izrek:} Če je $T$ drevo z $n$ vozlišči, tedaj je $N(T,K_n) = (n-1)(n-1)+1$.

\vspace{-2ex}
\section{Osnove metrične teorije grafov in matrika sosednosti}

\textsc{Definicije:}
\begin{enumerate}
\item \textbf{Interval $I_G(u,v)$} med vozliščema $u$ in $v$ v grafu $G$ je množica vozlišč, ki ležijo na najkrajši $u,v$-poti.
\item \textbf{Ekscentričnost vozlišča} $u$ grafa $G$, ecc$_g(u) = \max{\{d(u,x);x \in V(G) \}}$.
\item \textbf{Polmer} grafa $G$: rad$(G) = \min_u{\max_v{\{ d(u,x)\}}}$ = minimalna ekscentričnost grafa.
\item \textbf{Center} grafa $G$, $C(G)$, je množica vozlišč, ki realizirajo polmer.
\item \textbf{Premer} grafa, diam$(G) = \max_{u,v \in V(G)}d(u,v)$.
\item Podgraf $H$ grafa $G$ je \textbf{izometrični podgraf}, če velja: $\forall u,v \in V(H): d_H(u,v) = d_G(u,v)$.
\item Podgraf $H$ grafa $G$ je \textbf{konveksen}, če velja: $x,y \in V(H) \Longrightarrow I_G(x,y) \subseteq V(H)$.
\end{enumerate}

\textsc{Izrek:} Če je $G$ graf z $V(G) = \{ v_1,\ldots ,v_n \}$, tedaj je
$(A(G)^r)_{ij}$ število sprehodov med $v_i$ in $v_j$ dolžine $r$.

\textsc{Posledica:} Če je $G$ povezan graf, potem so diagonalci v $A(G)^2$
stopnje vozlišč grafa, sled matrike $A(G)^3$ je 6-krat število trikotnikov grafa
$G$.

$S_k(G) = \sum_{i=0}^kA(G)^i = I + A(G) + \cdots + A(G)^k$.

\textsc{Posledica:} rad$(G) $ je najmanjši $k$, tak da $S_k(G)$ premore vrstico
brez ničel in $C(G)$ ustreza vsem vozliščem s takimi vrsticami. diam$(G)$ je
najmanjši $k$, tak da $S_k(G)$ ne vsebuje nobene ničle.

\vspace{-2ex}
\section{Vložitve metričnih prostorov v grafe}

\textsc{Definicija:} Metrični prostor $(M,d)$ je \textbf{realiziran} z omrežjem
$G = (V,E,w)$, $w:E\longrightarrow \R^+$, če je $M \subseteq V$ in je $d(x,y) =
d_G(x,y)$ $\forall x,y \in M$. Realizacija $G=(V,E,w)$ metričnega prostora
$(M,d)$ je \textbf{optimalna:}, če je $w(G) = \sum_{e \in E} w(e)$ najmanjši
možen.

\textsc{Izrek:} Vsak metrični prostor premore realizacijo z omrežjem. Vsak
končen metričen prostor premore optimalno realizacijo z grafom.

\textsc{Lema:} Naj bo $(M,d)$ metrični prostor z $|M|=n$ in naj bo $G=(V,E,w)$
njegova realizacija, kjer je $|V| > 2^{\binom{n}{2}+1}+n$. Tedaj $(M,d)$ premore
realizacijo s pravim podgrafom od $G$, ki ima kvečjemu  $2^{\binom{n}{2}+1}+n$
vozlišč.

\textsc{Definicija:} Metrični prostor $(M,d)$ zadošča \textbf{pogoju 4 točk}, če
za vsako četverico $x,y,z,t \in M$ velja: $d(x,y)+d(z,t) \leq
\max{\{d(x,z)+d(y,z), d(x,t)+d(y,t)\}}$.

Ekvivalenten pogoj: $s_1 = d(x,y)+d(z,t), s_2 = d(x,z)+d(y,z), s_3=
d(x,t)+d(y,t)$, potem velja, da sta največja $s$-a enaka.

\textsc{Izrek:} Graf $G$ je drevo natanko tedaj, ko je povezan, brez trikotnikov
in njegova metrika zadošča pogoju 4 točk.

\textsc{Izrek:} Končen metričen prostor lahko karakteriziramo z drevesom natanko
tedaj, ko njegove točke zadoščajo pogoju 4 točk. V tem primeru je realizacija
enolična in jo lahko najdemo v polinomskem času.

\vspace{-2ex}
\section{Wienerjev indeks}

\textsc{Definicija:} \textbf{Wienerjev indeks} grafa $G$ je $W(G) =
\sum_{\{u,v\}}d_G(u,v) = \frac{1}{2}\sum_u \sum_v d_G(u,v)$.

\textsc{Izrek:} Če je $T$ drevo, tedaj je $W(T) = \sum_e n_1(e) n_2(e)$, kjer
sta $n_1(e), n_2(e)$ števbili vozlišč v povezanih komponentah grafa $T-e$.

\textsc{Definicija:} $G,H$ grafa. Tedaj je \textbf{kartezični produkt} $G
\square H$ grafov $G$ in $H$ graf z: $$ V(G \square H) = V(G) \times
V(H)\text{  in } (g,h) \sim (g',h') \Longleftrightarrow g=g' \text{ in } hh' \in
E(H) \text{ ali } gg' \in E(G) \text{ in } h=h'.  $$

\textsc{Lema o razdalji:} Če sta $G$ in $H$ povezana grafa, tedaj je $d_{G
\square H} = d_G + d_H$, natančneje $d_{G \square H}((g,h),(g',h')) =
d_G(g,g') + d_H(h,h')$.

\textsc{Trditev:} Če sta $G$ in $H$ povezana grafa, tedaj je $W(G\square H) =
|V(G)|^2W(H) + |V(H)|^2W(G)$.

\textsc{Posledica:} $W(Q_n) = n2^{2(n-1)}$; $n \geq 1$.

%%%% VAJE
\vspace{-2ex}
\section{Vaje}

\textsc{Trditev:} Naj bo $X = X_1  \amalg X_2 $, $|X_1|=n_1, |X_2|=n_2,|X|=n$.
Naj $G_1$ deluje na $X_1$ in $G_2$ na $X_2$. Potem velja, da je
$$ Z(G_1 \times G_2 ;x_1,x_2,\ldots x_n) = Z(G_1;x_1,\ldots ,x_{n_1}) Z(G_2 ;x_1,\ldots , x_{n_2}),$$
kjer $G_1 \times G_2$ deluje na $X$ na naraven način.

$N(2,k;2)=k$

\textsc{Definicija:} Naj bo $G$ graf. Podmnožica $A \subseteq V(G)$ je
\textbf{klika} v $G$, če sta vsaki dve vozlišči iz $A$ sosednji v $G$.
Podmnožica $B \subseteq V(G)$ je \textbf{neodvisna množica} v $G$, če nobeni dve
vozlišči iz $B$ nista sosednji v $G$.

\textsc{Lema:} V grafu z $n$ vozlišči obstaja klika velikosti $\lfloor
\frac{1}{2}\log_2{n}\rfloor$ ali neodvisna množica velikosti $\lfloor
\frac{1}{2}\log_2{n}\rfloor$.

\textsc{Trditev:} Naključni graf na $n$ vozliščih ima (z veliko verjetnostjo,
tj. $P\to1$ ko $n\to\infty$) največjo kliko in največjo
neodvisno množico velikosti $C\log{n}$.

\textsc{Lema:} Naključni graf na $n$ vozliščih ima največjo kliko in neodvisno
množico velikosti $\leq 2\log{n}$.

\textsc{Trditev:} Naj bosta $N(a-1,b;2)$ in $N(a,b-1;2)$ obe sodi. Potem velja:
$$ N(a,b;2) \leq N(a-1,b;2) + N(a,b-1;2) - 1.$$

\textsc{Lema:} Za vsak $m \in \N$ obstaja tak $n \in \N$, da vsako zaporedje
 $n$ realnih števil vsebuje monotono podzaporedje dolžine $m$. $n = N(m,m;2)$.

\textsc{Lema:} Za $n \geq N(m,m;2)$ ima vsaka $\{ 0,1 \}$-matrika glavno
podmatriko velikosti $m$, v kateri so vsi elementi nad diagonalo enaki. Za $n =
N(N(m,m;2),N(m,m;2);2)$ ima matrika glavno podmatriko, v kateri so vsi elementi
nad diagonalo enaki in vsi elementi pod diagonalo enaki.

\textsc{Lema:} $N(\underbrace{3,3,\ldots ,3}_k;2) \leq \lfloor e k! \rfloor + 1$.

Velja: $\lfloor ek! \rfloor = 1 + \lfloor e(k-1)! \rfloor k$.

\textsc{Schurov izrek:} Za vsak $k \in \N$ obstaja $n \in \N$, da za vsako
$k$-barvanje množice $[n]$ obstajajo števila $x,y,z \in [n]$ iste barve z
lastnostjo $x+y=z$.

\textsc{Lema:} $N(2K_3,K_3)=8$, $N(mK_3,mK_3)=5m$ za $m\geq 2$.

$T$ drevo. Potem $C(T)$ vsebuje bodisi 1 bodisi 2 vozlišči.

$\operatorname{rad}(G) \leq \operatorname{diam}(G) \leq 2\operatorname{rad}(G)$

$W(K_{1,n}) = n^2, n\geq 3$, $W(P_n) = \frac{1}{6}n(n^2-1), n\geq 2$, $W(K_{1,n}
\square P_n) = \frac{1}{6}(n+1)^2n(n^2-1)$.

Pogoj 4 točk implicira trikotniško neenakost.

Razdalja v drevesu zadošča pogoju 4 točk.

%%% UPORABNO
\vspace{-2ex}
\section{Uporabno}

$D_{2n} = $ grupa simetrij pravilnega $n$-kotnika. $|D_{2n}| = 2n$, $D_{2n} =
\langle \text{rotacija za }\frac{2\pi}{n},\text{zrcaljenje}\rangle$. Naj bo
$r=\text{ rotacija za }\frac{2\pi}{n}$ in $z$ zrcaljenje. Velja $r^iz =
zr^{-i}$. Vsi elementi $D_{2n}$ so oblike: $r^i$, $i=0,1,\ldots ,n-1$ in $zr^i$,
$i=0,1,\ldots ,n-1$.

Koliko je različnih ogrlic iz kroglic različnih barv? Poišči grupo
avtomorfizmov, oz. njene predstavnike in poglej, koliko že pobarvanih ogrlic
fiksirajo. Npr. $id$ fiksira $\binom{n}{r}$ ogrlic, kjer je $n$ število biserov
in $r$ število barv.

Število različnih objektov z npr. $k$ belimi in $l$ črnimi deli: uporabi izrek
Polya in iščeš koeficient pred $b^kc^l$.

Ciklični indeksi $S_2$, $S_3$, $S_4$, $S_5$, ki delujejo na $\Z_2$, $\Z_3$, $\Z_4$, $\Z_5$:
\begin{itemize}
 \item $Z(S_2;x_1,x_2)=\frac{1}{2}(x_1^2+x_2)$
 \item $Z(S_3;x_1,x_2,x_3)=\frac{1}{6}(x_1^3+3x_1x_2+2x_3)$
 \item $Z(S_4;x_1,x_2,x_3,x_4)=\frac{1}{24}(x_1^4+6x_2x_1^2+3x_2^2+8x_3x_1+6x_4)$
 \item $Z(S_5;x_1,x_2,x_3,x_4,x_5)=\frac{1}{120}(x_1^5+10x_2x_1^2+15x_2^2x_1+20x_3x_1^2+20x_3x_2+30x_4x_1+24x_5)$
\end{itemize}

$Q_n$: $V(Q_n) = \{0,1 \}^n = \{ b_1\ldots b_n; b_i \in \{0,1\} \}$,
$u=b_1\ldots b_n, v = c_1\ldots c_n \in V(Q_n)$. $uv \in E(Q_n)
\Longleftrightarrow \exists! i: b_i \neq c_i$.

\hfill Avtor: Klemen Sajovec, manjši popravki: Jure Slak

\end{document}
