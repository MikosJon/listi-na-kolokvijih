\documentclass[10pt,a4paper]{amsart}
\usepackage[slovene]{babel}
\usepackage[T1]{fontenc}
\usepackage[utf8]{inputenc}
\usepackage{amsmath,amssymb,amsfonts}
\usepackage{url}
\usepackage{enumerate}
\usepackage{titlesec}
\titleformat*{\section}{\sc\bfseries\centering}

\usepackage[
top    = 1.7cm,
bottom = 1.7cm,
left   = 1.5cm,
right  = 1.5cm]{geometry}

% ukazi za matematicna okolja
\theoremstyle{definition} % tekst napisan pokoncno
\newtheorem{definicija}{Definicija}[section]
\newtheorem{primer}[definicija]{Primer}
\newtheorem{opomba}[definicija]{Opomba}
\newtheorem{zgled}[definicija]{Zgled}

\theoremstyle{plain} % tekst napisan posevno
\newtheorem{lema}[definicija]{Lema}
\newtheorem{izrek}[definicija]{Izrek}
\newtheorem{trditev}[definicija]{Trditev}
\newtheorem{posledica}[definicija]{Posledica}


\newenvironment{itemize*}%
{
\vspace{-6pt}
\begin{itemize}
\setlength{\itemsep}{0pt}
\setlength{\parskip}{1pt}
}
{\end{itemize}}

\newenvironment{enumerate*}%
{
\vspace{-6pt}
\begin{enumerate}
\setlength{\itemsep}{0pt}
\setlength{\parskip}{1pt}
}
{\end{enumerate}}


\newcommand{\dx}{\ensuremath{\,\mathrm{d}x}}
\newcommand{\dy}{\ensuremath{\,\mathrm{d}y}}
\newcommand{\dt}{\ensuremath{\,\mathrm{d}t}}
\newcommand{\dd}[1]{\ensuremath{\,\mathrm{d}#1}}
\let\oldint\int
\renewcommand{\int}{\oldint \!}


\newcommand{\R}{\mathbb R}
\newcommand{\N}{\mathbb N}
\newcommand{\Z}{\mathbb Z}
\newcommand{\C}{\mathbb C}
\newcommand{\Q}{\mathbb Q}

\pagestyle{empty}

\begin{document}
\thispagestyle{empty}
\setlength{\parindent}{0pt}

% \fontsize{9pt}{9pt}

\textbf{Trik} za neskončne sisteme NDE za $x_n(t)$: rešimo ga s pomočjo rodovne
funkcije $Q(y,t) = \sum_{n=1}^\infty x_n(t) y^n$. Velja: $yQ_y =
\sum_{n=1}^\infty nx_n(t)y^n, Q_y - Q / y = \sum_{n=1}^\infty nx_{n+1}y^n$.
Z upoštevanjem rekurzivne zveze dobimo PDE za $Q$, rešimo, razvijemo rešitev v
vrsto po $y$.


Zanimivi vzorci: $(x^2)^. = 2x\dot{x}$, $(xy)^. = \dot{x}y + x\dot{y}$,
$(\ln{x})^. = \frac{\dot{x}}{x}$.


%%%%%%%%%%%%%%%%%%%%%%%%%%%%%%%%%%%%%%%%%%%%%%%%%%%%%

Če $u = u(x,y)$ določa ploskev v prostoru, je enačba tangentne ravnine na to
ploskev v točki $(x,y,u(x,y))$ enaka: $u_x (X-x) + u_y (Y-y) - (U-u)=0$ (normala
ravnine je $(A,B,C) = (u_x,u_y,-1)$). Razdalja od tangentne ravnine do točke
$(a,b,c)$ je: $$ \text{d}(AX+BY+CU+D=0, (a,b,c)) =
\frac{|Aa+Bb+Cc+D|}{\sqrt{A^2+B^2+C^2}}.  $$

%%%%%%%%%%%%%%%%%%%%%%%%%%%%%%%%%%%%%%%%%%%%%%%%%%%%%

\section*{Eksistenčni izrek za nelinearne PDE 1. reda}

\textsc{Izrek:} Naj bo $u=u(x, y)$ rešitev začetnega problema $$
F(x,y,u,u_x,u_y)=0, \quad u(\alpha(s),\beta(s))=\gamma(s), \quad \text{za } s
\in \mathcal{I}.  $$ Če sta $p(s)=u_x(\alpha(s),\beta(s))$ in $q(s) =
u_y(\alpha(s),\beta(s))$ edini funkciji, za kateri velja: \begin{enumerate}
  \item $ (T) = \det{\begin{bmatrix} \alpha' & \beta' \\ F_p & F_q
  \end{bmatrix}}(s) \neq 0 \quad \forall s \in \mathcal{I},$ \item
    $F(\alpha(s),\beta(s),\gamma(s),p(s),q(s))=0 \quad \forall s \in
    \mathcal{I},$ \item $(p(s),q(s),-1) \cdot (\alpha'(s),\beta'(s),\gamma'(s))
      = 0 \quad \forall s \in \mathcal{I}.$ \end{enumerate} Potem je rešitev $u$
  enolična.

%%%%%%%%%%%%%%%%%%%%%%%%%%%%%%%%%%%%%%%%%%%%%%%%%%%%%


%%%%%%%%%%%%%%%%%%%%%%%%%%%%%%%%%%%%%%%%%%%%%%%%%%%%%


%%%%%%%%%%%%%%%%%%%%%%%%%%%%%%%%%%%%%%%%%%%%%%%%%%%%%


%%%%%%%%%%%%%%%%%%%%%%%%%%%%%%%%%%%%%%%%%%%%%%%%%%%%%

\section*{Separacija spremenljivk}

$L^2([-\pi,\pi]) = \{f:[-\pi,\pi]\longrightarrow \R, \int_{-\pi}^\pi |f|^2
\dx < \infty \}$ je vektorski prostor s skalarnim produktom $\langle f, g\rangle =
\int_{-\pi}^\pi f(x)g(x)\dx$. Množica funkcij

  $\{\frac{1}{2\pi},
  \frac{1}{\pi}\sin{x},\frac{1}{\pi}\cos{x},\frac{1}{\pi}\sin{2x},\frac{1}{\pi}\cos{2x},\ldots
\}$. je \underline{kompleten} (vsako funkcijo se da na enoličen način razviti v
tem sistemu),  \underline{ortonormiran} sistem za ta produkt.

\textbf{Fourierjev razvoj:} $f \in L^2([-\pi,\pi])$:

$\tilde{f}(x) = \frac{a_0}{2} + \sum_{n=1}^{\infty}(a_n\cos{nx} + b_n\sin{nx})$,

$a_n = \langle f,\frac{1}{\pi}\cos{nx} \rangle =
\frac{1}{\pi}\int_{-\pi}^{\pi}f(x)\cos{nx}\dx, \quad n \in \N_0$,

$b_n = \langle f,\frac{1}{\pi}\sin{nx} \rangle =
\frac{1}{\pi}\int_{-\pi}^{\pi}f(x)\sin{nx}\dx, \quad n \in \N$.

\textbf{Sinusna in kosinusna vrsta:} $f \in L^2([0,\pi])$. Za tako funkcijo
obstaja liha in soda razširitev na $[-\pi,\pi]$. Za $\tilde{f}^S$ so $b_n = 0$,
za $\tilde{f}^L$ pa $a_n = 0$.

\textsc{Posledica:} Na $[0,\pi]$ za $f$ obstajata dva razvoja: \textbf{sinusna
vrsta}: $\tilde{f}(x) =  \sum_{n=1}^{\infty} \tilde{b}_n\sin{nx}$ in
\textbf{kosinusna vrsta}:  $\tilde{f}(x) =
\frac{\tilde{a}_0}{2}\sum_{n=1}^{\infty}\tilde{a}_n\cos{nx}$, kjer sta:

$\tilde{a}_n = \frac{2}{\pi}\int_0^{\pi}f(x)\cos{nx}\dx, \quad n \in \N$,

$\tilde{b}_n = \frac{2}{\pi}\int_0^{\pi}f(x)\sin{nx}\dx, \quad n \in \N_0$.

S substitucijo lahko razvoje prevedemo na poljuben interval $[-L,L]$ oz.
$[0,L]$, $L > 0$. V tem primeru je $\{\frac{1}{2L}, \frac{1}{L}\sin{\frac{n\pi
x}{L}},\frac{1}{L}\cos{\frac{n\pi x}{L}},\ldots \}$ KONS.


\textbf{Metoda separacije:} Kdaj jo uporabimo: \underline{Trivialen pogoj:}
Imamo eno spremenljivko na omejenem območju in s homogenimi robnimi pogoji, npr.

$x \in [0,L]$: $\alpha u(0,t) + \beta u_x(0,t) = 0, \quad \gamma u(_,t)+\delta
u_x(L,t) = 0, \quad \alpha, \beta, \gamma, \delta \in \R.$

\underline{Netrivialen pogoj:} Diferencialni operator, ki določa PDE, zadošča
Sturm-Liouvillovi teoriji, tj. množica lastnih funkcij, ki jih dobimo iz robnega
problema tvori K.O.N.S.


\underline{\textbf{Štirje koraki metode:}}

$\#1$: Separacija: nastavek $u(x,t)=X(x)T(t)$. (Nastavek vstavi v enačbo in loči
spremenljivke, dobljeno enačbo pa enači z $\mu \in \R$.)

$\#2$: Določanje lastnih funkcij $\{X_n\}_{n\in \N}$ iz robnega problema za NDE.
(Reši NDE za X, homogeni robni pogoji ti dajo začetne pogoje za NDE. Obravnavati
moraš možnosti $\mu >0, \mu = 0, \mu < 0$. Če je v kakšnem primeru $X \equiv 0$,
lastnih funkcij v tem primeru ni. Pri izbire množice lastnih funkcij, lahko
splošno konstanto za vsak člen BŠS postaviš na 1.)

$\#3$: Iskanje pripadajočih $\{T_n \}_{n \in \N}$. (Z $\mu$, ki ga dobiš v $\#2$
in določa družino lastnih funkcij, reši še NDE za $T$. Splošno konstanto lahko
tu pustiš, lahko si misliš, za je v njej spravljena konstanta iz množice lastnih
funkcij za $X$.)

$\#4$: Splošna rešitev $u = \sum_{n=1}^\infty X_nT_n$. (Rešitev naj bi bila
odvisna od števno mnogo konstant, ki jih določiš iz nehomogenega robnega pogoja.
Dobro je opaziti morebitne sinusne/kosinusne vrste, ki jih dobiš z robnim
pogojem, in upoštevati zvezo s koeficienti iz razvoja v sinusno/kosinusnov
vrsto, torej $C_n = a_n \text{ ali } b_n$.)

Če za nobeno od spremenljivk nimamo homogenega robnega pogoja, razbijemo problem
na dva dela, npr: $\triangle u = 0$ razbijemo na $u = v+w$, $\triangle v = 0$ in
$\triangle w = 0$, pri čemer $v$-ju in $w$-ju damo vsakemu en homogen robni
pogoj in en pogoj, ki je od $u$-ja.

\textbf{Reševanje nehomogene enačbe s separacijo:}

Naredimo $\#$1 in $\#$2 za homogen problem (pri drugem koraku preveri, da lastne funkcije tvorijo K.O.S., tj. $\langle X_n,X_m \rangle = c_n \delta_{n,m} = \begin{cases} c_n; & n = m \\ 0; & n \neq m \end{cases}$, korak $\#$3 pa naredimo tako, da rešitev iz $\#2$ vstavimo v $u(x,t) = \sum_{n=0}^\infty X_n(x)T_n(t)$. Tu $T_n$ ne poznamo in računamo za splošnega. Vstavimo v nehomogeno enačbo in primerjamo koeficiente s tistimi iz razvoja nehomogenega dela po $\{X_n\}$. Partikularno rešitev dobimo z nastavkom. Ko razvijamo nehomogeni del $f(x)$ po $\{X_n\}$, si napišemo $\tilde{f}(x) = \sum_{n=0}^\infty \frac{\langle f,X_n \rangle}{\langle X_n,X_n \rangle}X_n$ in izračunamo koeficiente iz razvoja.

\textbf{Laplace v polarnih koordinatah:} $\triangle u = u_{rr} + \frac{1}{r}u_r + \frac{1}{r^2}u_{\varphi \varphi}$

Pri polarnih koordinatah imamo namesto homogenega robnega pogoja lahko tudi naravni pogoj: $2\pi$-periodičnost: $u(r,0) = u(r,2\pi), u_\varphi (r,0) = u_\varphi (r,2\pi)$.

Sistem $M\vec{x} = 0 $ ima netrivialne rešitve $\Longleftrightarrow \det{M} = 0$.

%%%%%%%%%%%%%%%%%%eksistenca
\textbf{Eksistenca:}

$u_{tt}-c^2u_{xx}=0, c \in \R_+$ ima pri pogojih $u_x(0,t) = u_x(L,t)=0$ in $u(x,0) = u_t(x,0) = 0$ edino rešitev $u \equiv 0$. Za poljubne $a,b,f,g \in \mathcal{C}^\infty (\R)$ ima $u_{tt}-c^2u_{xx}=0, c \in \R_+$ enolično rešitev tudi pri pogojih $u_x(0,t) = a(t),u _x(L,t)=b(t), u(x,0) = f(x)$ in $u_t(x,0) = g(x)$

\vspace{-.4cm}  %Iz nekega razloga tukaj naredi velik prazen prostor (pojavilo se je, ko je nastal spodnje okolje enumerate, razmaka ni, če je vspace < -0.4

\section*{Sturm-Liouvilleova teorija}

$A \in \R^{n \times n}$ je sebi adjungiran, če $A^T = A$, lastni vektorji tvorijo ortogonalno bazo. Velja $\langle Av,w \rangle = (Av)^Tw = v^TA^Tw = v^T(Aw) = \langle v,Aw \rangle$.

\textbf{$SL$-operator}: $L:\mathcal{C}^2([a,b])\longrightarrow \mathcal{C}([a,b])$, $L(y) = \frac{1}{r(x)}[(p(x)y')'+q(x)y]$, $p,r > 0, x \in [a,b]$ + mešani ali periodični robni pogoj. Gledamo skrčitev operatorja na $V = \mathcal{C}^2(]a,b]) \cap \{\text{robni pogoji}\}$, $\langle f,g\rangle = \int_a^bf(x)g(x)r(x)dx$, kjer je $r$ utež.

$L$ je sebi adjungiran za robne pogoje:

\begin{enumerate}
\item $y(a)=y(b)=0$,
\item $y'(a)=y'(b)=0$,
\item $y(a)=y(b), p(a)y'(a)=p(b)y'(b)$.
\end{enumerate}

\textbf{Izrek} (o kompletnosti lastnih funkcij)

$p \in \mathcal{C}^1([a,b]); r,q \in \mathcal{C}([a,b]); p,r > 0$. Potem ima lastni problem $L(y) = \mu y$ pri robnih pogojih
\vspace{-0.2cm}
\begin{enumerate}[a)]
\item $\alpha y(a) + \beta y'(a) = 0$ in $\gamma y(b) +  \delta y'(b) = 0$; $\alpha^2 + \beta^2 \neq 0, \gamma^2 + \delta^2 \neq 0$
\item $y(a) = y(b)$ in $\alpha y'(a) = \beta y'(b)$; $\alpha^2+\beta^2 \neq 0$
\end{enumerate}
števno mnogo rešitev z lastnostmi:
\begin{enumerate}[i)]
\item $\mu_1 < \mu_2 < \ldots, $ $\lim_{n\rightarrow \infty}\mu_n = \infty$
\item $\{g_n\}_{n\in \N}$ tvorijo \underline{kompleten} ortonormiran sistem v $L^2([a,b])\cap \{\text{robni pogoji}\}$ in za $ \langle f,g \rangle = \int_a^b f(x)g(x)r(x)dx$.
\end{enumerate}

%%%%%%%%%%%%%%%%%%%%%%%%%%%%%%%%%%%%%%%%%%%%%%%%%%%%
\textbf{Uporabni integrali:}
$\int_a^b \sin(\frac{n\pi x}{b-a})^2 \dx = \int_a^b \cos(\frac{n\pi x}{b-a})^2 \dx =
\frac{(b-a) \left(\sin \left(\frac{2 \pi  a n}{b-a}\right)-\sin \left(\frac{2
\pi  b n}{b-a}\right)+2 \pi  n\right)}{4 \pi  n}, n \in \C$

$(\int_a^bx^i \sin(kx)\dx)_{1,2} = (\frac{-\sin (a k)+a k \cos (a k)+\sin (b k)-b k \cos (b k)}{k^2},\frac{(a^2 k^2-2) \cos (a k)-2 a k \sin (a k)+(2-b^2 k^2) \cos (b k)+2 b k \sin (b k)}{k^3})$

$(\int_a^bx^i \cos(kx)\dx)_{1,2} = (\frac{-a k \sin (a k)-\cos (a k)+b k \sin (b k)+\cos (b k)}{k^2},\frac{(2-a^2 k^2) \sin (a k)-2 a k \cos (a k)+(b^2 k^2-2) \sin (b k)+2 b k \cos (b k)}{k^3})$

$\int_a^b (x-a)(b-x)\sin(kx)\dx = \frac{k (a-b) (\sin (a k)+\sin (b k))+2 \cos (a k)-2 \cos (b k)}{k^3}$

$\int_a^b (x-a)(b-x)\cos(kx)\dx = \frac{k (a-b) (\cos (a k)+\cos (b k))-2 \sin (a k)+2 \sin (b k)}{k^3}$

$\int_a^b \sin(mx)\cos(kx)\dx = \frac{-k \sin (a k) \sin (a m)-m \cos (a k) \cos (a m)+k \sin (b k) \sin (b m)+m \cos (b k) \cos (b m)}{k^2-m^2}$

$\int_a^b \sin(mx)\sin(kx)\dx = \frac{-m \sin (a k) \cos (a m)+k \cos (a k) \sin (a m)+m \sin (b k) \cos (b m)-k \cos (b k) \sin (b m)}{k^2-m^2}$

$\int_a^b \cos(mx)\cos(kx)\dx = \frac{-k \sin (a k) \cos (a m)+m \cos (a k) \sin (a m)+k \sin (b k) \cos (b m)-m \cos (b k) \sin (b m)}{k^2-m^2}$, povsod so $m, n, k \in \C$


\textbf{Polarne koordinate:}
$u_x = \cos (\varphi ) u_r-\frac{\sin (\varphi)}{r}u_\varphi,
u_y = \sin (\varphi)u_r+\frac{\cos (\varphi )}{r}u_\varphi, \\
u_{xx} =
\cos^2(\varphi) u_{rr}
- \frac{\sin(2\varphi)}{r}u_{r\varphi}
+ \frac{\sin^2(\varphi)}{r^2}u_{\varphi\varphi}
+ \frac{\sin^2(\varphi)}{r}u_r
+ \frac{\sin(2\varphi)}{r^2}u_\varphi, \\
u_{xy} =
\frac{1}{2} \sin (2 \varphi) u_{rr}
+ \frac{\cos (2 \varphi )}{r}u_{r\varphi}
-\frac{\sin (2 \varphi)}{2 r^2} u_{\varphi\varphi}
- \frac{\sin (2 \varphi)}{2 r}u_r
- \frac{\cos (2 \varphi)}{r^2}u_\varphi, \\
u_{yy} =
\sin ^2(\varphi ) u_{rr}
+\frac{\sin (2\varphi )}{r} u_{r\varphi}
+ \frac{\cos ^2(\varphi )}{r^2} u_{\varphi\varphi}
+\frac{\cos ^2(\varphi )}{r} u_r
-\frac{\sin(2\varphi)}{r^2} u_{\varphi},
\Delta u = u_{rr} + \frac1r u_r + \frac{1}{r^2}u_{\varphi\varphi}
$



%%%%%%%%%%%%%%%%%%%%%%%%%%%%%%%%%%%%%%%%%%%%%%%%%%%%%


\section*{NDE višjih redov}
Ne nastopa $y$: uvedemo $z = y'$. \\ Obe
strani sta odvoda nečesa: integriramo in dodamo konstanto. \\ \hspace*{20pt}
Odvodi: $y'/y = (\log(y))', x y' + y = (xy)', \frac{y'' y - y'^2}{y^2} =
(\frac{y'}{y})', \frac{y' x - y}{x^2} = (\frac{y}{x})'$.\\ Ne nastopa $x$:
uvedemo $z(y) = y'$, $y$ neodvisna spr. $y'' = \dot{z}z$, $y''' = \ddot{z}z^2 +
\dot{z}^2z$. \\ Homogena: $F(x, ty, ty', \dots, ty^{(n)})$ = $t^k F(x, y, y',
\dots,  y^{(n)})$. Vpeljemo $z(x) = y'/y$. $y''/y = z' + z^2$.\\ Z utežjo:
$F(kx, k^my, k^{m-1}y', \dots, k^{m-n}y^{(n)}) = k^pF(x, y, y', \dots,
y^{(n)})$. Uvedemo: $x = e^t, y = u(t)e^{mt}$.
%%
%%\subsubsection*{Geometrija}
%%Tangenta v točki $(x,y)$: $Y-y=y'(X-x)$ \\
%%Normala v točki $(x,y)$: $Y-y=-\frac{1}{y'}(X-x)$\\
%%Ločna dolžina: $\int_a^b \sqrt{1 + y'(x)^2} \dx$ \\
%%$d(T_0,p)=\frac{|ax_0+by_0+c|}{\sqrt{a^2+b^2}}$ $ \quad T_0=(x_0,y_0), ax+by+c=0$
%
%%\begin{tabular}{ll}
%%Abscisa tangente: $X = x - y/y'$ & Ordinata tangente: $Y = y - xy'$ \\
%%Abscisa normale: $X = x+yy'$ & Ordinata normale: $Y = y + x/y'$
%%\end{tabular}
%
%
\section*{Integrali in formule} \begin{tabular}{llll} $\int \ln{x} \dx
= x \ln{x} - x + C$ & $\int \frac{1}{\sin(x)} \dx = \ln{\tan(x/2)} + C$ \\

$ \int x^m\log(x) \dx = x^{m+1}\left(\frac{\log x}{m+1} -
\frac{1}{(m+1)^2}\right) + C$ & $ \int \frac{1}{\cos(x)} \dx = -\log(\cot(x/2))
+ C$ \\

$ \int p(x) e^{k x} \dx = q(x) e^{k x} + C$, st($q$) = st($p$) & $ \int
\frac{1}{\tan(x)} \dx = \log(\sin(x)) + C$ \\

$ \int e^{a x} \sin(b x) \dx = \frac{e^{a x} }{ a^2 + b^2} (a \sin(b x) - b
\cos(b x)) + C$ & $ \int \tan(x) \dx = - \log(\cos(x)) + C$ \\

$ \int e^{a x} \cos(b x) \dx = \frac{e^{a x} }{ a^2 + b^2} (a \cos(b x) + b
\sin(b x)) + C$ & $ \int x/(1 + x) \dx = x - \log(x + 1) + C$ \\

$ \int \frac{1}{\sqrt{a^2 + x^2}} \dx =\text{arsh}\frac{x}{a} + C = \log|x +
\sqrt{x^2 + a^2}| + C$  & $ \int x/(1 + x) \dx = x - \log(x + 1) + C$ \\

$\int \frac{1}{\sqrt{a^2 - x^2}} \dx =\arcsin\frac{x}{a} + C$ & $ \int
\sin^2(x) \dx = \frac{1}{2} (x - \sin x \cos x) + C$ \\

$\int \frac{1}{a^2+x^2} \dx = \frac{1}{a}\arctan\frac{x}{a} + C$ & $ \int
\cos^2(x) \dx = \frac{1}{2} (x + \sin x \cos x) + C$  \\

$\sin^2(x/2) = (1 - \cos(x))/2$ & $\cos^2(x/2) = (1 + \cos(x))/2$ \\

\end{tabular}

\vspace{1ex}

\parbox{0.4\textwidth}{
\textbf{Faktorizacija:} \\
$\sin{x} + \sin{y} = 2\sin\frac{x+y}{2}\cos\frac{x-y}{2}$\\
$\sin{x} - \sin{y} = 2\cos\frac{x+y}{2}\sin\frac{x-y}{2}$\\
$\cos{x} + \cos{y} = 2\cos\frac{x+y}{2}\cos\frac{x-y}{2}$\\
$\cos{x} - \cos{y} =-2\sin\frac{x+y}{2}\sin\frac{x-y}{2}$
}
\parbox{0.4\textwidth}{
\textbf{Antifaktorizacija:} \\
$\sin\alpha\cos\beta = \frac{1}{2}\left[\sin(\alpha+\beta)+\sin(\alpha-\beta) \right]  $ \\
$\cos\alpha\sin\beta = \frac{1}{2}\left[\sin(\alpha+\beta)-\sin(\alpha-\beta) \right]  $ \\
$\cos\alpha\cos\beta = \frac{1}{2}\left[\cos(\alpha+\beta)+\cos(\alpha-\beta) \right]  $ \\
$\sin\alpha\sin\beta = -\frac{1}{2}\left[\cos(\alpha+\beta)-\cos(\alpha-\beta)  \right]$ \\
}
%
%
%
%$\displaystyle \int \frac{1}{a x^2 + bx + c} \dx = \begin{cases}
%\frac{1}{\sqrt{a}}\log|2ax + b + 2 \sqrt{a} \sqrt{ax^2 + bx + c}|+C, & a >0\\
%\frac{-1}{\sqrt{-a}} \arcsin((2ax + b)/\sqrt{D})+C, & a<0 \end{cases} $\\ $\int
%\frac{p(x)}{(x-a)^n (x^2 + bx + c)^m} \, \dx = A \log|x - a| + B \log|x^2 + bx
%+ c| + C \arctan(\frac{2x + b}{\sqrt{-D}}) + \frac{\text{polinom st. ena manj
%kot spodaj}}{(x-a)^{n-1} (x^2 + bx + c)^{m-1}}$ \\
%
%Substitucija: $t = \tan x, \sin^2 x = t^2 /(1 + t^2), \cos^2 x = 1/(1 + t^2),
%\dx = \dt/(1 + t^2)$\\ Substitucija: $u = \tan (x/2), \sin x = 2 u /(1 + u^2),
%\cos x = (1-u^2)/(1 + u^2), \dx = 2 du/(1 + u^2)$\\
%
%
%

\vfill
\hfill avtor: Klemen Sajovec, malo sprememb: Jure Slak

\end{document}

