\documentclass[11pt,a4paper]{amsart}
\usepackage[slovene]{babel}
\usepackage[T1]{fontenc}
\usepackage[utf8]{inputenc}
\usepackage{amsmath,amssymb,amsfonts}
\usepackage{url}
\usepackage{enumerate}
\usepackage{titlesec}
\titleformat*{\section}{\sc\bfseries\centering}

\usepackage[
top    = 0.9cm,
bottom = 0.9cm,
left   = 1cm,
right  = 1cm]{geometry}

% ukazi za matematicna okolja
\theoremstyle{definition} % tekst napisan pokoncno
\newtheorem{definicija}{Definicija}[section]
\newtheorem{primer}[definicija]{Primer}
\newtheorem{opomba}[definicija]{Opomba}
\newtheorem{zgled}[definicija]{Zgled}

\theoremstyle{plain} % tekst napisan posevno
\newtheorem{lema}[definicija]{Lema}
\newtheorem{izrek}[definicija]{Izrek}
\newtheorem{trditev}[definicija]{Trditev}
\newtheorem{posledica}[definicija]{Posledica}


\newenvironment{itemize*}%
{
\vspace{-6pt}
\begin{itemize}
\setlength{\itemsep}{0pt}
\setlength{\parskip}{1pt}
}
{\end{itemize}}

\newenvironment{enumerate*}%
{
\vspace{-6pt}
\begin{enumerate}
\setlength{\itemsep}{0pt}
\setlength{\parskip}{1pt}
}
{\end{enumerate}}

\newcommand{\ds}{\ensuremath{\,\mathrm{d}s}}
\newcommand{\dx}{\ensuremath{\,\mathrm{d}x}}
\newcommand{\dxi}{\ensuremath{\,\mathrm{d}\xi}}
\newcommand{\dy}{\ensuremath{\,\mathrm{d}y}}
\newcommand{\dt}{\ensuremath{\,\mathrm{d}t}}
\newcommand{\dd}[1]{\ensuremath{\,\mathrm{d}#1}}
\let\oldint\int
\renewcommand{\int}{\oldint \!}


\newcommand{\R}{\mathbb R}
\newcommand{\N}{\mathbb N}
\newcommand{\Z}{\mathbb Z}
\newcommand{\C}{\mathbb C}
\newcommand{\Q}{\mathbb Q}
\newcommand{\F}{\mathcal{F}}
\newcommand{\Cont}{\mathcal{C}}

\pagestyle{empty}

\begin{document}
\thispagestyle{empty}
\setlength{\parindent}{0pt}

% \fontsize{9pt}{9pt}


%%%%%%%%%%%%%%%%%%%%%%%%%%%%%%%%%%%%%%%%%%%%%%%%%%%%%

\section*{Separacija spremenljivk}

$L^2([-\pi,\pi]) = \{f:[-\pi,\pi]\longrightarrow \R, \int_{-\pi}^\pi |f|^2 \dx < \infty \}$ je vektorski prostor s skalarnim produktom $\langle f, g\rangle =
\int_{-\pi}^\pi f(x)g(x)\dx$. Množica funkcij
  $\{\frac{1}{2\pi},
  \frac{1}{\pi}\sin{x},\frac{1}{\pi}\cos{x},\frac{1}{\pi}\sin{2x},\frac{1}{\pi}\cos{2x},\ldots
\}$. je \underline{kompleten} (vsako funkcijo se da na enoličen način razviti v
tem sistemu),  \underline{ortonormiran} sistem za ta produkt.

\textbf{Fourierjev razvoj:} $f \in L^2([-\pi,\pi])$:

$\tilde{f}(x) = \frac{a_0}{2} + \sum_{n=1}^{\infty}(a_n\cos{nx} + b_n\sin{nx})$,

$a_n = \langle f,\frac{1}{\pi}\cos{nx} \rangle =
\frac{1}{\pi}\int_{-\pi}^{\pi}f(x)\cos{nx}\dx, \quad n \in \N_0$,

$b_n = \langle f,\frac{1}{\pi}\sin{nx} \rangle =
\frac{1}{\pi}\int_{-\pi}^{\pi}f(x)\sin{nx}\dx, \quad n \in \N$.

\textbf{Sinusna in kosinusna vrsta:} $f \in L^2([0,\pi])$. Za tako funkcijo
obstaja liha in soda razširitev na $[-\pi,\pi]$. Za $\tilde{f}^S$ so $b_n = 0$,
za $\tilde{f}^L$ pa $a_n = 0$.

\textsc{Posledica:} Na $[0,\pi]$ za $f$ obstajata dva razvoja: \textbf{sinusna
vrsta}: $\tilde{f}(x) =  \sum_{n=1}^{\infty} \tilde{b}_n\sin{nx}$ in
\textbf{kosinusna vrsta}:  $\tilde{f}(x) =
\frac{\tilde{a}_0}{2}\sum_{n=1}^{\infty}\tilde{a}_n\cos{nx}$, kjer sta:

$\tilde{a}_n = \frac{2}{\pi}\int_0^{\pi}f(x)\cos{nx}\dx, \quad n \in \N$,

$\tilde{b}_n = \frac{2}{\pi}\int_0^{\pi}f(x)\sin{nx}\dx, \quad n \in \N_0$.

S substitucijo lahko razvoje prevedemo na poljuben interval $[-L,L]$ oz.
$[0,L]$, $L > 0$. V tem primeru je $\{\frac{1}{2L}, \frac{1}{L}\sin{\frac{n\pi
x}{L}},\frac{1}{L}\cos{\frac{n\pi x}{L}},\ldots \}$ KONS.


\textbf{Metoda separacije:} Kdaj jo uporabimo: \underline{Trivialen pogoj:}
Imamo eno spremenljivko na omejenem območju in s homogenimi robnimi pogoji, npr.

$x \in [0,L]$: $\alpha u(0,t) + \beta u_x(0,t) = 0, \quad \gamma u(_,t)+\delta
u_x(L,t) = 0, \quad \alpha, \beta, \gamma, \delta \in \R.$

\underline{Netrivialen pogoj:} Diferencialni operator, ki določa PDE, zadošča
Sturm-Liouvillovi teoriji, tj. množica lastnih funkcij, ki jih dobimo iz robnega
problema tvori K.O.N.S.

Separacija v splošnem generira šibke rešitve, lahko so težave s konvergenco dobljene vrste!

\underline{\textbf{Štirje koraki metode:}}

$\#1$: Separacija: nastavek $u(x,t)=X(x)T(t)$. (Nastavek vstavi v enačbo in loči
spremenljivke, dobljeno enačbo pa enači z $\mu \in \R$.)

$\#2$: Določanje lastnih funkcij $\{X_n\}_{n\in \N}$ iz robnega problema za NDE.
(Reši NDE za X, homogeni robni pogoji ti dajo začetne pogoje za NDE. Obravnavati
moraš možnosti $\mu >0, \mu = 0, \mu < 0$. Če je v kakšnem primeru $X \equiv 0$,
lastnih funkcij v tem primeru ni. Pri izbire množice lastnih funkcij, lahko
splošno konstanto za vsak člen BŠS postaviš na 1.)

$\#3$: Iskanje pripadajočih $\{T_n \}_{n \in \N}$. (Z $\mu$, ki ga dobiš v $\#2$
in določa družino lastnih funkcij, reši še NDE za $T$. Splošno konstanto lahko
tu pustiš, lahko si misliš, za je v njej spravljena konstanta iz množice lastnih
funkcij za $X$.)

$\#4$: Splošna rešitev $u = \sum_{n=1}^\infty X_nT_n$. (Rešitev naj bi bila
odvisna od števno mnogo konstant, ki jih določiš iz nehomogenega robnega pogoja.
Dobro je opaziti morebitne sinusne/kosinusne vrste, ki jih dobiš z robnim
pogojem, in upoštevati zvezo s koeficienti iz razvoja v sinusno/kosinusnov
vrsto, torej $C_n = a_n \text{ ali } b_n$.)

Če za nobeno od spremenljivk nimamo homogenega robnega pogoja, razbijemo problem
na dva dela, npr: $\triangle u = 0$ razbijemo na $u = v+w$, $\triangle v = 0$ in
$\triangle w = 0$, pri čemer $v$-ju in $w$-ju damo vsakemu en homogen robni
pogoj in en pogoj, ki je od $u$-ja.

\textbf{Reševanje nehomogene enačbe s separacijo:}

Naredimo $\#$1 in $\#$2 za homogen problem (pri drugem koraku preveri, da lastne funkcije tvorijo K.O.S., tj. $\langle X_n,X_m \rangle = c_n \delta_{n,m} = \begin{cases} c_n; & n = m \\ 0; & n \neq m \end{cases}$, korak $\#$3 pa naredimo tako, da rešitev iz $\#2$ vstavimo v $u(x,t) = \sum_{n=0}^\infty X_n(x)T_n(t)$. Tu $T_n$ ne poznamo in računamo za splošnega. Vstavimo v nehomogeno enačbo in primerjamo koeficiente s tistimi iz razvoja nehomogenega dela po $\{X_n\}$. Partikularno rešitev dobimo z nastavkom. Ko razvijamo nehomogeni del $f(x)$ po $\{X_n\}$, si napišemo $\tilde{f}(x) = \sum_{n=0}^\infty \frac{\langle f,X_n \rangle}{\langle X_n,X_n \rangle}X_n$ in izračunamo koeficiente iz razvoja.

\textbf{Laplace v polarnih koordinatah:} $\triangle u = u_{rr} + \frac{1}{r}u_r + \frac{1}{r^2}u_{\varphi \varphi}$

Pri polarnih koordinatah imamo namesto homogenega robnega pogoja lahko tudi naravni pogoj: $2\pi$-periodičnost: $u(r,0) = u(r,2\pi), u_\varphi (r,0) = u_\varphi (r,2\pi)$.

Sistem $M\vec{x} = 0 $ ima netrivialne rešitve $\Longleftrightarrow \det{M} = 0$.

Rešitve enačbe $\triangle u = 0$ na enotskem disku so: $u(r, \varphi) = C_0 + D_0 \log r + \sum_{n = 1}^\infty (C_n r^n + D_n r^{-n}) (A_n \cos(n \varphi) + B_n \sin(n \varphi))$.

%%%%%%%%%%%%%%%%%%eksistenca
\textbf{Eksistenca:}

$u_{tt}-c^2u_{xx}=0, c \in \R_+$ ima pri pogojih $u_x(0,t) = u_x(L,t)=0$ in $u(x,0) = u_t(x,0) = 0$ edino rešitev $u \equiv 0$. Za poljubne $a,b,f,g \in \mathcal{C}^\infty (\R)$ ima $u_{tt}-c^2u_{xx}=0, c \in \R_+$ enolično rešitev tudi pri pogojih $u_x(0,t) = a(t),u _x(L,t)=b(t), u(x,0) = f(x)$ in $u_t(x,0) = g(x)$

To lahko dokažemo tako, da preverimo, da je nek energijski funkciona $E$l konstantno 0 (trik: $E' = 0$ in potem izračunamo v neki točki).

\vspace{-.4cm}  %Iz nekega razloga tukaj naredi velik prazen prostor (pojavilo se je, ko je nastal spodnje okolje enumerate, razmaka ni, če je vspace < -0.4

\section*{Sturm-Liouvilleova teorija}

$A \in \R^{n \times n}$ je sebi adjungiran, če $A^T = A$, lastni vektorji tvorijo ortogonalno bazo, lastne vrednosti so realne. Velja $\langle Av,w \rangle = (Av)^Tw = v^TA^Tw = v^T(Aw) = \langle v,Aw \rangle$. V splošnem je pogoj $\langle Av,w \rangle = \langle v,Aw \rangle$.

\textbf{$SL$-operator}: $L:\mathcal{C}^2([a,b])\longrightarrow \mathcal{C}([a,b])$, $L(y) = \frac{1}{r(x)}[(p(x)y')'+q(x)y]$, $p,r > 0, x \in [a,b]$ + mešani ali periodični robni pogoj. Gledamo skrčitev operatorja na $V = \mathcal{C}^2(]a,b]) \cap \{\text{robni pogoji}\}$, $\langle f,g\rangle = \int_a^bf(x)g(x)r(x)dx$, kjer je $r$ utež.

$L$ je sebi adjungiran za robne pogoje:

\begin{enumerate}
\item $y(a)=y(b)=0$,
\item $y'(a)=y'(b)=0$,
\item $y(a)=y(b), p(a)y'(a)=p(b)y'(b)$.
\end{enumerate}

\textbf{Izrek} (o kompletnosti lastnih funkcij)

$p \in \mathcal{C}^1([a,b]); r,q \in \mathcal{C}([a,b]); p,r > 0$. Potem ima lastni problem $L(y) = \mu y$ pri robnih pogojih
\vspace{-0.1cm}
\begin{enumerate}[a)]
\item $\alpha y(a) + \beta y'(a) = 0$ in $\gamma y(b) +  \delta y'(b) = 0$; $\alpha^2 + \beta^2 \neq 0, \gamma^2 + \delta^2 \neq 0$
\item $y(a) = y(b)$ in $\alpha y'(a) = \beta y'(b)$; $\alpha^2+\beta^2 \neq 0$
\end{enumerate}
števno mnogo rešitev z lastnostmi:
\begin{enumerate}[i)]
\item $\mu_1 > \mu_2 > \ldots, $ $\lim_{n\rightarrow \infty}\mu_n = -\infty$
\item $\{y_n\}_{n\in \N}$ tvorijo \underline{kompleten} ortonormiran sistem v $L^2([a,b])\cap \{\text{robni pogoji}\}$ in za $ \langle f,g \rangle = \int_a^b f(x)g(x)r(x)dx$.
\end{enumerate}

Enačbe oblike $u_t = au_{xx} + bu_x + cu$, kjer $a,b,c \in \R$, $a\neq 0$, $u(0,t) = u(L,t) = 0$ lahko rešujemo s separacijo spremenljivk za poljubne koeficiente $a,b,c$.

Trik: če ne moremo doseči $p, r > 0$, lahko poskusimo prevesti na $\tilde{\mu} = -\mu, \tilde{L}(X) = -L(X)$ in rešujemo $\tilde{L}(X) = \tilde{\mu}X$, ki morda ustreza pogoju $p, r > 0$.

$y(x) = \tilde{A}x^{ia} + \tilde{B}x^{-ia} = A\cos{(a\ln{x})} + B\sin{(a\ln{x})}$

Dejstvo, ali določena družina funkcij tvori K.O.S., preverjamo z identifikacijo istoležnih funkcij v $L(y) = \frac{1}{r(x)}[(p(x)y')'+q(x)y] = \mu y =$ naša enačba (npr.  $L(y) = \frac{1}{r(x)}[(p(x)y')'+q(x)y] = x^2y'' + xy' = \mu y$ za reševanje enačbe $ x^2y'' + xy' = \mu y$.) Poiščemo utež $r$ in za prostor vzamemo prostor funkcij, za katere rešujemo enačbo, presekan z robni pogoji.

\textbf{Legendrova enačba}

$L(y) = ((1-x^2)y')'=\mu y$, $x \in [-1,1]$. To je singularen diferencialni operator, saj $p(\pm 1) = 0$, izrek pa deluje za $p > 0$. Lastna funkcija $y$ je omejena v $x = \pm 1$ natanko tedaj, ko je $\mu = -n(n+1), n \in \N$. Tedaj obstajata neodvisni polinomski rešitvi stopenj $2m$ in $2m+1$.

\textbf{Kvocientni kriterij} za vrsto $\sum C_n x^n$: $\lim_{n\rightarrow \infty}\left| \frac{C_{n+1} x^{n+1}}{C_n x^{n}}\right| < 1$, potem vrsta konvergira za izbrani $x$.

\textbf{Raabejev kriterij} za vrsto $\sum C_n x^n$: $\lim_{n\rightarrow \infty}n(1-\frac{C_nx^n}{C_{n+1}x^{n+1}}) < 1$, potem ta vrsta divergira za izbrani $x$.

Če gledamo operator $L(y) = ((1-x^2)y')'=\mu y$ na prostoru $\mathcal{C}^2(-1,1) \cap \{\text{omejene funkcije v }\pm1 \}$, dobimo lastne pare $(-n(n+1),P_n)$ in $\{P_n \}_{n \in \N_0}$ je K.O.S.

\textbf{Laplace v sferičnih koordinatah:} $\triangle u = \frac{1}{r^2} \frac{\partial}{\partial r}(r^2u_r) + \frac{1}{r^2}\left[ \frac{1}{\cos \vartheta} \frac{\partial}{\partial \vartheta}(\cos \vartheta u_\vartheta) + \frac{1}{\cos \vartheta} u_{\varphi \varphi} \right]$, $r \in [0,\infty ),\varphi \in [0,2\pi], \vartheta \in [-\frac{\pi}{2}, \frac{\pi}{2}]$.

\textbf{Besslova enačba}

$x^2y''+xy'+(x^2-n^2)y = 0$, $x > 0$, $n \in \N_0$, singularna za $x = 0$. Z nastavkom $y = \sum_{m = 0}^\infty C_m x^{m+k}$, $k \in \N_0,$ $C_0 \neq 0$ dobimo rešitev, ki je omejena v $x = 0$: $J_n(x) = C_0 \sum_{l = 0}^\infty \frac{(-1)^l}{2^{2l}l!(n+l)(n+l-1)\cdots (n+1)}x^{2l + n}$.

\textbf{Dodatek k Besslovi enačbi:}
\begin{enumerate}
\item Enačbo lahko obravnavamo tudi za $n \in \R_+\backslash \N_0$, vendar v eksplicitni obliki namesto $(n+l)!$ dobimo $\Gamma (n+l+1)$.
\item Enačbo lahko obravnavamo tudi za $n \in \R_-$, vendar dobimo Besslove funkcije drugega reda $Y_n$, ki so singularne v $x=0$.
\item Splošna rešitev Besslove enačbe: $y(x) = AJ_n(x) + BY_n(x)$.
\item Besslova funkcija ima števno mnogo ničel. Vse razen $J_0$ se začnejo v $(0,0)$.
\end{enumerate}

\section*{Fourierova transformacija in PDE}

$f \in L^1(\R ) = \{ f: \R \longrightarrow \R ; \int_\R |f| dx < \infty \}$.

$\F (f) (x) = \int_{-\infty}^\infty f(s) e^{isx} ds$

$\F^{-1} (f) (x) = \frac{1}{2\pi} \int_{-\infty}^\infty f(s) e^{-isx} ds$

Lastnosti:
\begin{enumerate}

\item $\F$ je linearna: $\F(\alpha f + \beta g) = \alpha \F(f) + \beta \F (g)$

\item $\F (f')(x) = (-ix) \F(f)(x)$

\item $\frac{d}{dx} \left[ \F (f)(x) \right] = \F(ixf) (x)$

\item Če je $f$ soda funkcija, velja $\F^{-1}(f) = \frac{1}{2\pi}\F(f)$ in obe transformaciji sta realni funkciji.

\end{enumerate}

Nekaj izračunanih transformacij:
\begin{itemize}

\item $f_1(x) = \begin{cases} 1;& |x| \leq 1 \\ 0; &\text{sicer} \end{cases}$, $\F(f_1) = \frac{2\sin x}{x}$

\item $\F (e^{-a|x|})(x) = \frac{2a}{a^2+x^2}$, $a>0$

\item $\F (e^{-ax^2})(x) = \sqrt{\frac{\pi}{a}}e^{-\frac{x^2}{4a}}$

\item $\F^{-1} (e^{-ax^2})(x) = \frac{1}{2\pi}\sqrt{\frac{\pi}{a}}e^{-\frac{x^2}{4a}}$

\item $f_2 = \begin{cases} 1-|x|;& |x|< 1 \\  0; & \text{sicer} \end{cases}$, $\F(f_2) = \frac{2}{x^2}(1-\cos x)$

\item $\F (\cos (ax^2) ) = \sqrt{\frac{\pi}{a}} \cos (\frac{x^2}{4a}-\frac{ \pi}{4} )$, $a > 0$

\item $\F (\cos (x) ) = \pi (\delta(x+1) + \delta(x-1))$

\end{itemize}

\textbf{Uporaba Fourierovih transformacij v PDE}

Želimo reševati PDE, v kateri je ena spremenljivka neomejena, npr. $x \in \R, t > 0$.
$U(x,t) =  \F(u)(x,t) = \int_{-\infty}^\infty u(s,t)e^{isx}ds$ (transformacija po $x$)

Veljajo pravila:
\begin{enumerate}

\item $\F(\alpha u + \beta v) = \alpha \F(u) + \beta \F (v)$, $\alpha , \beta \in \C$

\item $\F(\frac{\partial^n}{\partial t^n} u) = \frac{\partial^n}{\partial t^n}\F(u) = \frac{\partial^n}{\partial t^n} U$

\item $\F(\frac{\partial^n}{\partial x^n}u) = (-ix)^n \F(u) = (-ix)^n U$

\end{enumerate}

Strategija: PDE z odvodi po $t$ in $x$ s Fourierovo transformacijo pretvorimo v NDE z odvodi po $t$ (tudi začetne pogoje), nato pa dobljeno rešitev NDE (pazi, konstante so odvisne od $x$!) z inverzno Fourierovo transformacijo pretvorimo v rešitev PDE.

Enačba $u_{xx} = u_t + u$ ima enolično rešitev pri pogojih $u(x,0)=g(x), \forall g \in \Cont^\infty (\R)$. Če je $g$ soda (oz. liha), je rešitev soda (oz. liha). (Včasih za uporabo te lastnosti lahko naše začetne podatke sodo (oz. liho) razširiti, odvisno katera razširitev nam da nov pogoj. Z razširjenim začetnim podatkom nalogo rešimo, na koncu pa vzamemo samo ustrezno polovico. Če je pogoj $u_x(0, t)= 0$ naredimo sodo razširitev, če je $u(0, t)= 0$ pa liho. )

\textbf{Konvolucija}: $f \ast g (x) = \int_{-\infty}^\infty f(\xi) g(x-\xi) \dxi$

Velja: $\F (f \ast g) = \F(f) \F (g)$

Splošna rešitev enačbe $u_t - 2u_{xx} = 0$ pri pogoju $u(x,0) = f(x)$ je: $u(x,t) = \sqrt{ \frac{1}{8 \pi t}} \int_{-\infty}^\infty f(\xi) e^{- \frac{(x- \xi )^2}{8t}} \dxi$

Če iščemo splošno rešitev za poljuben začetni pogoj, se pri uporabi inverzne Fourierove transformacije splača uporabiti lastnost konvolucije in vriniti $\F \F^{-1}$.


\vspace{-0.4cm}
%%%%%%%%%%%%%%%%%%% DIRAC
\section*{Diracova $\delta$-funkcija}

Diracova $\delta$-funkcija zadošča dvema lastnostma:
\begin{enumerate}

\item $\delta (x) = 0$ za $\R \backslash \{0\}$,

\item $\int_\R \delta \dx = 1$.

\end{enumerate}

Diracovo $\delta$-funkcijo lahko realiziramo tudi kot limito funkcij $f_n (x) = \begin{cases} n;& |x| \leq \frac{1}{2n} \\ 0;& \text{sicer}\end{cases}$ ali kot limito funkcij $g_n(x) = \frac{n}{\sqrt{\pi}} e^{-n^2x^2}$. Definiramo jo lahko tudi kot $\delta (x) := \F^{-1}(1) = \frac{1}{2\pi} \F (1)$.

Za $f \in \Cont^\infty (\R)$ velja: $\int_\R \delta (x-a) f(x) \dx = f(a)$.

\section*{Poissonovo jedro in Greenova funkcija}

Rešujemo za: $u \in \Cont^2 ( \Omega) \cap \Cont (\bar{\Omega}), f \in \Cont ( \Omega ), g \in \Cont ( \partial \Omega )$ in $\Omega$ odprta, povezana podmnožica v $\R^2$.

$\triangle u = f$ je Poissonova enačba.

Robni pogoji:
\begin{enumerate}

\item $u|_{\partial \Omega} = g$ Dirichletov

\item $\partial_{\vec{n}} u |_{\partial \Omega} = g$ Neumannov.

\end{enumerate}

Poseben primer tega problema za $f=0$ so harmonične funkcije.

Vse harmonične funkcije na $\mathbb{H}$ oblike $u = f(\frac{x}{y})$ so $u = D \arctan (\frac{x}{y}) + E$.

\textbf{Izrek o povprečni vrednosti za harmonične funkcije}: $u(x_0,y_0) = \frac{1}{2\pi R} \int_{\partial K((x_0,y_0),R)}u \ds$.

\textbf{Šibki princip maksima}: Če je $\Omega$ omejeno: $\max_{\Omega} v = \max_{\partial \Omega} v$ oz.  $\min_{\Omega} v = \min_{\partial \Omega} v$

\textbf{Krepki princip maksima}: Če je $\Omega$ neomejeno in če harmonična funkcija doseže lokalni ekstrem v notranjosti $\Omega$, je funkcija konstantna.

\textbf{Liouvillov izrek}: Omejena cela funkcija je konstantna.

Zveza med harmoničnimi in holomorfnimi funkcijami:
\begin{enumerate}
\item Če $f = u + iv$ holomorfna, potem sta $u$ in $v$ harmonični.
\item Če je $u$ harmonična, potem obstaja harmonična funkcija $v$, da je $f = u + iv$ holomorfna (velja samo za enostavno povezana območja $\Omega$).
\end{enumerate}

Množica harmoničnih homogenih polinomov stopnje $n$ ($p(\lambda x,\lambda y) = \lambda^n p(x,y)$) tvori vektorski prostor dimenzije 2 za vsak $n \in \N$.

Edina rešitev enačbe $\triangle u = \lambda u$, $\lambda \geq 0$, $u|_{\partial \Omega} = 0$ je funkcija $u \equiv 0$.

Dirichletov problem je na omejenem območju enolično rešljiv.

Krivuljni integral vektorskega polja: $\int_{\partial \Omega}\vec{V} \mathrm{d}\vec{s} = \int_\alpha^\beta \vec{V}(\gamma (t))\cdot \dot{\gamma} (t)  \mathrm{d}t = \int_{\partial \Omega} \vec{V} \cdot \vec{T} \ds$, kjer je $\vec{T} = \frac{\dot{\gamma}(t)}{||\dot{\gamma}(t)||}$

$\gamma (t) = (x(t),y(t))$, normalni vektor: $(\dot{y}(t), -\dot{x}(t))$: $\int_\alpha^\beta (P,Q)\cdot (\dot{x},\dot{y}) \mathrm{d}t = \int_\alpha^\beta (Q,-P)\cdot (\dot{y},-\dot{x}) \mathrm{d}t $

\textbf{Reševanje Dirichletovega problema}

Rešitev je vsota dveh problemov: $\triangle v = 0, v|_{\partial \Omega} = g$ (Poissonov del) in $\triangle w = f, w|_{\partial \Omega} = 0$ (Greenov del).

Poissonov del: $v(x,y) = \int_{\partial \Omega} P(x,y,s)g(s)\ds$, $P:\Omega \times \partial \Omega \rightarrow \R$, $(x,y,s) \mapsto P(x,y,s)$ (\underline{Poissonovo jedro})

Greenov del: $w(x,y) = \int_\Omega G(x,y;\xi,\eta ) f(\xi , \eta )  \mathrm{d}S(\xi,\eta)$, $G: \Omega \times \Omega \backslash \{ (p,p); p\in \Omega \}  \rightarrow \R$, $(x,y;\xi,\eta) \mapsto G(x,y; \xi, \eta)$ (\underline{Greenova funkcija}).

Poissonovo jedro se vedno da izračunati iz Greenove funkcije. Velja zveza: $ P(x,y,t) = \partial_{\vec{n}} G (x,y;\xi , \eta ) |_{(\xi ,\eta ) \in \partial \Omega}$.

\textbf{Kompleksni logaritem}: $\log z = \ln |z| + i \arg z$

\textbf{Fundamentalna rešitev}: $\Gamma (x,y;\xi,\eta) = \frac{1}{4\pi} \ln ((\xi - x)^2 + (\eta - y)^2)=\frac{1}{2\pi} \ln |z-w| = Re ( \frac{1}{2\pi} \log (z-w) )$

Greenove identitete:

\begin{itemize}
\item Za omejeno območje $\Omega \subseteq \R^2$ velja $\int_\Omega u \triangle u \mathrm{d}S + \int_\Omega \nabla u \cdot \nabla u \mathrm{d}S = \int_{\partial \Omega} u(\nabla u \cdot \vec{n})\ds$, kjer je $\vec{n}$ zunanja enotska normala na $\Omega$.
\item Greenova formula za integral vektorskega polja po normali: $\int_\Omega (Q_x - P_y)  \mathrm{d} S = \int_{\partial \Omega} (P,Q)  \mathrm{d}\vec{s} = \int_{\partial \Omega} (Q,-P) \cdot \vec{n} \ds$, kjer $\vec{n}$ enotska normala.
\item Greenova identiteta: $\int_\Omega (u \triangle v - v \triangle u)  \mathrm{d}S = \int_{\partial \Omega} (u \partial_{\vec{n}}v - v \partial_{\vec{n}} u) \ds$, kjer $\partial_{\vec{n}} v = \langle \nabla v , \vec{n} \rangle$ in $\vec{n}$ zunanja normala.
\item  $\int_\Omega \triangle u  \mathrm{d}S = \int_{\partial \Omega}  \partial_{\vec{n}} u \ds$
\item  $\int_\Omega ( \langle \nabla v, \nabla u \rangle + v \triangle u)  \mathrm{d}S = \int_{\partial \Omega} v \partial_{\vec{n}} u \ds$
\end{itemize}

Greenova funkcija za:

\begin{itemize}

\item polravnino $\mathbb{H} = \{ (x,y) \in \R^2, y>0 \} = \{ Im(z) > 0 \}$:
  $G(z,w) = \Gamma (z,w) - \Gamma ( \bar{z},w) =  \frac{1}{4\pi} \log
  \left(\frac{(\xi -x)^2+(\eta -y)^2}{(\xi -x)^2+(\eta +y)^2}\right)$

\item enotski disk $\mathbb{D} = \{ x^2+y^2<1 \}$: $G_{\mathbb{D}} = \frac{1}{2\pi} \ln | \frac{w-z}{1-\bar{z}w} |$

\item pas $\R \times (-1,1)$: $G_{\R \times (-1,1)}(z,w) = \Gamma (e^{\frac{\pi}{2}(z+i)} , e^{\frac{\pi}{2}(w+i)}) - \Gamma (e^{\frac{\pi}{2}(\bar{z}-i)} , e^{\frac{\pi}{2}(w+i)})$

\item disk z radijem $R$ presekan s polravnino $\mathbb{H}$: $G(z, w) = \frac{1}{2 \pi}\log\left|\frac{(z-w)(R^2 - zw)}{(z-\bar{w})(R^2-z\bar{w})} \right|$

\end{itemize}

Poissonovo jedro za:

\begin{itemize}

\item  polravnino $\mathbb{H}$: $ P(x, y, \xi) =\frac{1}{\pi} \frac{y}{  \left((\xi -x)^2+y^2\right)}$

\item enotski disk: $P(r, \varphi, \vartheta) =\frac{1}{2 \pi}  \frac{1-r^2}{1 - 2r \cos(\theta -\varphi) +r^2}$

\end{itemize}

Pas  $\R \times (-1,1)$ s preslikavo $z \mapsto (z+i)\frac{\pi}{2}$ preslikamo v pas  $\R \times (0,\pi)$, tega pa z $e^{x+iy} = e^x(\cos y + i \sin y)$ v polravnino $\mathbb{H} = \{ (x,y) \in \R^2, y>0 \}$ (ker je $y \in (0,\pi)$, je $\sin y > 0$)

Če je $\Omega \subset \C \backslash \mathbb{D}(a,r)$, potem obstaja natanko ena omejena rešitev Dirichletovega problema.

Vsaka rešitev enačbe $\triangle (\xi, \eta) G(x,y;\xi,\eta) = \delta (\xi - x, \eta - y)$, $G|_{(\xi, \eta) \in \partial \Omega} = 0$ je za $(x,y) \in \Omega$ Greenova funkcija.

Naj bo $\Phi \colon \Omega_1 \to \Omega_2$ biholomorfna in rob slika v rok. Če je $G_2$ Greenova funkcija za $\Omega_2$, potem je je $G_1 (z, w) = G_2(\Phi(z), \Phi(w))$ Greenova funkcija za $\Omega_1$. Velja še $\triangle G_1 = \triangle G_2 \det J_{\Phi}$.

%%%%%%%%%%%%%%%%%%%%%%%%%%%%%%%%%%%%%%%%%%%%%%%%%%%%

\vfill \hfill avtor: Klemen Sajovec, popravki: Jure Slak, Vesna Iršič

\end{document}
